\makeatletter
\pgfplotstableset{
  % For filtering data: http://tex.stackexchange.com/a/98016
  discard if not/.style 2 args={
    row predicate/.code={
      \def\pgfplotstable@loc@TMPd{\pgfplotstablegetelem{##1}{#1}\of}
      \expandafter\pgfplotstable@loc@TMPd\pgfplotstablename
      \edef\tempa{\pgfplotsretval}
      \edef\tempb{#2}
      \ifx\tempa\tempb
      \else
        \pgfplotstableuserowfalse
      \fi
      }
    }
  }
\makeatother
\pgfplotsset{
  % For filtering data: http://tex.stackexchange.com/a/98016
  % - Both pgfplots and pgfplotstable must be loaded.
  % - The table file must be specified directly, i.e.:
  %     \addplot table [x = xcol, y = ycol, discard if not = {z}{-4}] {ex.csv};
  %   not using a macro, which will produce no output:
  %     \pgfplotstableread{ex.csv}\excsv
  %     \addplot table [x = xcol, y = ycol, discard if not = {z}{-4}] \excsv;
  discard if not/.style 2 args = {
    x filter/.code = {
      \edef\tempa{\thisrow{#1}}
      \edef\tempb{#2}
      \ifx\tempa\tempb
      \else
        \def\pgfmathresult{}
      \fi
      }},
  % For multiple parallel versions of figures:
  layout/.is choice,
  layout/.initial = default,
  % in a document using this snippet, use the following:
  %
  %layouts/.style n args = {3}{
  %  layout/default/.style = {#1},
  %  layout/b/.style = {#2},
  %  layout/c/.style = {#3},
  %  },
  %
  % to define, for example, three different versions of the document. Then, in
  % the options to a figure, use:
  %
  % layouts = {thick}{very}{ultra thick, blue},
  % layout = c,
  %
  % to choose the style 'ultra thick, blue'.
  }
