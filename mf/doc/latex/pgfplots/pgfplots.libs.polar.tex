\subsection{Polar Axes}
{
\tikzset{external/figure name/.add={}{polar_}}%

\begin{pgfplotslibrary}{polar}
	A library to draw polar axes and plot types relying on polar coordinates, represented by angle (in degrees or, optionally, in radians) and radius.
\end{pgfplotslibrary}

\subsubsection{Polar Axes}
\begin{environment}{{polaraxis}}
	The |polar| library provides the |polaraxis| environment.
	 Inside of such an environment, all coordinates are expected to be given in polar representation of the form $(\meta{angle},\meta{radius})$, i.e.\ the $x$ coordinate is always the angle and the $y$ coordinate the radius:
\end{environment}
\begin{codeexample}[]
\begin{tikzpicture}
	\begin{polaraxis}
	\addplot coordinates {(0,1) (90,1) 
		(180,1) (270,1)};
	\end{polaraxis}
\end{tikzpicture}
\end{codeexample}

\begin{codeexample}[]
\begin{tikzpicture}
	\begin{polaraxis}
	\addplot+[domain=0:3] (360*x,x); % (angle,radius)
	\end{polaraxis}
\end{tikzpicture}
\end{codeexample}

\begin{codeexample}[]
\begin{tikzpicture}
	\begin{polaraxis}
	\addplot+[mark=none,domain=0:720,samples=600] 
		{sin(2*x)*cos(2*x)}; 
	% equivalent to (x,{sin(..)cos(..)}), i.e.
	% the expression is the RADIUS
	\end{polaraxis}
\end{tikzpicture}
\end{codeexample}

Polar axes support most of the \PGFPlots\ user interface, i.e.\ |legend entries|, any axis descriptions, |xtick|/|ytick| and so on:
\begin{codeexample}[]
\begin{tikzpicture}
	\begin{polaraxis}[
		xtick={0,90,180,270},
		title=A polar axis]
	
	\addplot coordinates {(0,1) (45,1)};
	\addlegendentry{First}

	\addplot coordinates {(180,0.5) (0,0)};
	\addlegendentry{Second}
	\end{polaraxis}
\end{tikzpicture}
\end{codeexample}
\noindent Furthermore, you can use all of the supported input coordinate methods (like \verbpdfref{\addplot coordinates}, \verbpdfref{\addplot table}, \verbpdfref{\addplot expression}). The only difference is that polar axes interpret the (first two) input coordinates as polar coordinates of the form $(\meta{angle in degrees},\meta{radius})$.

It is also possible to provide \verbpdfref{\addplot3}; in this case, the third coordinate will be ignored (although it can be used as color data using |point meta=z|). An example can be found below in Section~\ref{sec:polar:cart}.


\subsubsection{Using Radians instead of Degrees}
The initial configuration uses degrees for the angle ($x$ component of every input coordinate). \PGFPlots\ also supports to provide the angle in radians using the |data cs=polarrad| switch:
\begin{codeexample}[]
\begin{tikzpicture}
	\begin{polaraxis}[title={Degrees and/or Radians}]
	\addplot 
		coordinates {(0,1) (90,1) (180,1) (270,1)};
	\addlegendentry{Deg}

	\addplot+[data cs=polarrad] 
		coordinates {(0,1.5) (pi/2,1.5) 
		  (pi,1.5) (pi*3/2,1.5)};
	\addlegendentry{Rad}
	\end{polaraxis}
\end{tikzpicture}
\end{codeexample}
The |data cs| key is described in all detail on page~\pageref{key:data:cs}; it tells \PGFPlots\ the coordinate system of input data. \PGFPlots\ will then take steps to automatically transform each coordinate into the required coordinate system (in our case, this is |data cs=polar|).

\subsubsection{Mixing With Cartesian Coordinates}
\label{sec:polar:cart}
Similarly to the procedure described above, you can also provide Cartesian coordinates inside of a polar axis: simply tell \PGFPlots\ that it should automatically transform them to polar representation by means of |data cs=cart|:
\begin{codeexample}[]
\begin{tikzpicture}
	\begin{polaraxis}[title=Cartesian Input]
	\addplot+[data cs=cart]
		coordinates {(1,0) (0,1) (-1,0) (0,-1)};
	\end{polaraxis}
\end{tikzpicture}
\end{codeexample}
\noindent More details about the |data cs| key can be found on page~\pageref{key:data:cs}.

This does also allow more involved visualization techniques which may operate on Cartesian coordinates. The following example uses \verbpdfref{\addplot3} to sample a function $f\colon \R^2 \to \R$, computes |contour| lines (with the help of |gnuplot|) and displays the result in a |polaraxis|:
\pgfplotsexpensiveexample
\begin{codeexample}[]
\begin{tikzpicture}
	\begin{polaraxis}
	\addplot3[contour gnuplot,domain=-3:3,
	  data cs=cart]
		{exp(-x^2-y^2)};
	\end{polaraxis}
\end{tikzpicture}
\end{codeexample}
\noindent What happens is that $z=\exp(-x^2-y^2)$ is sampled for $x,y \in [-3,3]$, then contour lines are computed on $(x,y,z)$, then the resulting triples $(x,y,z)$ are transformed to polar coordinates $(\alpha,r,z)$ (leaving $z$ intact). Finally, the $z$ coordinate is used as |point meta| to determine the color.

Note that \verbpdfref{\addplot3} allows to process three--dimensional input types, but the result will always be two--dimensional (the $z$ coordinate is ignored for point placement in |polaraxis|). However, the $z$ coordinate can be used to determine point colors (using |point meta=z|).

\subsubsection{Special Polar Plot Types}
\begin{plottype}{polar comb}
	The |polar comb| plot handler is provided by \Tikz; it draws paths from the origin to the designated position and places |mark|s at the positions (similar to the |comb| plot handler). Since the paths always start at the origin, it is particularly suited for |polaraxis|:
\begin{codeexample}[]
\begin{tikzpicture}
	\begin{polaraxis}
	\addplot+[polar comb] 
		coordinates {(300,1) (20,0.3) (40,0.5) 
			(120,1) (200,0.4)};
	\end{polaraxis}
\end{tikzpicture}
\end{codeexample}
\end{plottype}

\subsubsection{Partial Polar Axes}
The |polar| library also supports partial axes. If you provide |xmin|/|xmax|, you can restrict the angles used for the axis:
\begin{codeexample}[]
\begin{tikzpicture}
	\begin{polaraxis}[xmin=45,xmax=360]
	\addplot coordinates {(0,1) (90,1) (180,1) (270,1)};
	\end{polaraxis}
\end{tikzpicture}
\end{codeexample}

Currently, the first angle must be lower than the second one. But you can employ the periodicity to get pies as follows:
\message{Overfull hbox is OK}%
\begin{codeexample}[]
\begin{tikzpicture}
	\begin{polaraxis}[xmin=90,xmax=270]
	\addplot coordinates {(0,1) (90,1) (180,1) (270,1)};
	\end{polaraxis}
\end{tikzpicture}~%
\begin{tikzpicture}
	\begin{polaraxis}[xmin=270,xmax=420]
	\addplot coordinates {(0,1) (90,1) (180,1) (270,1)};
	\end{polaraxis}
\end{tikzpicture}
\end{codeexample}
\noindent Similarly, an explicitly provided value for |ymin| allows to reduce the displayed range away from $0$:
\begin{codeexample}[]
\begin{tikzpicture}
	\begin{polaraxis}[ymin=0.3]
	\addplot coordinates {(0,1) (90,1) 
	  (180,1) (270,1)};
	\end{polaraxis}
\end{tikzpicture}
\end{codeexample}

\noindent Modifying |xmin| and |xmax| manually can also be used to move the $y$ axis line (the line with |ytick| and |yticklabels|):
\begin{codeexample}[]
\begin{tikzpicture}
	\begin{polaraxis}[xmin=45,xmax=405]
	\addplot coordinates {(0,1) (90,1) (180,1) (270,1)};
	\end{polaraxis}
\end{tikzpicture}
\end{codeexample}
}
