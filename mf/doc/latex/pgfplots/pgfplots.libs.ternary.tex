\subsection{Ternary Diagrams}

\begin{pgfplotslibrary}{ternary}
	A library to draw ternary diagrams.

	A ternary diagram visualizes three--component systems such that the sum of them yields $100\%$. Ternary diagrams are triangular axes.
\end{pgfplotslibrary}


\subsubsection{Ternary Axis}
\begin{environment}{{ternaryaxis}\oarg{options}}
	The axis environment for ternary axes.


\begin{codeexample}[]
\begin{tikzpicture}
\begin{ternaryaxis}
	\addplot3 coordinates {
	    (0.81,	0.19,	0.00)
	    (0.76,	0.17,	0.07)
	    (0.66,	0.16,	0.16)
	    (0.76,	0.07,	0.17)
	    (0.81,	0.00,	0.19)
	};

	\addplot3 coordinates {
	    (0.85,	0.15,	0.00)
	    (0.82,	0.13,	0.05)
	    (0.73,	0.14,	0.13)
	    (0.82,	0.06,	0.13)
	    (0.84,	0.00,	0.16)
	};
	\legend{$10$\textdegree, $20$\textdegree}
\end{ternaryaxis}
\end{tikzpicture}
\end{codeexample}

	A |ternaryaxis| works with \emph{relative coordinates}: each data point consists of three components $x$, $y$, $z$. Their sum forms a compound entity which has $100\%$ (of whatever). In the standard configuration, we have $x,y,z \in [0,1]$. The unit interval is not necessary: you can as well choose \emph{absolute data ranges} $x \in [x_{\text{min}},x_{\text{max}}]$, $y \in [y_{\text{min}},y_{\text{max}}]$ and $z \in [z_{\text{min}},z_{\text{max}}]$. The important thing is that the relative values 
	\[ \tilde x := \frac{x-x_{\text{min}}}{x_{\text{max}} - x_{\text{min}}},\;
	   \tilde y := \frac{y-y_{\text{min}}}{y_{\text{max}} - y_{\text{min}}},\;
	   \tilde z := \frac{z-z_{\text{min}}}{z_{\text{max}} - z_{\text{min}}} \]
	sum up to $100\%$, i.e.\ $\tilde x + \tilde y + \tilde z = 1$. Thus, \PGFPlots\ computes $\tilde x$, $\tilde y$ and $\tilde z$ and interpretes them as barycentric (triangular) coordinates.

	For this to work, it is \textbf{crucial to provide } |xmin|, |xmax|, |ymin|, |ymax| and |zmin|, |zmax| precisely! The initial configuration fixes them to the unit interval.

	What happens behind the scenes is that a data point $(x,y,z)$ is placed at $X,Y$ determined by
	\[ 
		\begin{bmatrix}
			X(x,y,z)\\
			Y(x,y,z)
		\end{bmatrix}
		=
		\tilde x A + \tilde y B + \tilde z C
		= 
		\begin{bmatrix}
			\frac12 {\tilde x+2 \tilde z}\\
			\frac{\sqrt 3}{2} {\tilde x}
			%\frac12 \frac{\tilde x+2 \tilde z}{\tilde x+\tilde y+\tilde z}\\
			%\frac{\sqrt 3}{2} \frac{\tilde x}{\tilde x+\tilde y+\tilde z}
		\end{bmatrix}
	\]
	where $A=(\nicefrac12,\nicefrac{\sqrt3}{2})$ is top corner of the triangle, $B=(0,0)$ the lower left and $C=(1,0)$ the lower right one. The $\tilde y$ component is not really necessary due to the linear dependency $\tilde x+\tilde y+\tilde z=1$.

	The input coordinate $(100\%,0\%,0\%)$ is mapped to $A$, the input coordinate $(0\%,100\%,0\%)$ to $B$ and $(0\%,0\%,100\%)$ to $C$ (Acrobat Reader: click into the axis to verify it).

\begin{codeexample}[]
\begin{tikzpicture}
\begin{ternaryaxis}[xlabel=A,ylabel=B,zlabel=C]
	\addplot3 coordinates {
        (0.81,  0.19,  0.00)
        (0.76,  0.17,  0.07)
        (0.66,  0.16,  0.16)
        (0.76,  0.07,  0.17)
        (0.81,  0.00,  0.19)
	};

	\addplot3 coordinates {
        (0.85,  0.15,  0.00)
        (0.82,  0.13,  0.05)
        (0.73,  0.14,  0.13)
        (0.82,  0.06,  0.13)
        (0.84,  0.00,  0.16)
	};

	\node[pin=130:Deduced $z$,draw=black] at (axis cs:0.2,0.2) {};

	\legend{$10$\textdegree, $20$\textdegree}
\end{ternaryaxis}
\end{tikzpicture}
\end{codeexample}
\end{environment}

	A |ternaryaxis| can contain zero, one or more |\addplot3| commands, just as a usual |axis|. In case you provide only two--dimensional coordinates (for example using |\addplot| or |axis cs|), the third component is deduced automatically such that components sum to $100\%$. The |\addplot3| command can use any of the accepted input formats, for example using |coordinates|, |table|, |expression| or whatever -- but the input is always interpreted as barycentric coordinates (three components summing up to $100\%$).

\begin{codeexample}[]
\begin{tikzpicture}
\begin{ternaryaxis}[
	title=Sloped labels and minor ticks,
	xlabel=Water,
	ylabel=D--Threonine,
	zlabel=L--Threonine,
	label style={sloped},
	minor tick num=2,
]
	\addplot3 coordinates {
        (0.82,  0.18,  0.00)
        (0.75,  0.17,  0.08)
        (0.77,  0.12,  0.11)
        (0.75,  0.08,  0.17)
        (0.81,  0.00,  0.19)
	};
	\addplot3 coordinates {
        (0.75,  0.25,  0.00)
        (0.69,  0.25,  0.06)
        (0.64,  0.24,  0.12)
        (0.655, 0.23,  0.115)
        (0.67,  0.17,  0.16)
        (0.66,  0.12,  0.22)
        (0.64,  0.11,  0.25)
        (0.69,  0.05,  0.26)
        (0.76,  0.01,  0.23)
	};
\end{ternaryaxis}
\end{tikzpicture}
\end{codeexample}

\begin{codeexample}[]
\begin{tikzpicture}
\begin{ternaryaxis}[
	title=Sloped labels and minor grids,
	xlabel=Water,
	ylabel=D--Threonine,
	zlabel=L--Threonine,
	label style={sloped},
	minor tick num=2,
	grid=both,
]
	\addplot3 coordinates {
        (0.82,  0.18,  0.00)
        (0.75,  0.17,  0.08)
        (0.77,  0.12,  0.11)
        (0.75,  0.08,  0.17)
        (0.81,  0.00,  0.19)
	};
	\addplot3 coordinates {
        (0.75,  0.25,  0.00)
        (0.69,  0.25,  0.06)
        (0.64,  0.24,  0.12)
        (0.655, 0.23,  0.115)
        (0.67,  0.17,  0.16)
        (0.66,  0.12,  0.22)
        (0.64,  0.11,  0.25)
        (0.69,  0.05,  0.26)
        (0.76,  0.01,  0.23)
	};
\end{ternaryaxis}
\end{tikzpicture}
\end{codeexample}

\noindent
A |ternaryaxis| supports (most of) the \PGFPlots\ axis interface, among them the |grid| option, the |xtick=|\marg{positions} way to provide ticks, including |extra x ticks| and its variants. Of course, it can also contain any of the |mark|, |color| and |cycle list| options of a normal axis.

\noindent The following example is a (crude) copy of an example of 

\url{http://www.sv.vt.edu/classes/MSE2094_NoteBook/96ClassProj/experimental/ternary2.html}

\noindent and uses |area style| to change |cycle list| and the legend appearance.

\begin{codeexample}[]
\begin{tikzpicture}
\begin{ternaryaxis}[
	title=Want--be--Stainless Steel,
	xlabel=Weight Percent Chromium,
	ylabel=Weight Percent Iron,
	zlabel=Weight Percent Nickel,
	label style=sloped,
	area style,
]
	\addplot3 table {
	A B C
	1 0 0
	0.5 0.4 0.1
	0.45 0.52 0.03
	0.36 0.6 0.04
	0.1 0.9 0
	};
	\addlegendentry{Cr}

	\addplot3 table {
	A B C
	1 0 0
	0.5 0.4 0.1
	0.28 0.35 0.37
	0.4 0 0.6
	};
	\addlegendentry{Cr+$\gamma$FeNi}

	\addplot3 table {
	0.4 0 0.6
	0.28 0.35 0.37
	0.25 0.6 0.15
	0.1 0.9 0
	0 1 0
	0 0 1
	};
	\addlegendentry{$\gamma$FeNi}

	\addplot3 table {
	0.1 0.9 0
	0.36 0.6 0.04
	0.25 0.6 0.15
	};
	\addlegendentry{Cr+$\gamma$FeNi}

	\addplot3 table {
	0.5 0.4 0.1
	0.45 0.52 0.03
	0.36 0.6 0.04
	0.25 0.6 0.15
	0.28 0.35 0.37
	};
	\addlegendentry{$\sigma$+$\gamma$FeNi}

	\node[inner sep=0.5pt,circle,draw,fill=white,pin=-15:\footnotesize Stainless Steel] 
	  at (axis cs:0.18,0.74,0.08) {};
	
\end{ternaryaxis}
\end{tikzpicture}
\end{codeexample}

Ternary plots can also use |contour prepared| to plot contour lines. 
The following example is a (crude) copy of an example of 

\url{http://www.sv.vt.edu/classes/MSE2094_NoteBook/96ClassProj/experimental/ternary2.html}:

\begin{codeexample}[]
\begin{tikzpicture}
\begin{ternaryaxis}[
	title=Want--be--Stainless Steel,
	xlabel=Weight Percent Chromium,
	ylabel=Weight Percent Iron,
	zlabel=Weight Percent Nickel,
	label style=sloped,
]

% plotdata/pgfplotsternary.example1.dat:
%
% Chromium Iron Nickel Temperature
% 0.90	0.0     0.10	1700
% 0.85	0.14	0.00	1700
%
% 0.85	0.00	0.15	1600
% 0.78	0.22	0.00	1600
% 0.71	0.29	0.00	1600
% ....
\addplot3[contour prepared={labels over line},
		point meta=\thisrow{Temperature}]
	table[x=Chromium,y=Iron,z=Nickel]
	{plotdata/pgfplotsternary.example1.dat};
\end{ternaryaxis}
\end{tikzpicture}
\end{codeexample}
\noindent \label{page:ternary:contour}The |contour prepared={labels over line}| installs the display style |contour/labels over line| and expects precomputed contour lines from the input stream. Here, the input stream is a table, consisting of the three relative components for Chromium, Iron and Nickel -- and the |point meta| is set to be the Temperature column. The |contour prepared| style uses the $(x,y,z)$ coordinate to plot the data point and the |point meta| to determine contour labels (the initial configuration of |contour prepared| is to use |point meta=z|). The output thus allows to use both barycentric coordinates (ternary components) \emph{and} contour labels.

\begin{pgfplotskeylist}{%
	ternary limits relative=\mchoice{true,false} (initially true),
	ternary relative limits=\mchoice{true,false} (initially true)}
	Allows to switch tick labels between relative numbers in the range $[0,100]$ or absolute numbers.
	
	The choice |ternary limits relative=true| accepts data in any input number range, for example $(x,y,z) \in [0,1]^3$, or $(x,y,z) \in [0,100]^3$ or in any absolute scala of the form $x_i \in [\underline x_i,\overline x_i]$ for $x_i \in \{x,y,z\}$ (remember that it is crucial to communicate these limits to \PGFPlots\ explicitly using |xmin|, |xmax|, |ymin|, |ymax| and |zmin|, |zmax| such that relative coordinates can be computed, see the description above for details). In every case, relative tick labels are drawn, i.e.\ tick labels in the range $[0,100]$.
\begin{codeexample}[]
\begin{tikzpicture}
\begin{ternaryaxis}[
	ternary limits relative,
	title={Data range $[0,1]$, limits relative},
	area style]
\addplot3 coordinates {
	(0.2,0.8,0)
	(0.31,0.4,0.29)
	(0.34,0.2,0.46)
	(0.4,0,0.6)
	(1,0,0)
};
\addplot3 coordinates {
	(0.4,0,0.6)
	(0.34,0.2,0.46)
	(0.31,0.4,0.29)
	(0.14,0.46,0.4)
	(0,0.37,0.63)
	(0,0,1)
};
\node[fill=white] 
	at (axis cs:0.56,0.28,0.16) {$F 42$};
\node[fill=white] 
	at (0.7,0.2) {$F 43$};
\end{ternaryaxis}
\end{tikzpicture}
\end{codeexample}

\begin{codeexample}[]
\begin{tikzpicture}
\begin{ternaryaxis}[
	xmax=500,ymin=1,ymax=2,
	ternary limits relative,
	title={Data range $x\in[0,500]$, 
		$y\in[1,2]$, $z\in[0,1]$ limits relative},
	area style]
\addplot3 coordinates {
	(100,1.8,0)
	(155,1.4,0.29)
	(170,1.2,0.46)
	(200,1,0.6)
	(500,1,0)
};
\addplot3 coordinates {
	(200,1,0.6)
	(170,1.2,0.46)
	(155,1.4,0.29)
	(70,1.46,0.4)
	(0,1.37,0.63)
	(0,1,1)
};
\node[fill=white] 
	at (axis cs:280,1.28,0.16) {$F 42$};
\node[fill=white] 
	at (0.7,0.2) {$F 43$};
\end{ternaryaxis}
\end{tikzpicture}
\end{codeexample}

	The choice |ternary limits relative=false| accepts the same data ranges, but it draws tick labels in the very same data ranges.
\begin{codeexample}[]
\begin{tikzpicture}
\begin{ternaryaxis}[
	ternary limits relative=false,
	xmax=500,ymin=1,ymax=2,
	title={Data range $x\in[0,500]$, 
		$y\in[1,2]$, $z\in[0,1]$ limits absolute},
	footnotesize, % just for the sake of demonstration...
	area style]
\addplot3 coordinates {
	(100,1.8,0)
	(155,1.4,0.29)
	(170,1.2,0.46)
	(200,1,0.6)
	(500,1,0)
};
\addplot3 coordinates {
	(200,1,0.6)
	(170,1.2,0.46)
	(155,1.4,0.29)
	(70,1.46,0.4)
	(0,1.37,0.63)
	(0,1,1)
};
\node[fill=white] 
	at (axis cs:280,1.28,0.16) {$F 42$};
\node[fill=white] 
	at (0.7,0.2) {$F 43$};
\end{ternaryaxis}
\end{tikzpicture}
\end{codeexample}
\end{pgfplotskeylist}

\begin{coordinatesystem}{cartesian cs}
	A coordinate system which allows Cartesian coordinates. The lower left point has coordinate $(0,0)$, the lower right point has $(1,0)$ and the upper point of the triangle is at $(\nicefrac12, \nicefrac{\sqrt3}{2})$.

	If you use the standard point syntax $(x,y)$ in path commands inside of the axis, you'll get Cartesian coordinates. If you want to use it for axis descriptions (like |xlabel|), you'll have to write |cartesian cs:0,0| explicitly (axis labels have the default coordinate system |axis description cs|).
\begin{codeexample}[]
\begin{tikzpicture}
	\begin{ternaryaxis}[
		title=Cartesian Annotations,
		clip=false]

	\addplot3 coordinates {
		(0.1,0.5,0.4)
		(0.2,0.5,0.3)
		(0.3,0.6,0.1)
	};

	\node[fill=white,draw] at (0,0) {$y (0,0)$};
	\node[fill=white,draw] at (1,0) {$z (1,0)$};
	\node[fill=white,draw] at (0.5,{sqrt(3)/2}) 
		{$x (\frac12,\frac{\sqrt3}{2})$};
	
	\draw[red,-stealth] (0.5,0) -- (0.5,0.7);
	\end{ternaryaxis}
\end{tikzpicture}
\end{codeexample}
\end{coordinatesystem}

\begin{stylekey}{/pgfplots/every ternary axis}
	A style which is installed at the beginning of every ternary axis. It is used to adjust some of the \PGFPlots\ keys to fit the triangular shape.

	The initial configuration is
\begin{codeexample}[code only]
\pgfplotsset{
	every ternary axis/.style={
		tick align=outside,
		grid=major,
		xticklabel style={anchor=west},
		every 3d description/.style={},
		every axis x label/.style={at={(ticklabel cs:0.5)},anchor=near ticklabel},
		every axis y label/.style={at={(ticklabel cs:0.5)},anchor=near ticklabel},
		every axis z label/.style={at={(ticklabel cs:0.5)},anchor=near ticklabel},
		every x tick scale label/.style=
		  {at={(xticklabel cs:0.95,5pt)},anchor=near xticklabel,inner sep=0pt},
		every y tick scale label/.style=
		  {at={(yticklabel cs:0.95,5pt)},anchor=near yticklabel,inner sep=0pt},
		every z tick scale label/.style=
		  {at={(yticklabel cs:0.95,5pt)},anchor=near yticklabel,inner sep=0pt},
		every axis title shift=15pt,
		every axis legend/.style={
			cells={anchor=center},
			inner xsep=3pt,inner ysep=2pt,nodes={inner sep=2pt,text depth=0.15em},
			shape=rectangle,
			fill=white,
			draw=black,
			at={(1.03,1.03)},
			anchor=north west,
		},
		annot/point format 3d/.initial={(\%.2f, \%.2f, \%.2f)},
	},
}
\end{codeexample}
\end{stylekey}

\subsubsection{Tieline Plots}
\begin{plottype}{tieline=\marg{options with {\normalfont tieline/} prefix}}
\pgfkeys{/pdflinks/search key prefixes in/.add={/pgfplots/tieline/,}{}}
	A plot handler for use in ternary diagrams which plots tie lines and binodal curves.

	On input, it accepts \emph{pairs} of coordinates, $A^{(i)}=(A_x^{(i)},A_y^{(i)},A_z^{(i)})$ and $B^{(i)}=(B_x^{(1)},B_y^{(2)},B_z^{(3)})$, for $i=1,\dotsc,N$ (i.e.\ it requires a total of six coordinates, perhaps plus additional color data).

	On output, it connects the pairs, i.e.\ for every fixed $i=1,\dotsc,N$, it connects $A^{(i)}$ --- $B^{(i)}$ (the so-called ``tie lines''). In addition, it also draws the binodal curve, which is made up by connecting all $A^{(i)}$ and then, in reverse ordering, all $B^{(i)}$: $A^{(1)}$ --- $ A^{(2)}$ --- $ \dotsb $ --- $ A^{(N)}$ --- $B^{(N)}$ --- $ B^{(N-1)}$ --- $ \dotsb $ --- $B^{(1)}$:
\begin{codeexample}[]
\begin{tikzpicture}
\begin{ternaryaxis}[
	xlabel=x (IPA),
	ylabel=y (water),
	zlabel=z (propene),
	axis on top,
]
% plotdata/ternary_data.txt is a table of the form
%A_propene A_water A_IPA  B_propene B_water B_IPA
% 0.0009   0.9990  0      0.9333    0.0667  0
% 0.0009   0.9988  0.0002 0.9303    0.0665  0.0032
% 0.0011   0.9975  0.0013 0.9135    0.0673  0.0191
% 0.0013   0.9962  0.0024 0.8956    0.0693  0.0351
%...
	\addplot3[tieline,fill=blue!10]
	table [x=A_IPA,y=A_water,z=A_propene] 
		{plotdata/ternary_data.txt};
\end{ternaryaxis}
\end{tikzpicture}
\end{codeexample}

	We see that each input line has six columns, and each six columns are taken into account (this is different from other plot handlers!). The six columns make up the three components of the $A$ and $B$ points, respectively. In the example above, we used explicit column names and provided $A_x$ using |x=A_IPA|, $A_y$ using |y=A_water| and $A_z$ using |z=A_propene|. Note that these keys are the common input method for |\addplot table|; they are nothing special (that means we could also use |x index| instead). The three columns for $B$ can be provided manually (see below), or deduced automatically: in our case, the value for $B_x$ has been found in the third column after |x=A_IPA| (which is |B_IPA|); the value for $B_y$ has been found in the third column after |y=A_water| and $B_z$ is made up from the third column after |z=A_propene|. In other words, the $B$ value is searched (by default) by adding $3$ to the column index of the respective $A$ coordinate. 
	
	You do not need to provide \emph{any} column names; in this case, the first three columns make up $A$ (in the order of appearance) and the following three make up $B$.

	The only supported input type for |tieline| plots is table input. It is optimized to use |\addplot3 table| (as described above). To use the two--dimensional variant |\addplot table|, you need to tell \PGFPlots\ explicitly which columns make up $A_x,A_y,B_x,B_y$; the $z$ coordinates are deduced automatically such that the result sums to $100\%$.
\begin{pgfplotsxykeylist}{%
	table/tie end \x=\marg{colname} (initially empty),%
	table/tie end \x\space index=\marg{col index} (initially empty)}
	These keys can be used to provide column names or column indices for $B_x$, $B_y$ and $B_z$, respectively. They can be provided like 

	|\addplot3[tieline] table[tie end y=B_water] ...|.

	Note that the |tie end x| keys are \emph{only} available if the |tieline| option has been used before.

	The values for $A$ are provided with |table/x|, |table/x index| and its variants as for any other plot type.
\end{pgfplotsxykeylist}


	The |tieline| plot handler accepts several options to customize the appearance. You can provide them as argument after |tieline|, using |tieline=|\marg{options}. In this case, the |tieline/| prefix can be omitted. The keys are described in the following:
\pgfkeys{/pgfmanual/gray key prefixes=/pgfplots/tieline}
\begin{pgfplotskey}{tieline/each nth tie=\marg{number} (initially empty)}
	Allows to draw only each $n$th tie line, even though the binodal curve uses all provided coordinates:
\begin{codeexample}[]
\begin{tikzpicture}
\begin{ternaryaxis}[
	xlabel=x (IPA),
	ylabel=y (water),
	zlabel=z (propene),
	axis on top,
]
% plotdata/ternary_data.txt is a table of the form
%A_propene A_water A_IPA  B_propene B_water B_IPA
% 0.0009   0.9990  0      0.9333    0.0667  0
% 0.0009   0.9988  0.0002 0.9303    0.0665  0.0032
% 0.0011   0.9975  0.0013 0.9135    0.0673  0.0191
% 0.0013   0.9962  0.0024 0.8956    0.0693  0.0351
%...
	\addplot3[
		tieline={each nth tie=5},
		fill=blue!10,
	]
	table [x=A_IPA,y=A_water,z=A_propene] 
		{plotdata/ternary_data.txt};
\end{ternaryaxis}
\end{tikzpicture}
\end{codeexample}

	Note that plot |mark|s (if any) are drawn on every input position, use the |mark repeat| option to change that.
\end{pgfplotskey}

\begin{pgfplotskey}{tieline/tieline style=\marg{options}}
	Appends \meta{options} to the style \declareandlabel{tieline/every tieline}.

	Useful \meta{options} are, for example, other plot handlers to adjust the appearance of tie lines. Suppose that you have additional color data for every tie line (which might have been provided as further input column). In our case, we provide random color data using |point meta=rand|, and visualize the single tielines as with |contour prepared|:
\begin{codeexample}[]
\begin{tikzpicture}
\begin{ternaryaxis}[
	xlabel=x (IPA),
	ylabel=y (water),
	zlabel=z (propene),
	axis on top,
]
% plotdata/ternary_data.txt is a table of the form
%A_propene A_water A_IPA  B_propene B_water B_IPA
% 0.0009   0.9990  0      0.9333    0.0667  0
% 0.0009   0.9988  0.0002 0.9303    0.0665  0.0032
% 0.0011   0.9975  0.0013 0.9135    0.0673  0.0191
% 0.0013   0.9962  0.0024 0.8956    0.0693  0.0351
%...
	\addplot3[
		point meta=rand,
		tieline={
			each nth tie=8,
			tieline style={contour prepared}
		},
		fill=blue!10,
	]
	table [x=A_IPA,y=A_water,z=A_propene] 
		{plotdata/ternary_data.txt};
\end{ternaryaxis}
\end{tikzpicture}
\end{codeexample}
	\noindent The effect here is that contour labels and line colors are chosen for every tie line, where the actual color is determined using |point meta| and |colormap|. Other choices for plot handlers in |tieline style| might be the |mesh|.
\end{pgfplotskey}

\begin{pgfplotskey}{tieline/curve style=\marg{options}}
	Appends \meta{options} to the style \declareandlabel{tieline/every curve}.

	The |curve style| allows to customize the plot handler for the curve. A possible choice might be |curve style={smooth}| or a separate fill/draw color.
\end{pgfplotskey}
\end{plottype}
