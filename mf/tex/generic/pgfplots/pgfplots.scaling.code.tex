%--------------------------------------------
%
% Package pgfplots
%
% Provides a user-friendly interface to create function plots (normal
% plots, semi-logplots and double-logplots).
% 
% It is based on Till Tantau's PGF package.
%
% Copyright 2007-2012 by Christian Feuersänger.
%
% This program is free software: you can redistribute it and/or modify
% it under the terms of the GNU General Public License as published by
% the Free Software Foundation, either version 3 of the License, or
% (at your option) any later version.
% 
% This program is distributed in the hope that it will be useful,
% but WITHOUT ANY WARRANTY; without even the implied warranty of
% MERCHANTABILITY or FITNESS FOR A PARTICULAR PURPOSE.  See the
% GNU General Public License for more details.
% 
% You should have received a copy of the GNU General Public License
% along with this program.  If not, see <http://www.gnu.org/licenses/>.
%
%--------------------------------------------

% PRECONDITION: 
% 	- final axis limits are given in transformed range
% 	-  \pgfplots@set@default@size@options has been invoked before
% POSTCONDITION: 
% 	- the current x,y and z unit vectors are defined properly;
% 	- the fast-access registers are initialised for the axis limits,
%	- the following macros are assigned:
%		\pgfplots@[xyz]@veclength
%		\pgfplots@[xyz]@inverseveclength
%		\pgfplotspointxaxis
%		\pgfplotspointyaxis
%		\pgfplotspointzaxis
%		\pgfplotspointcenter
%		\pgfplotspointminminmin
%
\def\pgfplots@initsizes{%
	% INIT.
	%
	%
	\pgfplots@xmin@reg=\pgfplots@xmin pt %
	\pgfplots@xmax@reg=\pgfplots@xmax pt %
	\pgfplots@ymin@reg=\pgfplots@ymin pt %
	\pgfplots@ymax@reg=\pgfplots@ymax pt %
	\ifpgfplots@threedim
		\pgfplots@zmin@reg=\pgfplots@zmin pt %
		\pgfplots@zmax@reg=\pgfplots@zmax pt %
	\fi
	%
	%-----------------------------------------
	% PROCESS THE 'width' and 'height' options
	%-----------------------------------------
	%
	%
	\pgfkeysgetvalue{/pgfplots/view/az}{\pgfplots@view@az}%
	\pgfkeysgetvalue{/pgfplots/view/el}{\pgfplots@view@el}%
	\ifpgfplots@threedim
	\else
		\let\pgfplots@view@el=\pgfutil@empty
		\let\pgfplots@view@az=\pgfutil@empty
	\fi
	\ifx\pgfplots@view@az\pgfutil@empty
		% Note that in presence of "x,y,z" options, the
		% \pgfplots@set@default@size@options RESETS /pgfplots/view/az.
		%
		%\let\pgfplots@rectangle@width=\pgfutil@empty
		%\let\pgfplots@rectangle@height=\pgfutil@empty
		\pgfplotsmathvectorfromstring{0,0,1}{default}%
		\let\pgfplots@view@dir@threedim=\pgfplotsretval
		%
		\ifx\pgfplots@x\pgfutil@empty
			\ifx\pgfplots@width\pgfutil@empty
				\pgfplots@error{INTERNAL LOGIC ERROR! WIDTH NOT SET}%
			\fi
		\fi
		%
		\ifx\pgfplots@y\pgfutil@empty
			\ifx\pgfplots@height\pgfutil@empty
				\pgfplots@error{INTERNAL LOGIC ERROR! HEIGHT NOT SET}%
			\fi
		\fi
		\ifpgfplots@threedim
			\def\pgfplots@tmp@Zscale{1}%
		\else
			\def\pgfplots@tmp@Zscale{0}%
		\fi
		%
		\pgfplots@initsizes@setunitvector{x}{0}{1}{\pgfplots@tmp@xisaxisparallel}%
		\pgfplots@initsizes@setunitvector{y}{1}{1}{\pgfplots@tmp@yisaxisparallel}%
		\pgfplots@initsizes@setunitvector{z}{2}{\pgfplots@tmp@Zscale}{\pgfplots@loc@TMPc}%
		%
		\pgfplots@scaling@changewidthheight@for@enlargelimits@by@dimension%
		%
		\pgfplots@scale@plotbox@to@widthheight
	\else
		% 3D case by `view':
		\let\pgfplots@x=\pgfutil@empty
		\let\pgfplots@y=\pgfutil@empty
		\let\pgfplots@z=\pgfutil@empty
		\pgfplotssetaxesfromazel{\pgfplots@view@az}{\pgfplots@view@el}{\pgfplots@tmp@xisaxisparallel}%
		%
		\pgfplots@scaling@changewidthheight@for@enlargelimits@by@dimension%
		%
		\pgfplots@scale@plotbox@to@widthheight
		\if1\pgfplots@tmp@xisaxisparallel%
			\def\pgfplots@tmp@yisaxisparallel{1}%
		\fi
	\fi
	\pgfplots@computeunitvectorlengths
	%
	\pgfplots@scaling@apply@enlargelimits@by@dimension{x}%
	\pgfplots@scaling@apply@enlargelimits@by@dimension{y}%
	\ifpgfplots@threedim
		\pgfplots@scaling@apply@enlargelimits@by@dimension{z}%
	\fi
	%
	\ifpgfplots@threedim
		\pgfplotsgetnormalforcurrentview
	\fi
%\message{Pgfplots debug: initialised unit vectors to x=(\the\pgf@xx,\the\pgf@xy), y=(\the\pgf@yx,\the\pgf@yy), z=(\the\pgf@zx,\the\pgf@zy), n = (\pgfplots@view@dir@threedim). Limits are x=\pgfplots@xmin:\pgfplots@xmax, y=\pgfplots@ymin:\pgfplots@ymax^^J }%
	%
	\let\pgfplotsmathfloatviewdepthxyz@=\pgfplotsmathfloatviewdepthxyz@infigure
	\let\pgfplotsmathviewdepthxyz@=\pgfplotsmathviewdepthxyz@infigure
	%
	\pgfplotsmath@ifzero{\pgfplots@x@veclength}{\pgfplots@hide@xtrue\pgfplots@shownothingof@xtrue}{}%
	\pgfplotsmath@ifzero{\pgfplots@y@veclength}{\pgfplots@hide@ytrue\pgfplots@shownothingof@ytrue}{}%
	\ifpgfplots@threedim
		\pgfplotsmath@ifzero{\pgfplots@z@veclength}{\pgfplots@hide@ztrue\pgfplots@shownothingof@ztrue}{}%
	\else
		\if1\pgfplots@tmp@xisaxisparallel%
			\if1\pgfplots@tmp@yisaxisparallel%
				% Optimize for axis-parallel case!
				% puh. Did not make any measureable difference!? Ok...
				\let\pgfplotsqpointxy=\pgfplotsqpointxy@orthogonal
			\fi
		\fi
	\fi
}

% Defines \pgfmathresult to be the desired width without axis labels.
%
\def\pgfplots@initsizes@get@width@withoutlabels{%
	\pgfplots@initsizes@handle@label@const{\pgfplots@width}{45pt}{width}%
}

% Defines \pgfmathresult to be the desired height without axis labels.
\def\pgfplots@initsizes@get@height@withoutlabels{%
	\pgfplots@initsizes@handle@label@const{\pgfplots@height}{45pt}{height}%
}

\def\pgfplots@initsizes@handle@label@const#1#2#3{%
	\begingroup
	\pgf@xa=#1\relax
	% EXPECTED WIDTH = X = \pgfplots@width
	% ACTUAL WIDTH = c + x * (xmax-xmin)
	% where c is a CONSTANT (for the axis labels/tick labels).
	% -> \pgfplots@tmpXscale = (X - c) / (x *(xmax-xmin))
	%
	% \pgf@xa := X-c:
	\ifpgfplots@scale@only@axis
	\else
		\advance\pgf@xa by-#2 % FIXME determine 'c' correctly!
	\fi
	\ifdim\pgf@xa<0pt
		\pgfplots@error{Error: Plot #3 `#1' is too small. This cannot be implemented while maintaining constant size for labels. Sorry, label sizes are only approximate. You will need to adjust your #3.}%
		\pgf@xa=0pt
	\fi
	\edef\pgfmathresult{\the\pgf@xa}%
	\pgfmath@smuggleone\pgfmathresult
	\endgroup
}%

% #1: axis
\def\pgfplots@axis@apply@post@scale#1{%
	%
	\pgfkeysgetvalue{/pgfplots/#1 post scale}\pgfplots@loc@TMPd
	\ifx\pgfplots@loc@TMPd\pgfutil@empty
	\else
		\pgfmathparse{\csname pgfplots@target@unit@scale@#1x\endcsname*\pgfplots@loc@TMPd}%
		\expandafter\let\csname pgfplots@target@unit@scale@#1x\endcsname=\pgfmathresult
		%
		\pgfmathparse{\csname pgfplots@target@unit@scale@#1y\endcsname*\pgfplots@loc@TMPd}%
		\expandafter\let\csname pgfplots@target@unit@scale@#1y\endcsname=\pgfmathresult
	\fi
}

% Takes azimuth (horizontal angle) '#1' and elongation (vertical
% angle) '#2' (both in degrees) and computes 
% x,y and z vectors which define the view in the direction
% defined by '#1' and '#2'.
%
% 'azimuth' means a rotation around the viewport's x axis. 'elongation' means
% a rotation around the original coordinate system's z axis.
%
% The method works by computing 
% Az = [ cos(azimuth) -sin(azimuth) 0; ...
%     sin(azimuth) cos(azimuth) 0; ...
%     0 0 1 ];
% 
%
% Ax = [ 1 0 0; ...
%     0 cos(elevation) -sin(elevation) ;...
%     0 sin(elevation) cos(elevation) ];
%
% v= Ax * Az;
% = [ ...
% 	cosaz  -sinaz cosel  sinaz sinel; ...
% 	sinaz  cosaz cosel   -sinel cosaz; ...
% 	0	  sinel         cosel ];
%
% Then, we use the rotated XZ plane as viewport, that means 
%   xvec = v * [1 0 0]' = <first column of v>
%   zvec = v * [0 0 1]' = <third column of v>
% and we define the projection onto the twodimensional surface
% spanned by 'xvec' and 'zvec' as
%   P( q ) = [ q^T xvec,  q^T zvec ]'
% for q in R^3.
% As a consequence, we compute the three unit vectors as
%  x = P( [1 0 0] )
%  	 = [ cosaz,  sinaz sinel ]'
%  y = P( [0 1 0] )
%    = [ sinaz,  -sinel cosaz ]'
%  z = P( [0 0 1] )
%    = [ 0,  cosel]'
%
% Furthermore, the 3D view vector which points into the direction of the view
% is
%   n = v * [0 1 0 ]' = <second column of v> =  [-sinaz cosel,  cosaz cosel,  sinel]' 
% because the normal view point was the XZ plane with y as its normal
% vector.
% The 3D vector n is returned by this routine as well - it is
% necessary for some kind of z buffering (determining what is
% foreground and what is background).
%
% INPUT:
% - #1 : azimuth ("yaw")
% - #2 : elevation ("pitch")
% OUTPUT:
% - #3 : a macro which will be set to '1' if and only if 
%      the viewport is the standard XY axis (i.e. azimuth=0, elevation=90).
% - [xyz] vectors,
%   \pgfplots@view@dir@threedim will contain the three components
%   of 'n' (without the suffix 'pt', but in units of 'pt') (see
%   \pgfplotsmathvectorfromstring).
\def\pgfplotssetaxesfromazel#1#2#3{%
	\begingroup
	\pgfmathparse{#1}%
	\let\pgfplots@az=\pgfmathresult
	\pgfmathparse{#2}%
	\edef\pgfplots@el{-\pgfmathresult}%
	\pgfmathsin@{\pgfplots@az}%
	\let\sinaz=\pgfmathresult
	\pgfmathcos@{\pgfplots@az}%
	\let\cosaz=\pgfmathresult
	\pgfmathsin@{\pgfplots@el}%
	\let\sinel=\pgfmathresult
	\pgfmathcos@{\pgfplots@el}%
	\let\cosel=\pgfmathresult
	% x:
	\pgfmathmultiply@{\sinaz}{\sinel}%
	\xdef\pgfplots@glob@TMPa{\noexpand\pgfqpoint{\cosaz pt}{\pgfmathresult pt}}%
	% y:
	\pgfmathmultiply@{-\sinel}{\cosaz}%
	\xdef\pgfplots@glob@TMPb{\noexpand\pgfqpoint{\sinaz pt}{\pgfmathresult pt}}%
	% z:
	\xdef\pgfplots@glob@TMPc{\noexpand\pgfqpoint{0pt}{\cosel pt}}%
	%
	\pgfkeysgetvalue{/pgfplots/x dir/value}\pgfplots@loc@dirvalue@x
	\pgfkeysgetvalue{/pgfplots/y dir/value}\pgfplots@loc@dirvalue@y
	\pgfkeysgetvalue{/pgfplots/z dir/value}\pgfplots@loc@dirvalue@z
	\if r\pgfplots@loc@dirvalue@x
		\t@pgfplots@toka=\expandafter{\pgfplots@glob@TMPa}%
		\xdef\pgfplots@glob@TMPa{\noexpand\pgfqpointscale{-1}{\the\t@pgfplots@toka}}%
	\fi
	\if r\pgfplots@loc@dirvalue@y
		\t@pgfplots@toka=\expandafter{\pgfplots@glob@TMPb}%
		\xdef\pgfplots@glob@TMPb{\noexpand\pgfqpointscale{-1}{\the\t@pgfplots@toka}}%
	\fi
	\if r\pgfplots@loc@dirvalue@z
		\t@pgfplots@toka=\expandafter{\pgfplots@glob@TMPc}%
		\xdef\pgfplots@glob@TMPc{\noexpand\pgfqpointscale{-1}{\the\t@pgfplots@toka}}%
	\fi
	%
	% Process 'plot box ratio':
	\def\pgfplots@extract@plot@box@ratio##1##2##3##4\pgfplots@EOI{%
		\pgfmathparse{##1}\let\pgfplots@plotboxratio@x=\pgfmathresult
		\pgfmathparse{##2}\let\pgfplots@plotboxratio@y=\pgfmathresult
		\pgfmathparse{##3}\let\pgfplots@plotboxratio@z=\pgfmathresult
	}%
	\def\pgfplots@extract@plot@box@ratio@spaces##1 ##2 ##3 ##4\pgfplots@EOI{%
		\pgfplots@extract@plot@box@ratio{##1}{##2}{##3}{##4}\pgfplots@EOI
	}%
	\pgfkeysgetvalue{/pgfplots/plot box ratio}\pgfplots@loc@TMPa
	% Auto-determine input format which is either '{x}{y}{z}' or 'x y z'
	\def\pgfplots@loc@TMPb{%
		\pgfutil@ifnextchar\bgroup{%
			\pgfplots@loc@tmptrue
			\pgfplots@gobble@until@EOI
		}{%
			\pgfplots@loc@tmpfalse
			\pgfplots@gobble@until@EOI
		}%
	}%
	\expandafter\pgfplots@loc@TMPb\pgfplots@loc@TMPa\pgfplots@EOI
	\ifpgfplots@loc@tmp
		% Ah- braces format.
		\edef\pgfplots@loc@TMPa{\pgfplots@loc@TMPa{1}{1}{1}}%
		\expandafter\pgfplots@extract@plot@box@ratio\pgfplots@loc@TMPa\pgfplots@EOI
	\else
		% Ah- space-separated
		\edef\pgfplots@loc@TMPa{\pgfplots@loc@TMPa\space 1 1 1}%
		\expandafter\pgfplots@extract@plot@box@ratio@spaces\pgfplots@loc@TMPa\pgfplots@EOI
	\fi
	%
	% process it:
	\ifdim\pgfplots@plotboxratio@x pt=1pt
	\else
		\t@pgfplots@toka=\expandafter{\pgfplots@glob@TMPa}%
		\xdef\pgfplots@glob@TMPa{\noexpand\pgfqpointscale{\pgfplots@plotboxratio@x}{\the\t@pgfplots@toka}}%
	\fi
	\ifdim\pgfplots@plotboxratio@y pt=1pt
	\else
		\t@pgfplots@toka=\expandafter{\pgfplots@glob@TMPb}%
		\xdef\pgfplots@glob@TMPb{\noexpand\pgfqpointscale{\pgfplots@plotboxratio@y}{\the\t@pgfplots@toka}}%
	\fi
	\ifdim\pgfplots@plotboxratio@z pt=1pt
	\else
		\t@pgfplots@toka=\expandafter{\pgfplots@glob@TMPc}%
		\xdef\pgfplots@glob@TMPc{\noexpand\pgfqpointscale{\pgfplots@plotboxratio@z}{\the\t@pgfplots@toka}}%
	\fi
	%
	% n (3D!)
	\pgfmathmultiply@{-\sinaz}{\cosel}%
	\let\pgfmathresultNx=\pgfmathresult
	\pgfmathmultiply@{\cosaz}{\cosel}%
	\xdef\pgfplots@glob@TMPd{{\pgfmathresultNx}{\pgfmathresult}{\sinel}}%
	\endgroup
%\message{Setting x,y and z from {#1}{#2} to^^J x = \meaning\pgfplots@glob@TMPa,^^J y = \meaning\pgfplots@glob@TMPb,^^J z = \meaning\pgfplots@glob@TMPc,^^J n = \pgfplots@glob@TMPd.^^J}%
	\pgfsetxvec{\pgfplots@glob@TMPa}%
	\pgfsetyvec{\pgfplots@glob@TMPb}%
	\pgfsetzvec{\pgfplots@glob@TMPc}%
	\def\pgfplots@loc@TMPa##1##2##3{%
		\pgfplotsmathvectorfromstring{##1,##2,##3}{default}%
		\let\pgfplots@view@dir@threedim=\pgfplotsretval
	}%
	\expandafter\pgfplots@loc@TMPa\pgfplots@glob@TMPd\relax
	\def#3{0}%
}%

% Takes the current plot box, defined by the actual PGF x,y and z unit
% vectors, and re-scales it such that it fits into the 
% width and height of the axis (as they have been provided by the
% user).
%
% @see \pgfplots@scale@axisbox@to@widthheight
% @see\pgfplots@scaleaxes@to@BB
\def\pgfplots@scale@plotbox@to@widthheight{%
	\pgfplots@initsizes@get@width@withoutlabels
	\let\pgfplots@loc@TMPa=\pgfmathresult
	\pgfplots@initsizes@get@height@withoutlabels
	%
	\edef\pgfplots@loc@TMPa{{\pgfplots@loc@TMPa}{\pgfmathresult}}%
	\expandafter\pgfplots@scaleaxes@to@BB\pgfplots@loc@TMPa
}


% Takes the current PGF x,y and z unit vectors and scales them such
% that the bounding box of the final image has width #1 and height #2.
%
% The relative length of the input vectors is important for the 3D case: it
% will be scaled as-is. 
%
% PRECONDITION
% 	- the x, y and z unit vectors have been set to the proper
% 	DIRECTIONS. Their relative vector lengths are set-up properly
% 	(i.e. y is twice as large as x and half as large as z or so).
%	- the \ifpgfplots@threedim boolean is set.
%	- the data limits have been initialised and transformed according
%	to the data transformation.
%	- the data transformation has ONLY been applied to the axis limits
%	(not other axis inputs). It may be changed by this method.
% 
% POSTCONDITION
% 	- the unit vectors have been re-scaled such that the final plot
% 	has the desired dimensions.
% 	- the @veclength and @inverseveclength have been initialized
\def\pgfplots@scaleaxes@to@BB#1#2{%
	\if0\pgfplots@scale@mode@choice
		% scale mode=auto
		\def\pgfplots@scale@mode@choice{2}% stretch to fill
	\fi
	\pgfplots@scaleaxes@to@BB@{#1}{#2}%
	% NOTE: we have not yet computed the lengths of unit vectors. In
	% addition, we have not yet updated the normal vector.
}%

% DEPRECATED:
\def\pgfplots@rescale@view@dir{%
	\expandafter\ifx\csname pgfplots@view@dir@threedim\endcsname\relax
	\else
		% At this point, we ALREADY HAVE a normal vector. However, it
		% might be skewed due to the scaling.
		%
		% -> recompute normal vector. In earlier versions, I tried to
		% rescale it - but that was too complicated (for me). This
		% here produces correct results, and it is a correct approach
		% anyway.
		\pgfplotsgetnormalforcurrentview
	\fi
}%

% \pgfplots@BB@for@plotbox@get@unit@scales@for@limits{#1}{#2}{#3}:
% a helper tool which computes individual unit vector scales in order
% to respect the limits.
%
% This method ignores width/height; its purpose is only to make sure
% that [xmin,xmax] fits into the CURRENT plot box. 
%
% In this context, each unit vector is supposed to be scaled such that
% width/height fit if xmin=0 and xmax=1.
%
% #1 [output] a macro name which will contain the INVERSE scale for x
% #2 [output] a macro name which will contain the INVERSE scale for y
% #3 [output] a macro name which will contain the INVERSE scale for z
% 
\def\pgfplots@BB@for@plotbox@get@unit@scales@for@limits#1#2#3{%
	\if1\b@pgfplots@plotbox@xisunit
		% Consequently, we have to multiply  with 1/(max-min):
		% compute 1/(xmax - xmin) in float for more recent versions (see /pgfplots/compat/scaling).
		% I observed that it is much more accurate 
		\pgfmathsubtract@{\pgfplots@xmax}{\pgfplots@xmin}%
	\else
		\def\pgfmathresult{1}%
	\fi
	\let#1=\pgfmathresult
	%
	\if1\b@pgfplots@plotbox@yisunit
		\pgfmathsubtract@{\pgfplots@ymax}{\pgfplots@ymin}%
	\else
		\def\pgfmathresult{1}%
	\fi
	\let#2=\pgfmathresult
	%
	\ifpgfplots@threedim
		\if1\b@pgfplots@plotbox@zisunit
			\pgfmathsubtract@{\pgfplots@zmax}{\pgfplots@zmin}%
		\else
			\def\pgfmathresult{1}%
		\fi
	\else
		\def\pgfmathresult{1}%
	\fi
	\let#3=\pgfmathresult
}%
\def\pgfplots@BB@for@plotbox{%
	\ifpgfplots@threedim
		\pgfpathmoveto{\pgfqpointxyz\pgfplots@plotbox@xmin\pgfplots@plotbox@ymin\pgfplots@plotbox@zmin}%
		\pgfpathmoveto{\pgfqpointxyz\pgfplots@plotbox@xmin\pgfplots@plotbox@ymin\pgfplots@plotbox@zmax}%
		\pgfpathmoveto{\pgfqpointxyz\pgfplots@plotbox@xmin\pgfplots@plotbox@ymax\pgfplots@plotbox@zmin}%
		\pgfpathmoveto{\pgfqpointxyz\pgfplots@plotbox@xmin\pgfplots@plotbox@ymax\pgfplots@plotbox@zmax}%
		\pgfpathmoveto{\pgfqpointxyz\pgfplots@plotbox@xmax\pgfplots@plotbox@ymin\pgfplots@plotbox@zmin}%
		\pgfpathmoveto{\pgfqpointxyz\pgfplots@plotbox@xmax\pgfplots@plotbox@ymin\pgfplots@plotbox@zmax}%
		\pgfpathmoveto{\pgfqpointxyz\pgfplots@plotbox@xmax\pgfplots@plotbox@ymax\pgfplots@plotbox@zmin}%
		\pgfpathmoveto{\pgfqpointxyz\pgfplots@plotbox@xmax\pgfplots@plotbox@ymax\pgfplots@plotbox@zmax}%
	\else
		\pgfpathmoveto{\pgfqpointxy\pgfplots@plotbox@xmin\pgfplots@plotbox@ymin}%
		\pgfpathmoveto{\pgfqpointxy\pgfplots@plotbox@xmin\pgfplots@plotbox@ymax}%
		\pgfpathmoveto{\pgfqpointxy\pgfplots@plotbox@xmax\pgfplots@plotbox@ymin}%
		\pgfpathmoveto{\pgfqpointxy\pgfplots@plotbox@xmax\pgfplots@plotbox@ymax}%
	\fi
}%



% Returns width and height of the current plot box
% (the path produced by \pgfplots@BB@for@plotbox).
%
% PRECONDITION: \pgfplots@BB@for@plotbox is defined to produce a path
% for the plot box
%
% POSTCONDITION: \pgfplotsretval contains the with and
% \pgfplotsretvalb contains the height
\def\pgfplots@get@dimension@of@BB{%
	\begingroup
	\pgfinterruptboundingbox
	%
	% the result of this call will be used to scale to target
	% dimensions. If we omit \pgftransformreset here, we might
	% accidentally UNDO the PGF transformation matrix (compare by 
	% writing \tikzpicture[scale=0.5] before the axis).
	\pgftransformreset
	%
	% STEP 1: compute the bounding box for the plot box.
	\pgfplots@BB@for@plotbox
	%
	% TMPa = width
	\pgf@xa=\pgf@pathmaxx
	\advance\pgf@xa by-\pgf@pathminx
	% TMPb = height
	\pgf@xb=\pgf@pathmaxy
	\advance\pgf@xb by-\pgf@pathminy
	\xdef\pgfplots@glob@TMPa{%
		\def\noexpand\pgfplotsretval{\the\pgf@xa}%
		\def\noexpand\pgfplotsretvalb{\the\pgf@xb}%
	}%
	\pgfusepath{discard}%
	\endpgfinterruptboundingbox
	\endgroup
	\pgfplots@glob@TMPa
}%

\def\pgfplots@scaleaxes@to@BB@prepare@plotbox@limits{%
	\def\b@pgfplots@rescale@x{1}%
	\def\b@pgfplots@rescale@y{1}%
	\def\b@pgfplots@rescale@z{1}%
	%
	\pgfplots@scaleaxes@to@BB@prepare@plotbox@limits@ x%
	\pgfplots@scaleaxes@to@BB@prepare@plotbox@limits@ y%
	\pgfplots@scaleaxes@to@BB@prepare@plotbox@limits@ z%
}%

\def\pgfplots@scaleaxes@to@BB@prepare@plotbox@limits@#1{%
	\expandafter\ifx\csname pgfplots@#1\endcsname\pgfutil@empty
		\expandafter\def\csname pgfplots@plotbox@#1min\endcsname{0}%
		\expandafter\def\csname pgfplots@plotbox@#1max\endcsname{1}%
		\expandafter\def\csname b@pgfplots@plotbox@#1isunit\endcsname{1}%
	\else
		\pgfutil@namelet{pgfplots@plotbox@#1min}{pgfplots@#1min}%
		\pgfutil@namelet{pgfplots@plotbox@#1max}{pgfplots@#1max}%
		\expandafter\def\csname b@pgfplots@plotbox@#1isunit\endcsname{0}%
		\if2\pgfplots@scale@mode@choice
			% scale mode=stretch to fill
			\expandafter\def\csname b@pgfplots@rescale@#1\endcsname{0}%
		\fi
	\fi
	\expandafter\def\csname b@pgfplots@unitvec@is@zero@#1\endcsname{0}%
	\ifdim\csname pgf@#1x\endcsname=0pt %
		\ifdim\csname pgf@#1y\endcsname=0pt %
			\expandafter\def\csname b@pgfplots@unitvec@is@zero@#1\endcsname{1}%
		\fi
	\fi
}


\def\pgfplots@scaleaxes@to@BB@#1#2{%
	\begingroup
	%
	\pgfplots@scaleaxes@to@BB@prepare@plotbox@limits
	\def\pgfplots@target@limitrescale@x{1}%
	\def\pgfplots@target@limitrescale@y{1}%
	\def\pgfplots@target@limitrescale@z{1}%
	\if1\pgfplots@scale@mode@choice
		% scale mode=none
		\def\xscale{1}%
		\def\yscale{1}%
		\def\pgfplots@target@unit@scale@inv@x{1}%
		\def\pgfplots@target@unit@scale@inv@y{1}%
		\def\pgfplots@target@unit@scale@inv@z{1}%
	\else
		%
		% This here CAN cause anisotropic (different) scaling factors.
		\pgfplots@BB@for@plotbox@get@unit@scales@for@limits
			{\pgfplots@target@unit@scale@inv@x}
			{\pgfplots@target@unit@scale@inv@y}
			{\pgfplots@target@unit@scale@inv@z}%
		%
%\message{got scales to fit limits into BB: x=1/\pgfplots@target@unit@scale@inv@x, y=1/\pgfplots@target@unit@scale@inv@y, z=1/\pgfplots@target@unit@scale@inv@z^^J}%
		%
		\if3\pgfplots@scale@mode@choice
			% scale mode=scale uniformly
			% 
			% We need to recompensate in case the previous method chose
			% different unit scaling scalings:
			\pgfplots@BB@for@plotbox@get@unit@scales@compensated@axis@limits
				{\pgfplots@target@unit@scale@inv@x}
				{\pgfplots@target@unit@scale@inv@y}
				{\pgfplots@target@unit@scale@inv@z}
				{\pgfplots@target@limitrescale@x}{\pgfplots@target@limitrescale@y}{\pgfplots@target@limitrescale@z}%
			%
			%\pgfplots@BB@update@cumulative@limit@compensations
		\fi
%\message{adjusted scales for 'scale mode': x=1/\pgfplots@target@unit@scale@inv@x, y=1/\pgfplots@target@unit@scale@inv@y, z=1/\pgfplots@target@unit@scale@inv@z;  ^^J   axis limit componsation scales x=\pgfplots@target@limitrescale@x, y=\pgfplots@target@limitrescale@y, z=\pgfplots@target@limitrescale@z^^J}%
		%
		% ATTENTION: this MODIFIES \pgfplots@target@limitrescale@x and its
		% variants directly - and it needs the input values.
		\pgfplots@get@scale@horiz@and@vert
			{#1}%
			{#2}%
			{\xscale}%
			{\yscale}% yscale
			{\pgfplots@target@limitrescale@x}%
			{\pgfplots@target@limitrescale@y}%
			{\pgfplots@target@limitrescale@z}%
%\message{Got W/H scale for all x components: \xscale; for all y components: \xscale; ^^J   axis limit componsation scales x=1/\pgfplots@target@limitrescale@x, y=1/\pgfplots@target@limitrescale@y, z=1/\pgfplots@target@limitrescale@z^^J}%
		% Ok, we know the W,H scalings now.
		%
		%
		%
		\pgfplots@apply@unit@ratio
			{\pgfplots@target@unit@scale@inv@x}
			{\pgfplots@target@unit@scale@inv@y}
			{\pgfplots@target@unit@scale@inv@z}
			{\pgfplots@target@limitrescale@x@}{\pgfplots@target@limitrescale@y@}{\pgfplots@target@limitrescale@z@}%
		\pgfplots@BB@update@cumulative@limit@compensations
		%
%\message{adjusted scales for 'unit vector ratio': x=1/\pgfplots@target@unit@scale@inv@x, y=1/\pgfplots@target@unit@scale@inv@y, z=1/\pgfplots@target@unit@scale@inv@z;  ^^J   axis limit componsation scales x=\pgfplots@target@limitrescale@x, y=\pgfplots@target@limitrescale@y, z=\pgfplots@target@limitrescale@z^^J}%
	\fi
	%
	%
	\pgfplots@scaling@minimize@limitrescale%
	%
	%
	\pgfplots@scaling@adjust@datascaling%
%\message{adjusted scales for data scale trafo: x=1/\pgfplots@target@unit@scale@inv@x, y=1/\pgfplots@target@unit@scale@inv@y, z=1/\pgfplots@target@unit@scale@inv@z;  ^^J   axis limit componsation scales x=\pgfplots@target@limitrescale@x, y=\pgfplots@target@limitrescale@y, z=\pgfplots@target@limitrescale@z;^^J  data scale trafo exponents x=\pgfplots@target@datascaletrafo@x@exponent@old -> \pgfplots@target@datascaletrafo@x@exponent, y=\pgfplots@target@datascaletrafo@y@exponent@old -> \pgfplots@target@datascaletrafo@y@exponent, z=\pgfplots@target@datascaletrafo@z@exponent@old -> \pgfplots@target@datascaletrafo@z@exponent^^J}%
	%
	\pgfplots@scaling@compute@final@scales%
			{\xscale}{\yscale}%
			{\pgfplots@target@unit@scale@inv@x}%
			{\pgfplots@target@unit@scale@inv@y}%
			{\pgfplots@target@unit@scale@inv@z}%
	%
	\pgfplots@axis@apply@post@scale{x}%
	\pgfplots@axis@apply@post@scale{y}%
	\ifpgfplots@threedim
		\pgfplots@axis@apply@post@scale{z}%
	\fi
	%
	% and finally, resize limits appropriately and add all cumulative limit compensations:
	\xdef\pgfplots@glob@TMPa{%
		%
		\pgf@xx=\pgfplots@target@unit@scale@xx\pgf@xx
		\pgf@xy=\pgfplots@target@unit@scale@xy\pgf@xy
		%
		\pgf@yx=\pgfplots@target@unit@scale@yx\pgf@yx
		\pgf@yy=\pgfplots@target@unit@scale@yy\pgf@yy
		%
		\ifpgfplots@threedim
			\pgf@zx=\pgfplots@target@unit@scale@zx\pgf@zx
			\pgf@zy=\pgfplots@target@unit@scale@zy\pgf@zy
		\fi
		%
		\noexpand\pgfplots@apply@datascaletrafo@change@{x}{\pgfplots@target@datascaletrafo@x@exponent}%
		\noexpand\pgfplots@apply@datascaletrafo@change@{y}{\pgfplots@target@datascaletrafo@y@exponent}%
		\noexpand\pgfplots@apply@datascaletrafo@change@{z}{\pgfplots@target@datascaletrafo@z@exponent}%
		%
		\noexpand\pgfplots@apply@unit@vector@rescale@keep@size{x}{\pgfplots@target@limitrescale@x}%
		\noexpand\pgfplots@apply@unit@vector@rescale@keep@size{y}{\pgfplots@target@limitrescale@y}%
		\noexpand\pgfplots@apply@unit@vector@rescale@keep@size{z}{\pgfplots@target@limitrescale@z}%
		%
		\noexpand\pgfplots@notify@final@scalings{%
			x unit scale=\pgfplots@target@unit@scale@x,%
			y unit scale=\pgfplots@target@unit@scale@y,%
			z unit scale=\pgfplots@target@unit@scale@z,%
			x datatrafo exponent=\pgfplots@target@datascaletrafo@x@exponent,%
			y datatrafo exponent=\pgfplots@target@datascaletrafo@y@exponent,%
			z datatrafo exponent=\pgfplots@target@datascaletrafo@z@exponent,%
			x limit rescale=\pgfplots@target@limitrescale@x,%
			y limit rescale=\pgfplots@target@limitrescale@y,%
			z limit rescale=\pgfplots@target@limitrescale@z,%
		}%
	}%
	\endgroup
	\pgfplots@glob@TMPa
}%

% Checks for the case the ALL (visible) limit compensation scales are
% bigger than one (for example x = 1.22, y = 2). In such a case, we
% want to MINIMIZE the rescaling. This can happen if unit vector ratio
% is active.
%
% In our example, we want to use limit rescaling factors x = 1, y = 2/1.22
% and, consequently, unit rescaling factors x *= 1.22, y *= 1.22 .
%
% This method checks for the case and applies the rescaling if
% necessary.
%
\def\pgfplots@scaling@minimize@limitrescale{%
	% boolean allLimitScalesAreBiggerThanOne;
	\pgfplots@loc@tmptrue
	\if0\b@pgfplots@unitvec@is@zero@x
		\ifdim\pgfplots@target@limitrescale@x pt<1.002pt %
			\pgfplots@loc@tmpfalse
		\fi
	\fi
	\if0\b@pgfplots@unitvec@is@zero@y
		\ifdim\pgfplots@target@limitrescale@y pt<1.002pt %
			\pgfplots@loc@tmpfalse
		\fi
	\fi
	\if0\b@pgfplots@unitvec@is@zero@z
		\ifdim\pgfplots@target@limitrescale@z pt<1.002pt %
			\pgfplots@loc@tmpfalse
		\fi
	\fi
	%
	\ifpgfplots@loc@tmp
		\begingroup
		% Ah -- all non-vanishing limit rescaling factors are BIGGER
		% THAN ONE.
		% In this case, we can save some rescalings!
		%
		% Search for the smallest rescaling factor.
		\let\pgfplots@smallest=\pgf@x
		\pgfplots@smallest=16000pt %
		\def\pgfplots@smallest@arg{}%
		\if0\b@pgfplots@unitvec@is@zero@x
			\pgf@xa=\pgfplots@target@limitrescale@x pt %
			\ifdim\pgf@xa<\pgfplots@smallest%
				\pgfplots@smallest=\pgf@xa
				\def\pgfplots@smallest@arg{x}%
			\fi
		\fi
		\if0\b@pgfplots@unitvec@is@zero@y
			\pgf@xa=\pgfplots@target@limitrescale@y pt %
			\ifdim\pgf@xa<\pgfplots@smallest%
				\pgfplots@smallest=\pgf@xa
				\def\pgfplots@smallest@arg{y}%
			\fi
		\fi
		\if0\b@pgfplots@unitvec@is@zero@z
			\pgf@xa=\pgfplots@target@limitrescale@z pt %
			\ifdim\pgf@xa<\pgfplots@smallest%
				\pgfplots@smallest=\pgf@xa
				\def\pgfplots@smallest@arg{z}%
			\fi
		\fi
		%
		% OK. We have the smallest scaling factor. It is > 1.
		\pgfplotscoordmath{default}{parsenumber}{\pgfplots@target@limitrescale@x}%
		\let\pgfplots@target@limitrescale@x=\pgfmathresult
		\pgfplotscoordmath{default}{parsenumber}{\pgfplots@target@limitrescale@y}%
		\let\pgfplots@target@limitrescale@y=\pgfmathresult
		%
		%
		\pgfplotscoordmath{default}{parsenumber}{\pgfplots@target@unit@scale@inv@x}%
		\let\pgfplots@target@unit@scale@inv@x=\pgfmathresult
		\pgfplotscoordmath{default}{parsenumber}{\pgfplots@target@unit@scale@inv@y}%
		\let\pgfplots@target@unit@scale@inv@y=\pgfmathresult
		%
		\if0\b@pgfplots@unitvec@is@zero@z
			\pgfplotscoordmath{default}{parsenumber}{\pgfplots@target@limitrescale@z}%
			\let\pgfplots@target@limitrescale@z=\pgfmathresult
			\pgfplotscoordmath{default}{parsenumber}{\pgfplots@target@unit@scale@inv@z}%
			\let\pgfplots@target@unit@scale@inv@z=\pgfmathresult
		\fi
		%
		\pgfplotscoordmath{default}{op}{reciprocal}{{\csname pgfplots@target@limitrescale@\pgfplots@smallest@arg\endcsname}}%
		\let\scale=\pgfmathresult
		%
		\pgfplotsforeachentryinCSV\value{%
			\pgfplots@target@unit@scale@inv@x,%
			\pgfplots@target@unit@scale@inv@y,%
			\pgfplots@target@limitrescale@x,%
			\pgfplots@target@limitrescale@y%
		}{%
			\pgfplotscoordmath{default}{op}{multiply}{{\scale}{\value}}%
			\pgfplotscoordmath{default}{tofixed}{\pgfmathresult}%
			\expandafter\let\value=\pgfmathresult
		}%
		\if0\b@pgfplots@unitvec@is@zero@z
			\pgfplotsforeachentryinCSV\value{%
				\pgfplots@target@unit@scale@inv@z,%
				\pgfplots@target@limitrescale@z%
			}{%
				\pgfplotscoordmath{default}{op}{multiply}{{\scale}{\value}}%
				\pgfplotscoordmath{default}{tofixed}{\pgfmathresult}%
				\expandafter\let\value=\pgfmathresult
			}%
		\fi
		%
		\xdef\pgfplots@glob@TMPa{%
			\noexpand\def\noexpand\pgfplots@target@unit@scale@inv@x{\pgfplots@target@unit@scale@inv@x}%
			\noexpand\def\noexpand\pgfplots@target@unit@scale@inv@y{\pgfplots@target@unit@scale@inv@y}%
			\noexpand\def\noexpand\pgfplots@target@unit@scale@inv@z{\pgfplots@target@unit@scale@inv@z}%
			\noexpand\def\noexpand\pgfplots@target@limitrescale@x{\pgfplots@target@limitrescale@x}%
			\noexpand\def\noexpand\pgfplots@target@limitrescale@y{\pgfplots@target@limitrescale@y}%
			\noexpand\def\noexpand\pgfplots@target@limitrescale@z{\pgfplots@target@limitrescale@z}%
		}%
		\endgroup
		\pgfplots@glob@TMPa
%
%\message{adjusted scales by minimizing common scaling factors: x=1/\pgfplots@target@unit@scale@inv@x, y=1/\pgfplots@target@unit@scale@inv@y, z=1/\pgfplots@target@unit@scale@inv@z;  ^^J   axis limit componsation scales x=\pgfplots@target@limitrescale@x, y=\pgfplots@target@limitrescale@y, z=\pgfplots@target@limitrescale@z;^^J}%
	\fi
}%

% Defines 
% \pgfplots@target@unit@scale@xx
% \pgfplots@target@unit@scale@xy
% \pgfplots@target@unit@scale@yx
% \pgfplots@target@unit@scale@yy
% \pgfplots@target@unit@scale@zx
% \pgfplots@target@unit@scale@zy
% %
% \pgfplots@target@unit@scale@x
% \pgfplots@target@unit@scale@y
% \pgfplots@target@unit@scale@z
% by combining the input args.
%
% #1: the scale to be applied to ALL x components
% #2: the scale to be applied to ALL y components
% #3: the scale to be applied to x unit
% #4: the scale to be applied to y unit
% #5: the scale to be applied to z unit
\def\pgfplots@scaling@compute@final@scales#1#2#3#4#5{%
	% ##1: the axis (x,y,or z)
	% ##2: the horizontal scale
	% ##3: the vertical scale
	% ##4: the inverse unit scale for this axis
	\def\pgfplots@loc@TMPa##1##2##3##4{%
		\pgfplotscoordmath{\pgfplots@compat@scaling@coordmath}{parsenumber}{##2}%
		\let\xscale@@=\pgfmathresult
		\pgfplotscoordmath{\pgfplots@compat@scaling@coordmath}{parsenumber}{##3}%
		\let\yscale@@=\pgfmathresult
		\pgfplotscoordmath{\pgfplots@compat@scaling@coordmath}{parsenumber}{##4}%
		\let\unitscale@inv@@=\pgfmathresult
		%
		% NOTE : it *would* be more efficient to use
		% 1/\unitscale@inv@@ in the routines above. BUT THAT IS NOT BACKWARDS COMPATIBLE.
		% Leave it this way!
		\pgfplotscoordmath{\pgfplots@compat@scaling@coordmath}{op}{reciprocal}{{\unitscale@inv@@}}%
		\let\unitscale@@=\pgfmathresult
		\pgfplotscoordmath{\pgfplots@compat@scaling@coordmath}{tofixed}{\pgfmathresult}%
		\expandafter\let\csname pgfplots@target@unit@scale@##1\endcsname=\pgfmathresult
		%
		%
		\ifx\pgfplots@compat@scaling@coordmath@final\pgfplots@compat@scaling@coordmath
		\else
			% backwards compatibility is such a burden.... :-(
			%
			% earlier versions relied on TeX's dimen arithmetics to
			% multiply the final scales. Make sure we do the same -
			% rounding errors on unit vectors are instable, i.e. the
			% errors add up considerably.
			\pgfplotscoordmath{\pgfplots@compat@scaling@coordmath@final}{parsenumber}{\xscale@@}%
			\let\xscale@@=\pgfmathresult
			\pgfplotscoordmath{\pgfplots@compat@scaling@coordmath@final}{parsenumber}{\yscale@@}%
			\let\yscale@@=\pgfmathresult
			\pgfplotscoordmath{\pgfplots@compat@scaling@coordmath@final}{parsenumber}{\unitscale@@}%
			\let\unitscale@@=\pgfmathresult
			\pgfplotscoordmath{\pgfplots@compat@scaling@coordmath@final}{parsenumber}{\unitscale@inv@@}%
			\let\unitscale@inv@@=\pgfmathresult
		\fi
		%
		\ifpgfplots@threedim
			% backw. compatibility: this is how it used to be in 3d
			% axes:
			\pgfplotscoordmath{\pgfplots@compat@scaling@coordmath@final}{op}{multiply}{{\xscale@@}{\unitscale@@}}%
			\pgfplotscoordmath{\pgfplots@compat@scaling@coordmath@final}{tofixed}{\pgfmathresult}%
			\expandafter\let\csname pgfplots@target@unit@scale@##1x\endcsname=\pgfmathresult
			%
			\pgfplotscoordmath{\pgfplots@compat@scaling@coordmath@final}{op}{multiply}{{\yscale@@}{\unitscale@@}}%
			\pgfplotscoordmath{\pgfplots@compat@scaling@coordmath@final}{tofixed}{\pgfmathresult}%
			\expandafter\let\csname pgfplots@target@unit@scale@##1y\endcsname=\pgfmathresult
		\else
			% backw. compatibility: 2d axes used divide in earlier
			% versions, not reciprocal. Believe it or not; for
			% \pgfplots@compat@scaling@coordmath=pgfbasic, it makes a
			% visible difference of about 2-3pt in the complete figure
			% size.
			\pgfplotscoordmath{\pgfplots@compat@scaling@coordmath@final}{op}{divide}{{\xscale@@}{\unitscale@inv@@}}%
			\pgfplotscoordmath{\pgfplots@compat@scaling@coordmath@final}{tofixed}{\pgfmathresult}%
			\expandafter\let\csname pgfplots@target@unit@scale@##1x\endcsname=\pgfmathresult
			%
			\pgfplotscoordmath{\pgfplots@compat@scaling@coordmath@final}{op}{divide}{{\yscale@@}{\unitscale@inv@@}}%
			\pgfplotscoordmath{\pgfplots@compat@scaling@coordmath@final}{tofixed}{\pgfmathresult}%
			\expandafter\let\csname pgfplots@target@unit@scale@##1y\endcsname=\pgfmathresult
		\fi
		%
	}%
	\if1\b@pgfplots@rescale@x
		\pgfplots@loc@TMPa{x}{\xscale}{\yscale}{#3}%
	\else
		\pgfplots@loc@TMPa{x}{1}{1}{#3}%
	\fi
	%
	\if1\b@pgfplots@rescale@y
		\pgfplots@loc@TMPa{y}{\xscale}{\yscale}{#4}%
	\else
		\pgfplots@loc@TMPa{y}{1}{1}{#4}%
	\fi
	%
	\ifpgfplots@threedim
		\if1\b@pgfplots@rescale@z
			\pgfplots@loc@TMPa{z}{\xscale}{\yscale}{#5}%
		\else
			\pgfplots@loc@TMPa{z}{1}{1}{#5}%
		\fi
	\else
		\def\pgfplots@target@unit@scale@z{0}%
		\def\pgfplots@target@unit@scale@zx{0}%
		\def\pgfplots@target@unit@scale@zy{0}%
		\def\pgfplots@target@unit@scale@inv@z{inf}%
	\fi
	%
}%

\def\pgfplots@notify@final@scalings#1{%
	\pgfkeys{/pgfplots/scaling/.cd,	
		.unknown/.code={%
%\message{setting key '\pgfkeyscurrentkey' to {##1}^^J}
			\pgfkeyssetvalue{\pgfkeyscurrentkey}{##1}%
		},
		#1%
	}%
}%

% #1: either x,y, or z
% #2: the new exponent
\def\pgfplots@apply@datascaletrafo@change@#1#2{%
	\pgfplots@if{pgfplots@apply@datatrafo@#1}{%
		\pgfplotscoordmath{#1}{datascaletrafo get params}%
		\edef\pgfplots@loc@TMPa{\expandafter\pgfutil@firstoftwo\pgfmathresult}%
		\edef\pgfplots@loc@TMPb{#2}%
		\ifx\pgfplots@loc@TMPa\pgfplots@loc@TMPb
			% ok; the data scale trafo did not change at all - we
			% still have the same exponent.
		\else
			% Ah - we have a new data scale trafo!
			\pgfplotscoordmath{#1}{datascaletrafo inverse}{\csname pgfplots@#1min\endcsname}%
			\let\pgfplots@loc@TMPa=\pgfmathresult
			\pgfplotscoordmath{#1}{datascaletrafo inverse}{\csname pgfplots@#1max\endcsname}%
			\let\pgfplots@loc@TMPb=\pgfmathresult
			%
			% first: determine the optimal shift (which is the
			% transformed lower limit):
			\pgfplotscoordmath{#1}{datascaletrafo set params}{#2}{0}%
			\pgfplotscoordmath{#1}{datascaletrafo}{\pgfplots@loc@TMPa}%
			%
			% ok, finalize the data trafo:
			\pgfplotscoordmath{#1}{datascaletrafo set params}{#2}{\pgfmathresult}%
			%
			% ... and recompute axis limits:
			\pgfplotscoordmath{#1}{datascaletrafo}{\pgfplots@loc@TMPa}%
			\expandafter\let\csname pgfplots@#1min\endcsname=\pgfmathresult
			\pgfplotscoordmath{#1}{datascaletrafo}{\pgfplots@loc@TMPb}%
			\expandafter\let\csname pgfplots@#1max\endcsname=\pgfmathresult
		\fi
	}{}%
}%

% Inspects the limit enlargement factors and reinitializes the data
% scale transformations.
%
% The purpose of this method is to avoid "dimension too large" if the
% factors exceed certain limits.
%
% INPUT:
%  \pgfplots@target@limitrescale@x and its variants for y and z
%  \pgfplots@target@unit@scale@inv@x and its variants for y and z
%
% OUTPUT:
%  \pgfplots@target@datascaletrafo@x@exponent and its variants for y and z
%     -> contains NEW datascaletrafo exponents 
%  \pgfplots@target@datascaletrafo@x@exponent@old and its variants for y and z
%     -> contains OLD datascaletrafo exponents
%  \pgfplots@target@unit@scale@inv@x and its variants for y and z
%     -> contains (modified) unit vector scales
\def\pgfplots@scaling@adjust@datascaling{%
	\pgfplots@scaling@adjust@datascaling@for x%
	\pgfplots@scaling@adjust@datascaling@for y%
	\pgfplots@scaling@adjust@datascaling@for z%
}
\def\pgfplots@scaling@adjust@datascaling@for#1{%
	\pgfplots@if{pgfplots@apply@datatrafo@#1}{%
		\pgfplotscoordmath{#1}{datascaletrafo get params}%
		\def\pgfplots@loc@TMPa##1##2{%
			\expandafter\def\csname pgfplots@target@datascaletrafo@#1@exponent\endcsname{##1}%
			\expandafter\def\csname pgfplots@target@datascaletrafo@#1@exponent@old\endcsname{##1}%
		}%
		\expandafter\pgfplots@loc@TMPa\pgfmathresult
		\pgf@xa=\csname pgfplots@target@limitrescale@#1\endcsname pt
		\ifdim\pgf@xa>5pt %
			% We want to enlarge axis limits considerably! 
			%
			\pgfplots@scaling@adjust@datascaling@for@get@compensation{\pgf@xa}%
			%
			% Ok, make sure that we do not get "dimension too large"
			% by adjusting the data scale trafo.
			%
			% Note that the data scale trafo has (only) been applied
			% to axis limits, so we have to reapply it before these
			% changes can take effect:
			\pgf@xa=\csname pgfplots@target@unit@scale@inv@#1\endcsname pt
			\divide\pgf@xa by\pgfplotsretval\relax %
			\expandafter\edef\csname pgfplots@target@unit@scale@inv@#1\endcsname{\pgf@sys@tonumber\pgf@xa}%
			%
			\c@pgf@countd=\csname pgfplots@target@datascaletrafo@#1@exponent\endcsname\relax
			\advance\c@pgf@countd by-\pgfplotsretvalb\relax %
			\expandafter\edef\csname pgfplots@target@datascaletrafo@#1@exponent\endcsname{\the\c@pgf@countd}%
		\fi
	}{%
		\expandafter\def\csname pgfplots@target@datascaletrafo@#1@exponent\endcsname{0}%
		\expandafter\def\csname pgfplots@target@datascaletrafo@#1@exponent@old\endcsname{0}%
	}%
}

% Returns 
%  \pgfplotsretval -> the absolute scaling 
%  \pgfplotsretvalb -> the log10 of the scaling
\def\pgfplots@scaling@adjust@datascaling@for@get@compensation#1{
	\ifdim#1<100pt %
		\def\pgfplotsretval{10}%
		\def\pgfplotsretvalb{1}%
	\else
		\ifdim#1<1000pt %
			\def\pgfplotsretval{100}%
			\def\pgfplotsretvalb{2}%
		\else
			\ifdim#1<10000pt %
				\def\pgfplotsretval{1000}%
				\def\pgfplotsretvalb{3}%
			\else
				% too much for this approach anyway... and probably no
				% use-case at all.
				\def\pgfplotsretval{1000}%
				\def\pgfplotsretvalb{3}%
			\fi
		\fi
	\fi
}%

% Computes the initial scale from a plot box of unit size to the
% desired with and height.
%
% #1 the desired width
% #2 the desired height
% #3 [output] a macro which will contain the horizontal (x) scale
% #4 [output] a macro which will contain the vertical (y) scale
% #5 [input/output] a macro which, on input, contains the x axis limit compensation scale
%    which is required to select a single unit vector scale without
%    reducing the plots dimension (without actually respecting the
%    final dimension). On output, the input has been multiplied by
%    some additional x limit componensation scale (selected by scale
%    uniformly strategy).
% #6 [input/output] a macro which will contain a y axis limit
% compensation scale; it works in the same way as #5
% #7 [input/output] a macro which will contain a z axis limit
% compensation scale; it works in the same way as #5
\def\pgfplots@get@scale@horiz@and@vert#1#2#3#4#5#6#7{%
	\begingroup
	\edef\pgfplots@target@limitrescale@x{#5}%
	\edef\pgfplots@target@limitrescale@y{#6}%
	\edef\pgfplots@target@limitrescale@z{#7}%
	\pgfplots@get@dimension@of@BB
	\pgf@xa=\pgfplotsretval\relax
	\pgf@xb=\pgfplotsretvalb\relax
	\pgf@ya=#1\relax
	\pgf@yb=#2\relax
	\edef\w{\pgf@sys@tonumber\pgf@xa}%
	\edef\h{\pgf@sys@tonumber\pgf@xb}%
	\edef\W{\pgf@sys@tonumber\pgf@ya}%
	\edef\H{\pgf@sys@tonumber\pgf@yb}%
%\message{PGFPLOTS: the current unit vectors result in a UNIT BB of (\the\pgf@xa,\the\pgf@xb). Scaling it to (\the\pgf@ya,\the\pgf@yb)...^^J}%
	\ifcase\pgfplots@scale@mode@choice
		% scale mode=auto  does not happen here
	\or
		% scale mode=none  does not happen here
	\or
		% scale mode=stretch to fill
		% 
		% This is very simple:
		%
		% Compute individual scaling factors for X and Y
		% such that the UNIT-BB will have size #1,#2. Keep limits.
		\pgfmathdivide@{\W}{\w}%
		\let\scalex=\pgfmathresult
		%
		\pgfmathdivide@{\H}{\h}%
		\let\scaley=\pgfmathresult
		% 
		% no changes to the axis limits - we only rescale units.
		\def\pgfplots@target@limitrescale@x@{1}%
		\def\pgfplots@target@limitrescale@y@{1}%
		\def\pgfplots@target@limitrescale@z@{1}%
		\pgfplots@BB@update@cumulative@limit@compensations
	\or
		% scale mode=scale uniformly
		% compute ONE common scale for both, X and Y - and satisfy
		% width/height constraints by adjusting the axis limits.
		%
		% The idea is as follows:
		% we WANT to have width W and height H.
		% The constraint is that each unit vector must get the same
		% scale -- but the axis limits can receive individual
		% compensation scales. But it should "look reasonable well".
		%
		% currently, we have 
		% w = r_x e_xx  + r_y e_yx + rz e_zx  (with e_zx = 0 typically)
		% h = r_x e_xy  + r_y e_yy + rz e_zy
		%
		% where r_x, r_y, r_z are the maximal range of the data in
		% x,y,z respectively. Depending on the context of this method,
		% they are either 1 (relative coords) or
		% (xmax-xmin) (absolute coords).
		%
		% Now, search for a set of real numbers 
		% Rx, Ry, Rz, s
		% such that
		% W = (Rx r_x) (s e_xx)  + (Ry r_y) (s e_yx)  + (Rz r_z) (s e_zx)
		% H = (Rx r_x) (s e_xy)  + (Ry r_y) (s e_yy)  + (Rz r_z) (s e_zy)
		%
		% clearly, the solution is not unique.
		% ONE choice is to employ the fact that e_zx = 0  (or, for 2d
		% plots, e_zx=0, e_zy=0 and e_yx=0):
		%
		% in that case, we can compute s such that the equation for W
		% is satisfied and compensate only the limit r_z, i.e. to
		% choose
		% s := W / w,    (scale to satisfy width constraint)
		% Rx := Ry := 1  (keep limits in X and Y)
		% Rz = ( H - s (w - r_z e_zy) ) / (s r_z e_zy)    (adjust z limit to satisfy height constraint)
		%
		% This approach works well if W < H . If W > H, it will look
		% bad: Rz will be less than 1, causing the limit to become
		% smaller. This, in turn, will clip away parts of the image.
		%
		% 
		%
		% Another solution is to make it the other way: to keep the
		% limit r_z, but to reduce the size and enlarge the other
		% limits to satisfy the size constraints. This solution is
		% considerably more involved; it requires to solve a nonlinear
		% set of equations.
		%
		% Formally, this second solution uses
		% Rz := 1       (no limit componensation scale for z -- keep z limit)
		% R:= Rx := Ry  (same limit componensation scale for both X and Y)
		% R and s still need to be determined from the two equations for W
		% and H.
		%
		% Substituting the given choices into the equations for W and H, we find
		%
		% R = W / (s w)
		%
		% s = H * (R * (h-r_z e_zy) + r_z e_zy)^-1
		%
		% Here, we employed the definition of 'h', see above. The
		% equations are non-linear.
		%
		% ATTENTION: we assume that the datascaletrafo set params
		% method has been called with THE SAME SCALE IN EACH
		% DIRECTION.
		\if0\pgfplots@scaleuniformly@choice
			% scale uniformly strategy=auto
			\pgfplots@get@scale@horiz@and@vert@scaleuniformly@of@optimal@strategy
		\else
			\pgfplots@get@scale@horiz@and@vert@scaleuniformly
			\pgfplots@BB@update@cumulative@limit@compensations
		\fi
	\fi
	%
	\xdef\pgfplots@glob@TMPa{%
		\noexpand\def\noexpand#3{\scalex}%
		\noexpand\def\noexpand#4{\scaley}%
		\noexpand\def\noexpand#5{\pgfplots@target@limitrescale@x}%
		\noexpand\def\noexpand#6{\pgfplots@target@limitrescale@y}%
		\noexpand\def\noexpand#7{\pgfplots@target@limitrescale@z}%
	}%
	\endgroup
	\pgfplots@glob@TMPa
}%

% This is the implementation for 'scale uniformly strategy=auto'.
%
% It works by finding the strategy which involves the minimal scaling
% overhead.
%
% To this end, it computes the result for each 'scale uniformly
% strategy', and computes a cost function. The one with optimal cost
% function wins, and its results are returned.
%
% The cost function is the overal scaling which is applied to AXIS
% LIMITS. It works as follows:
% 1. if a choice requires to REDUCE the axis limits in order to
% fulfill all constraints, it is neglected (using maximal cost 16000).
% Reducing axis limits may clip away information.
% 
% 2. if a choice requires to ENLARGE some axis limits, its cost is the
% sum of the individual scaling factors (even if they are are one -
% who cares).
%
% Note that this method *is* relevant and the optimization appears to
% be necessary. 
% Examples are
%   unittest_scalemode_2d_standard_1.tex
% and perhaps
%   unittest_scalemode_2d_standard_0.tex 
% and more involved 3d examples are also available.
%
% My first guess was that it is sufficient to decide the optimal
% strategy in advance by comparing the target width and the target
% height - but that proved to be insufficient: it leads to correct
% results, but wastes too much space (i.e. enlarges limits too much).
%
% ATTENTION: the cost function INCLUDES RESULTS OF 
%	\pgfplots@BB@for@plotbox@get@unit@scales@for@limits and its
%	corrector
%	\pgfplots@BB@for@plotbox@get@unit@scales@compensated@axis@limits.
%
% More precisely, it relies on already computes limit compensation
% factors which do not depend on the target width/target height: both
% \pgfplots@BB@for@plotbox@get@unit@scales@compensated@axis@limits and
% this implementation of  'scale uniformly strategy' can be used to compute 
% the cost of a strategy.
% 
\def\pgfplots@get@scale@horiz@and@vert@scaleuniformly@of@optimal@strategy{%
	\begingroup
		\def\mathclass{default}%
		\pgfplotscoordmath{\mathclass}{max limit}%
		\let\pgfplots@cost@for@choice@superhigh=\pgfmathresult%
		%
		% private helpers to compute the cost. 
		\def\pgfplots@scalestrategy@compute@cost{%
			\begingroup
			% ATTENTION: this call changes
			% '\pgfplots@target@limitrescale@x' and its variants.
			% Restore its value after the iteration:
			\pgfplots@BB@update@cumulative@limit@compensations
			\pgfplotscoordmath{\mathclass}{one}%
			\let\ONE=\pgfmathresult
			\pgfplotscoordmath{\mathclass}{parsenumber}{\pgfplots@target@limitrescale@x}%
			\let\X=\pgfmathresult
			\pgfplotscoordmath{\mathclass}{parsenumber}{\pgfplots@target@limitrescale@y}%
			\let\Y=\pgfmathresult
			\ifpgfplots@threedim
				\pgfplotscoordmath{\mathclass}{parsenumber}{\pgfplots@target@limitrescale@z}%
				\let\Z=\pgfmathresult
			\else
				\let\Z=\pgfplots@target@limitrescale@z
			\fi
			%
			% If one of the resulting limit compensation scales is
			% less than 1, we can immediately skip it - we do not want
			% to risk to clip away image content.
			\pgfplotscoordmath{\mathclass}{if less than}{\X}{\ONE}{%
				\let\pgfplots@cost@for@choice=\pgfplots@cost@for@choice@superhigh
			}{%
				\pgfplotscoordmath{\mathclass}{if less than}{\Y}{\ONE}{%
					\let\pgfplots@cost@for@choice=\pgfplots@cost@for@choice@superhigh
				}{%
					\ifpgfplots@threedim
						\pgfplotscoordmath{\mathclass}{if less than}{\Z}{\ONE}{%
							\let\pgfplots@cost@for@choice=\pgfplots@cost@for@choice@superhigh
						}{%
							% ah - 3 limit scales >= 1. Good, assign cost:
							\pgfplots@scalestrategy@compute@cost@
						}%
					\else
						% ah - all limit scales >=1. Good, assign cost:
						\pgfplots@scalestrategy@compute@cost@
					\fi
				}%
			}%
%\message{scale uniformly strategy=auto: '\pgfplots@tostring@scaleuniformlystrategy{\pgfplots@scaleuniformly@choice}' has cost \pgfplots@cost@for@choice\space(limit rescaling factors x=\X, y=\Y, z=\Z)^^J}%
			\xdef\pgfplots@glob@TMPa{%
				\noexpand\def\noexpand\pgfplots@scaleuniformly@choice{\pgfplots@scaleuniformly@choice}%
				\noexpand\def\noexpand\scalex{\scalex}%
				\noexpand\def\noexpand\scaley{\scaley}%
				\noexpand\def\noexpand\pgfplots@target@limitrescale@x{\pgfplots@target@limitrescale@x}%
				\noexpand\def\noexpand\pgfplots@target@limitrescale@y{\pgfplots@target@limitrescale@y}%
				\noexpand\def\noexpand\pgfplots@target@limitrescale@z{\pgfplots@target@limitrescale@z}%
			}%
			\pgfmath@smuggleone\pgfplots@cost@for@choice
			% keep in mind that this scope IS NECESSARY: we have
			% changed the target quantities
			% \pgfplots@target@limitrescale@x and its variants!
			\endgroup
			\let\pgfplots@scalestrategy@values=\pgfplots@glob@TMPa
		}%
		\def\pgfplots@scalestrategy@compute@cost@{%
			\pgfplotscoordmath{\mathclass}{op}{add}{{\X}{\Y}}%
			\ifpgfplots@threedim
				\pgfplotscoordmath{\mathclass}{op}{add}{{\pgfmathresult}{\Z}}%
			\fi
			\let\pgfplots@cost@for@choice=\pgfmathresult
		}%
		%
		% compute initial cost:
		\def\pgfplots@scaleuniformly@choice{3}% change horizontal limits
		\pgfplots@get@scale@horiz@and@vert@scaleuniformly
		\pgfplots@scalestrategy@compute@cost
		%
		% init minimum:
		\let\pgfplots@cost@for@choice@arg=\pgfplots@scalestrategy@values
		\let\pgfplots@cost@for@choice@sofar=\pgfplots@cost@for@choice%
		%
		% compute cost of next strategy:
		\def\pgfplots@scaleuniformly@choice{2}% change vertical limits
		\pgfplots@get@scale@horiz@and@vert@scaleuniformly
		\pgfplots@scalestrategy@compute@cost
		%
		% update minimum:
		\pgfplotscoordmath{\mathclass}{if less than}{\pgfplots@cost@for@choice}{\pgfplots@cost@for@choice@sofar}{%
			\let\pgfplots@cost@for@choice@arg=\pgfplots@scalestrategy@values
			\let\pgfplots@cost@for@choice@sofar=\pgfplots@cost@for@choice%
		}{%
		}%
		%
		\ifx\pgfplots@cost@for@choice@sofar\pgfplots@cost@for@choice@superhigh
			% the algorithm discarded every available strategy.
			\def\pgfplots@scaleuniformly@choice{1}% fall back to 'units only'
			\pgfplots@get@scale@horiz@and@vert@scaleuniformly
			\pgfplots@scalestrategy@compute@cost
			\let\pgfplots@cost@for@choice@arg=\pgfplots@scalestrategy@values
			\let\pgfplots@cost@for@choice@sofar=\pgfplots@cost@for@choice%
		\fi
		%
		%
	\global\let\pgfplots@glob@TMPa=\pgfplots@cost@for@choice@arg
	\endgroup
	\pgfplots@glob@TMPa
%\message{scale uniformly strategy=auto: choosing '\pgfplots@tostring@scaleuniformlystrategy{\pgfplots@scaleuniformly@choice}'^^J}%
}

\def\pgfplots@tostring@scaleuniformlystrategy#1{%
	% scale uniformly strategy:
	\ifcase#1\relax 
		auto
	\or
		units only
	\or
		change vertical limits
	\or
		change horizontal limits
	\fi
}
% Does the work for 'scale mode=scale uniformly' inside of
% \pgfplots@get@scale@horiz@and@vert.
%
% It returns its result into \pgfplots@target@limitrescale@x@ (i.e.
% with an extra '@')
\def\pgfplots@get@scale@horiz@and@vert@scaleuniformly{%
	\ifcase\pgfplots@scaleuniformly@choice\relax
		% scale uniformly strategy=auto does not happen here.
	\or
		% scale uniformly strategy=units only
		\pgfplots@scaleuniformly@onlyunits
	\or
		% scale uniformly strategy=change vertical limits
		%
		% first, scale to the width ...
		\pgfplots@scaleuniformly@onlyunits@{\w}{\W}%
		% ... and change (only) vertical limits to get the "correct"
		% height:
		\ifdim\pgf@zy=0pt
			\ifdim\pgf@yx=0pt
				\pgfplots@prepare@vertical@rescaling@for@scale@uniformly@in@dir{y}\returninto\pgfplots@target@limitrescale@y@
			\else
				\pgfplots@scale@uniformly@fallback
			\fi
		\else
			\ifdim\pgf@zx=0pt
				\pgfplots@prepare@vertical@rescaling@for@scale@uniformly@in@dir{z}\returninto\pgfplots@target@limitrescale@z@
			\else
				\pgfplots@scale@uniformly@fallback
			\fi
		\fi
	\or
		% scale uniformly strategy=change horizontal limits
		\ifdim\pgf@zy=0pt
			\ifdim\pgf@yx=0pt
				\ifdim\pgf@xy=0pt
					% special 2d routine with explicit solution
					\pgfplots@scaleuniformly@change@horizontal@limits@twodim
						{\scalex}
						{\pgfplots@target@limitrescale@x@}
						{\pgfplots@target@limitrescale@y@}
						{\pgfplots@target@limitrescale@z@}%
				\else
					\pgfplots@scale@uniformly@fallback
				\fi
			\else
				\pgfplots@scale@uniformly@fallback
			\fi
		\else
			\ifdim\pgf@zx=0pt
				\pgfplots@scaleuniformly@change@horizontal@limits
					{\scalex}
					{\pgfplots@target@limitrescale@x@}
					{\pgfplots@target@limitrescale@y@}
					{\pgfplots@target@limitrescale@z@}%
			\else
				\pgfplots@scale@uniformly@fallback
			\fi
		\fi
		\let\scaley=\scalex
	\fi
}

\def\pgfplots@scaleuniformly@onlyunits{%
	% scale to the smaller target dimension:
	\ifdim\W pt<\H pt %
		\pgfplots@scaleuniformly@onlyunits@{\w}{\W}%
	\else
		\pgfplots@scaleuniformly@onlyunits@{\h}{\H}%
	\fi
}%

% #1 : the actual dimension
% #2 : the target dimension
\def\pgfplots@scaleuniformly@onlyunits@#1#2{%
	\def\pgfplots@target@limitrescale@x@{1}%
	\def\pgfplots@target@limitrescale@y@{1}%
	\def\pgfplots@target@limitrescale@z@{1}%
	\pgfmathdivide@{#2}{#1}%
	\let\scalex=\pgfmathresult
	\let\scaley=\scalex % we *need* the same unit scale.
}%

% Computes 'scale uniformly strategy=change horizontal limits'.
% This is a complicated solution, see the documentation in the
% implementation for 
% 'scale mode=scale uniformly'
%
% #1 [output] a macro which will contain the (uniform) scale for the
% unit vectors
% #2 [output] a macro which will contain a x axis limit compensation scale
% #3 [output] a macro which will contain a x axis limit compensation scale
% #4 [output] a macro which will contain a x axis limit compensation scale
\def\pgfplots@scaleuniformly@change@horizontal@limits#1#2#3#4{%
	\begingroup
	%
	\pgfplots@BB@for@plotbox@getunitheight{\pgf@xc}{z}%
	%
	% compute the rest in floating point - intermediate results may
	% become too huge for TeX.
	\pgfplotscoordmath{default}{parsenumber}{\expandafter\pgf@sys@tonumber\csname pgf@xc\endcsname}%
	\let\M=\pgfmathresult
	%
	\pgfplotscoordmath{default}{parsenumber}{\w}%
	\let\w=\pgfmathresult
	\pgfplotscoordmath{default}{parsenumber}{\W}%
	\let\W=\pgfmathresult
	\pgfplotscoordmath{default}{parsenumber}{\h}%
	\let\h=\pgfmathresult
	\pgfplotscoordmath{default}{parsenumber}{\H}%
	\let\H=\pgfmathresult
	\pgfplotscoordmath{default}{op}{divide}{{\W}{\w}}%
	\let\Wwinv=\pgfmathresult
	\pgfplotscoordmath{default}{op}{subtract}{{\h}{\M}}%
	\let\hminusM=\pgfmathresult
	%
	\pgfplotscoordmath{default}{one}%
	\let\S=\pgfmathresult%
	\let\R=\pgfmathresult%
	\let\Rx=\pgfmathresult
	\def\Rz{1}%
	%
	\def\pgfplots@hold@S@get@R{%
		\pgfplotscoordmath{default}{op}{divide}{{\Wwinv}{\S}}%
		\let\R=\pgfmathresult
%\message{updated R = \R\space ( S = \S ) ^^J}%
	}%
	\def\pgfplots@hold@R@get@S{%
		\pgfplotscoordmath{default}{op}{multiply}{{\R}{\hminusM}}%
		\pgfplotscoordmath{default}{op}{add}{{\pgfmathresult}{\M}}%
		\pgfplotscoordmath{default}{op}{divide}{{\H}{\pgfmathresult}}%
		\let\S=\pgfmathresult
%\message{updated S = \S\space ( R = \R ) ^^J}%
	}%
	%
	% This is the (most stupid) nonlinear method which is at hand:
	% fix point iteration.
	% choose R arbitrarily (R=1 seems adequate), solve for s. 
	% Then, fix s and solve for R. Then, fix R and
	% solve for s until convergence.
	\c@pgf@countc=0
	\pgfplotsloop{%
		\ifnum\c@pgf@countc<\pgfkeysvalueof{/pgfplots/scale uniformly strategy iter} %
			\pgfplotsloopcontinuetrue
		\else
			\pgfplotsloopcontinuefalse
		\fi
	}{%
		\pgfplots@hold@R@get@S \pgfplots@hold@S@get@R
		\advance\c@pgf@countc by1 %
	}%
	%
	\pgfplotscoordmath{default}{tofixed}{\R}\let\R=\pgfmathresult
	\pgfplotscoordmath{default}{tofixed}{\S}\let\S=\pgfmathresult
	\xdef\pgfplots@glob@TMPa{%
		\noexpand\def\noexpand#1{\S}%
		\noexpand\def\noexpand#2{\R}%
		\noexpand\def\noexpand#3{\R}%
		\noexpand\def\noexpand#4{\Rz}%
	}%
	\endgroup
	%
	\pgfplots@glob@TMPa
}%

% Computes 'scale uniformly strategy=change horizontal limits'.
% 
% This is a simplified closed solution assuming that e_xy=0 and e_yx = 0
%
% #1 [output] a macro which will contain the (uniform) scale for the
% unit vectors
% #2 [output] a macro which will contain a x axis limit compensation scale
% #3 [output] a macro which will contain a x axis limit compensation scale
% #4 [output] a macro which will contain a x axis limit compensation scale
\def\pgfplots@scaleuniformly@change@horizontal@limits@twodim#1#2#3#4{%
	\begingroup
	% Assuming that we have a standard 2d axis, i.e. 
	% e_zx = e_zy = 0,  e_xy = 0, and e_yx =0,
	% we can immediately compute a solution.
	%
	% In this case, we have the actual width
	% w = r_x e_xx  + r_y e_yx + rz e_zx
	%   = r_x e_xx
	% and actual height
	% h = r_x e_xy  + r_y e_yy + rz e_zy
	%   = r_y e_yy
	% and, consequently, desired width
	% W = (Rx r_x) (s e_xx)  + (Ry r_y) (s e_yx)  + (Rz r_z) (s e_zx)
	%   = (Rx r_x) (s e_xx)
	% and desired height
	% H = (Rx r_x) (s e_xy)  + (Ry r_y) (s e_yy)  + (Rz r_z) (s e_zy)
	%   = (Ry r_y) (s e_yy).
	% since this strategy changes horizontal limits (only), we have
	% Ry := 1.
	% We find
	% s : = H/h 
	% and
	% Rx : = W/w /s .
	%
	\pgfplotscoordmath{default}{parsenumber}{\w}%
	\let\w=\pgfmathresult
	\pgfplotscoordmath{default}{parsenumber}{\W}%
	\let\W=\pgfmathresult
	\pgfplotscoordmath{default}{parsenumber}{\h}%
	\let\h=\pgfmathresult
	\pgfplotscoordmath{default}{parsenumber}{\H}%
	\let\H=\pgfmathresult
	\pgfplotscoordmath{default}{op}{divide}{{\H}{\h}}%
	\let\S=\pgfmathresult
	\pgfplotscoordmath{default}{op}{divide}{{\W}{\w}}%
	\pgfplotscoordmath{default}{op}{divide}{{\pgfmathresult}{\S}}%
	\let\Rx=\pgfmathresult
	\def\Ry{1}%
	\def\Rz{1}%
	%
	\pgfplotscoordmath{default}{tofixed}{\Rx}\let\Rx=\pgfmathresult
	\pgfplotscoordmath{default}{tofixed}{\S}\let\S=\pgfmathresult
	\xdef\pgfplots@glob@TMPa{%
		\noexpand\def\noexpand#1{\S}%
		\noexpand\def\noexpand#2{\Rx}%
		\noexpand\def\noexpand#3{\Ry}%
		\noexpand\def\noexpand#4{\Rz}%
	}%
	\endgroup
	%
	\pgfplots@glob@TMPa
}%

\def\pgfplots@BB@update@cumulative@limit@compensations{%
%\message{  -> additional limit componensation scales x=\pgfplots@target@limitrescale@x@, y=\pgfplots@target@limitrescale@y@, z=\pgfplots@target@limitrescale@z@^^J}%
	% add limit compensation to what we have from earlier
	% operations:
	\pgfplotscoordmath{pgfbasic}{op}{multiply}{{\pgfplots@target@limitrescale@x@}{\pgfplots@target@limitrescale@x}}%
	\let\pgfplots@target@limitrescale@x=\pgfmathresult
	\pgfplotscoordmath{pgfbasic}{op}{multiply}{{\pgfplots@target@limitrescale@y@}{\pgfplots@target@limitrescale@y}}%
	\let\pgfplots@target@limitrescale@y=\pgfmathresult
	\pgfplotscoordmath{pgfbasic}{op}{multiply}{{\pgfplots@target@limitrescale@z@}{\pgfplots@target@limitrescale@z}}%
	\let\pgfplots@target@limitrescale@z=\pgfmathresult
}%

\def\pgfplots@scale@uniformly@fallback{%
	\ifpgfplots@scaleuniformly@warning
		\pgfplots@warning{Sorry, 'scale uniformly' failed because its actual implementation works only if y_x = 0 and x_y = 0 (for 2d axes) or if z_x = 0 (for 3d axes). The result will not fill the prescribed dimensions. Falling back to 'scale uniformly strategy=units only. (use scale uniformly warning=false to disable this warning)}%
	\fi
	\pgfplots@scaleuniformly@onlyunits
}%

% This is part of the implementation of 'scale mode=scale uniformly'.
%
% Its purpose it to set up the initial scaling such that 
% 1. each unit vector gets the same scale
% 2. the axis limits are resized (enlarged) to keep the plot box ratio
% (as far as possible)
%
% It repairs the outcome of
% \pgfplots@BB@for@plotbox@get@unit@scales@for@limits .
%
% The assumption is that on input #1, #2, and #3 are the factors which
% would be used by stretch-to-fill in order to squeze the axis limits
% into the plot box defined by e_x, e_y, and e_z (the unit vectors).
%
% On output, #1, #2, and #3 will be modified such that *each has the
% same value*. The value will be chosen with care. More precisely, it
% is the *minimum* of {#1,#2,#3}.
%
% Clearly, 'scale mode=scale uniformly' has less freedom than
% strech-to-fill. In order to keep the plot box ratio intact (as far
% as possible), the axis limits will be rescaled to componsate for the
% ignored scaling factors. More precisely, if direction i is not the
% extremal value (as discussed in the last paragraph), the axis limits
% will be rescaled by #i/extremum .
%
%
%
% #1: on input, it contains the x unit scale which would be taken without the
% compensation. On output, it contains the x unit scale which *will* be
% used.
% #2: same as #1, but for y
% #3: same as #1, but for z
% #4: [output] a scale for use as argument of \pgfplots@apply@unit@vector@rescale@keep@size{x}{<arg>}
% #5: [output] a scale for use as argument of \pgfplots@apply@unit@vector@rescale@keep@size{y}{<arg>}
% #6: [output] a scale for use as argument of \pgfplots@apply@unit@vector@rescale@keep@size{z}{<arg>}
%
% The output arguments need to be applied before they take effect.
\def\pgfplots@BB@for@plotbox@get@unit@scales@compensated@axis@limits#1#2#3#4#5#6{%
	\begingroup
	% ATTENTION : this code ASSUMES that the datascaling trafo is
	% initialized with THE SAME SCALE IN EACH DIRECTION.
	% The data scaling also leads to (potentially non-uniform) scaling per component.
	%
	% Note that we could handle the datascaling here -- but we would
	% leave the supported number range easily. That's why that part of
	% the 'scale mode=scale uniformly' implementation has been moved
	% to \pgfplots@set@optimal@datatrafos@allaxes
	%
	% This here handles the limits only.
	\edef\pgfplots@scale@unitx{#1}%
	\edef\pgfplots@scale@unity{#2}%
	\edef\pgfplots@scale@unitz{#3}%
	%
	% compute extreme + arg extreme of these scales:
	\def\pgfplots@extreme@scale{-16300}%
	\def\pgfplots@extreme@scale@arg{NONE}%
	%
	\if0\b@pgfplots@unitvec@is@zero@x
		\ifdim\pgfplots@extreme@scale pt<\pgfplots@scale@unitx pt
			\let\pgfplots@extreme@scale=\pgfplots@scale@unitx
			\def\pgfplots@extreme@scale@arg{x}%
		\fi
	\fi
	\if0\b@pgfplots@unitvec@is@zero@y
		\ifdim\pgfplots@extreme@scale pt<\pgfplots@scale@unity pt
			\let\pgfplots@extreme@scale=\pgfplots@scale@unity
			\def\pgfplots@extreme@scale@arg{y}%
		\fi
	\fi
	\if0\b@pgfplots@unitvec@is@zero@z
		\ifdim\pgfplots@extreme@scale pt<\pgfplots@scale@unitz pt
			\let\pgfplots@extreme@scale=\pgfplots@scale@unitz
			\def\pgfplots@extreme@scale@arg{z}%
		\fi
	\fi
	%
	% Now, adjust axis limits to compensate for the effect: we still
	% want to have a plot box which is as close as possible to the
	% target plot box.
	\def\pgfplots@loc@TMPa##1##2{%
		\if0\csname b@pgfplots@unitvec@is@zero@##1\endcsname
			\if1\pgfplots@scaleuniformly@choice % FIXME : this appears to be too much. Disable this!?
				% ok, nothing to do for this direction.
				\pgfplotscoordmath{pgfbasic}{one}%
				\let##2=\pgfmathresult
			\else
				\if\pgfplots@extreme@scale@arg ##1%
					% ok, nothing to do for this direction.
					\pgfplotscoordmath{pgfbasic}{one}%
					\let##2=\pgfmathresult
				\else
					\pgfplotscoordmath{pgfbasic}{op}{divide}{{\pgfplots@extreme@scale}{\csname pgfplots@scale@unit##1\endcsname}}%
					% do not call apply@unit@rescale immediately because the
					% unit vectors may not be in their final state. Postpone until
					% they are final.
					\edef##2{\pgfmathresult}%
				\fi
			\fi
		\else
			\def##2{1}%
		\fi
	}%
	\pgfplots@loc@TMPa{x}{#4}%
	\pgfplots@loc@TMPa{y}{#5}%
	\pgfplots@loc@TMPa{z}{#6}%
	%
	\toks0=\expandafter{#4}%
	\toks1=\expandafter{#5}%
	\toks2=\expandafter{#6}%
	\xdef\pgfplots@glob@TMPa{%
		% same scale in each dir:
		\def\noexpand#1{\pgfplots@extreme@scale}%
		\def\noexpand#2{\pgfplots@extreme@scale}%
		\def\noexpand#3{\pgfplots@extreme@scale}%
		\def\noexpand#4{\the\toks0}%
		\def\noexpand#5{\the\toks1}%
		\def\noexpand#6{\the\toks2}%
	}%
	\endgroup
	\pgfplots@glob@TMPa
}

% #1 : a dimen register
% #2 : x, y, or z
\def\pgfplots@BB@for@plotbox@getunitheight#1#2{%
	#1=\csname pgfplots@plotbox@#2max\endcsname\csname pgf@#2y\endcsname
	\advance#1 by -\csname pgfplots@plotbox@#2min\endcsname\csname pgf@#2y\endcsname
	\ifdim#1<0pt %
		% we want to return a height. It is also bigger than 0.
		% the difference above may be negative if the unit points
		% downward (special combinations of view/h and view/v)
		#1=-#1\relax
	\fi
}%


% Modifies the AXIS LIMITS to ensure that a suitable width/height is
% achieved.
%
% This does NOT introduce a further scale to the unit vectors.
%
% #1: a direction (x,y, or z)
% #2: a macro name. It will be assigned globally. It will contain
% EXECUTABLE instructions which will modify the axis limits to fit the
% scaling.
%
% PRECONDITION: 
% - \pgfplots@glob@TMPa contains the already computed
% scaling factor for 'scale uniformly'
% - \pgf@xb is the actual height and \pgf@yb is the desired height
%   (set as in the scaling routine)
%
% POSTCONDITION:
% #2 will contain the argument <arg> for \pgfplots@apply@unit@vector@rescale@keep@size{#1}{<arg>}
\def\pgfplots@prepare@vertical@rescaling@for@scale@uniformly@in@dir#1\returninto#2{%
	% The strategy is as follows:
	% 1. I want to fit the axis into width #1 (\pgf@ya) and
	% height #1 (\pgf@yb).
	% 2. I want to MAINTAIN the unit vector ratio. 
	% 3. I want to MAINTAIN the unit vector directions.
	%
	% I already know the scaling factor to fit the width (it
	% is stored in \scalex = \scaley).
	% Let's call it "s".
	%
	% Consequently, a uniform scaling by "s" leads to the image
	% height
	% h = s* (r_x * e_xy + r_y * e_yy + r_z * e_zy)
	% where r_i = (imax - imin). This here is essentially the
	% same as the bounding box computation above (at least for
	% standart orthographic 3D axes).
	%
	% What I want now is to enlarge the limits such that I
	% have BOTH, width #1 AND height #2, without obscuring the
	% unit vector ratio. Recall that width #1 is already
	% given.
	%
	% This strategy achieves this goal by
	% modifying axis limits for an axis whose unit vector is
	% parallel to the canvas y axis, i.e. e_i = (0,*). 
	%
	% That means I have to introduce a SECOND scale s_z which
	% applies only to the Z unit vector (since e_z = (0,*) ).
	% If H = #2 is the desired height, I find the target
	% equation for s_z,
	%
	% H = s* r_x e_xy + s * r_y e_yy + s_z * s * r_z * e_zy
	% =>
	% s_z = ( H- s*r_x e_xy - s*r_y e_yy) / ( s * r_z * e_zy).
	%
	% Remember that 
	% s = \scalex
	% H = \H
	% h = r_x * e_xy + r_y * e_yy + r_z * e_zy  = \h
	% =>
	% s_z = ( H- s*( h - r_z * e_zy) ) / ( s * r_z * e_zy).
	% 
	\begingroup
	\pgfplots@BB@for@plotbox@getunitheight{\pgf@xc}{#1}%
	%
	% compute the rest in floating point - intermediate results may
	% become too huge for TeX.
	\pgfplotscoordmath{default}{parsenumber}{\expandafter\pgf@sys@tonumber\csname pgf@xc\endcsname}%
	\let\pgfplots@diff=\pgfmathresult
	%
	\pgfplotscoordmath{default}{parsenumber}{\scalex}%
	\let\pgfplots@s=\pgfmathresult
	%
	% this is a precondition of this method:
	\pgfplotscoordmath{default}{parsenumber}{\h}%
	\let\h=\pgfmathresult
	\pgfplotscoordmath{default}{parsenumber}{\H}%
	\let\H=\pgfmathresult
	%
	% compute counter := H - s * (h - (max-min))
	\pgfplotscoordmath{default}{op}{subtract}{{\h}{\pgfplots@diff}}%
	\pgfplotscoordmath{default}{op}{multiply}{{\pgfplots@s}{\pgfmathresult}}%
	\pgfplotscoordmath{default}{op}{subtract}{{\H}{\pgfmathresult}}%
	\let\pgfplots@counter=\pgfmathresult
	%
	% computer denom := s * (max-min)
	\pgfplotscoordmath{default}{op}{multiply}{{\pgfplots@s}{\pgfplots@diff}}%
	\let\pgfplots@denom=\pgfmathresult
	%
	\pgfplotscoordmath{default}{op}{divide}{{\pgfplots@counter}{\pgfplots@denom}}%
	\pgfplotscoordmath{default}{tofixed}{\pgfmathresult}%
	%
	% Now, s_z = \pgfmathresult .
	%
	% Now, adjust the z limits.
	% Note that \pgfplots@apply@unit@vector@rescale@keep@size
	% has a slightly different context; it assumes that the
	% unit vector has been rescaled, not the axis limits.
	% Consequently, the inverse of the scaling factor enters.
	% Since \pgfplots@apply@unit@vector@rescale@keep@size
	% expects the inverse of the scale, we can provide
	% \pgfmathresult:
	\pgfmath@smuggleone\pgfmathresult
	\endgroup
	\let#2=\pgfmathresult
}

\def\pgfplots@computeunitvectorlengths{%
	\pgfplotsutil@edef@invoke\pgfmathveclen@{%
		{\pgf@sys@tonumber\pgf@xx}%
		{\pgf@sys@tonumber\pgf@xy}%
	}%
	\let\pgfplots@x@veclength=\pgfmathresult
	\pgfplotsmath@ifzero{\pgfplots@x@veclength}{%
		\def\pgfmathresult{infty}%
		% this case will be caught in \pgfplots@initsizes
	}{%
		\expandafter\pgfmath@basic@reciprocal@\expandafter{\pgfmathresult}%
	}%
	\let\pgfplots@x@inverseveclength=\pgfmathresult
	%
	\pgfplotsutil@edef@invoke\pgfmathveclen@{%
		{\pgf@sys@tonumber\pgf@yx}%
		{\pgf@sys@tonumber\pgf@yy}%
	}%
	\let\pgfplots@y@veclength=\pgfmathresult
	\pgfplotsmath@ifzero{\pgfplots@y@veclength}{%
		\def\pgfmathresult{infty}%
		% this case will be caught in \pgfplots@initsizes
	}{%
		\expandafter\pgfmath@basic@reciprocal@\expandafter{\pgfmathresult}%
	}%
	\let\pgfplots@y@inverseveclength=\pgfmathresult
	%
	\ifpgfplots@threedim
		\pgfplotsutil@edef@invoke\pgfmathveclen@{%
			{\pgf@sys@tonumber\pgf@zx}%
			{\pgf@sys@tonumber\pgf@zy}%
		}%
		\let\pgfplots@z@veclength=\pgfmathresult
		\pgfplotsmath@ifzero{\pgfplots@z@veclength}{%
			\def\pgfmathresult{infty}%
			% this case will be caught in \pgfplots@initsizes
		}{%
			\expandafter\pgfmath@basic@reciprocal@\expandafter{\pgfmathresult}%
		}%
		\let\pgfplots@z@inverseveclength=\pgfmathresult
	\else
		\def\pgfplots@z@veclength{0}%
		\def\pgfplots@z@inverseveclength{infty}%
	\fi
}%

% Defines \pgfplots@view@dir@threedim according to the actual
% configuration of x,y,z (2d) unit vectors, assuming the associated
% unit vectors form a right-handed-system.
%
% The algorithm works for standard three dimensional axes. It works as
% follows:
%
% First, observe that we have a normal direction N if all its
% multiples are mapped onto the same point in 2D canvas
% coordinates. In other words: all 3D coordinates which are mapped
% onto an arbitrary point in 2D canvas coordinates (take, for example,
% the origin (0,0)  ) are on a line in direction of N.
%
% We use this observation to compute the normal axis, i.e. we search
% for all points which are mapped onto the 2D canvas coordinate (0,0):
%  N_x e_xx + N_y e_yx + N_z e_zx = 0
%  N_x e_xy + N_y e_yy + N_z e_zy = 0.
% All solutions make up a linear space of dimension 1 (up to special
% cases). In the general case, we can chose an arbitrary  N_z != 0
% and reduce the linear system to
%  N_x e_xx + N_y e_yx = - N_z e_zx
%  N_x e_xy + N_y e_yy = - N_z e_zy.
% Choosing *any* N_z != 0, say, N_z=-1 (which corresponds to view
% from above) will lead to a vector parallel to the normal direction.
% But it might have the wrong sign.
%
% FIXME : this fails if one of e_x or e_y is zero.
%
% To find the correct sign for N, I have made several case
% distinctions to identify the cases when we have to multiply with -1.
% The key idea is to assume a right-handed-system of unit vectors;
% this is the condition which allows to determine the sign.
%
% Furthermore, I assume that e_z points to the top, i.e. that e_zy >0.
% Then, there are (mainly) four conditions on the signs of e_x and e_y
% which indicate that we are viewing from below and should switch the
% sign of N (keep in mind that our initial choice was N_z =-1, see above).
%
% The conditions can be identified by drawing a 3D box and
% identifying the corner which represents the lower left 3D limits.
%
% You can visualize these cases using
%--------------------------------------------------
% \pgfplotsset{
% 		separate axis lines,
% 		every outer x axis line/.append style= {-stealth},
% 		every outer y axis line/.append style= {-stealth},
% 		every outer z axis line/.append style= {-stealth},
% 		samples=2,shader=interp,title={view=\h,\v},
% 		domain=0:1,
% 		enlargelimits=false,
% 		view=\h\v,xlabel=x,ylabel=y,
% 		extra description/.code={%
% 			\node[draw,fill=white] at (axis cs:0,0,0) {};
% 		},
% }
% 
% \def\v{30}
% \foreach \h in {30,120,210,300} {
% \message{VIEW={\h}{\v}^^J}
% \begin{tikzpicture}
% 	\begin{axis}
% 	\addplot3[surf] {x};
% 	\end{axis}
% \end{tikzpicture}
% 
% }
% 
% \def\v{-30}
% \foreach \h in {30,120,210,300} {
% \message{VIEW={\h}{\v}^^J}
% \begin{tikzpicture}
% 	\begin{axis}
% 	\addplot3[surf] {x};
% 	\end{axis}
% \end{tikzpicture}
% 
% }
%-------------------------------------------------- 
%  The precise formulas can be found below in the source code.
%
% You can override this function by the /pgfplots/view dir key.
\def\pgfplotsgetnormalforcurrentview{%
	\pgfkeysgetvalue{/pgfplots/view dir}\pgfplots@loc@TMPc
	\ifx\pgfplots@loc@TMPc\pgfutil@empty
	\begingroup	
		% temporarily undo the effects of reversed axes -- we *really*
		% need a right-handed-coordinate system here:
		\if r\pgfkeysvalueof{/pgfplots/x dir/value}%
			\pgf@xx=-\pgf@xx
			\pgf@xy=-\pgf@xy
		\fi
		\if r\pgfkeysvalueof{/pgfplots/y dir/value}%
			\pgf@yx=-\pgf@yx
			\pgf@yy=-\pgf@yy
		\fi
		\if r\pgfkeysvalueof{/pgfplots/z dir/value}%
			\pgf@zx=-\pgf@zx
			\pgf@zy=-\pgf@zy
		\fi
		% FIRST: check for special cases.
		\let\pgfplots@view@dir@threedim=\pgfutil@empty%
		% Special case:
		% e_xx = e_xy = 0 
		%
		% i.e.:
		%
		%   ^
		%   | |---|
		%   z |   |
		%     |---|
		%       y->
		% 
		% In this case, N must be the x axis.
		\ifdim\pgf@xx=0pt %
			\ifdim\pgf@xy=0pt %
				\def\pgfplots@view@dir@threedim{-1,0,0}%
			\fi
		\fi
		% Special case:
		% e_yx = e_yy = 0 
		%
		% i.e.:
		%
		%   ^
		%   | |---|
		%   z |   |
		%     |---|
		%       x->
		% 
		% In this case, N must be the y axis.
		\ifdim\pgf@yx=0pt %
			\ifdim\pgf@yy=0pt %
				\def\pgfplots@view@dir@threedim{0,1,0}%
			\fi
		\fi
		% Special case:
		% e_xy = e_yy = 0  (i.e. one row)
		%
		% that is hard to draw, use view={30}{0} to see it. 
		% 
		% In this case, N_z must be 0 and we have a different system.
		\ifdim\pgf@xy=0pt %
			\ifdim\pgf@yy=0pt %
				\ifx\pgfplots@view@dir@threedim\pgfutil@empty
					% we have N_x e_xx + N_y e_yx = 0
					% Note that e_xx != 0 and e_yx != 0 (otherwise one
					% of our other special cases above would have
					% caught the case)
					% -> we have N_x = -N_y e_yx / e_xx and N_y
					%  arbitrary. only the sign needs to be fixed.
					\def\pgfplots@view@dir@threedim@z{0}%
					\def\pgfplots@view@dir@threedim@y{1}% fix it somehow. We correct the sign later.
					\edef\pgfplots@loc@TMPa{-(\pgfplots@view@dir@threedim@y) * \pgf@sys@tonumber\pgf@yx / (\pgf@sys@tonumber\pgf@xx)}%
					\pgfmathparse{\pgfplots@loc@TMPa}%
					\let\pgfplots@view@dir@threedim@x=\pgfmathresult
					%
					\def\pgfplots@scale{1}%
					% I identified these cases by comparing the
					% results with \pgfplots@scale{1} with those of
					% the view dir generated by
					% \pgfplotssetaxesfromazel (which has the correct quality of solution)
					\ifdim\pgf@zy>0pt %
						\ifdim\pgf@xx<0pt %
							\def\pgfplots@scale{-1}%
						\fi
					\else
						\ifdim\pgf@xx>0pt %
							\def\pgfplots@scale{-1}%
						\fi
					\fi
					\pgfmathmultiply@{\pgfplots@scale}{\pgfplots@view@dir@threedim@x}%
					\let\pgfplots@view@dir@threedim@x\pgfmathresult
					\pgfmathmultiply@{\pgfplots@scale}{\pgfplots@view@dir@threedim@y}%
					\let\pgfplots@view@dir@threedim@y\pgfmathresult
					\pgfmathmultiply@{\pgfplots@scale}{\pgfplots@view@dir@threedim@z}%
					\let\pgfplots@view@dir@threedim@z\pgfmathresult
					%
					\edef\pgfplots@view@dir@threedim{\pgfplots@view@dir@threedim@x,\pgfplots@view@dir@threedim@y,\pgfplots@view@dir@threedim@z}%
				\else
					% Ah - we already caught that special case above.
				\fi
			\fi
		\fi
		%
		% NOTE : the case e_xx = e_yx = 0 IS NO USE-CASE (would
		% require a rotated z axis which is forbidden currently)
		%
		\ifx\pgfplots@view@dir@threedim\pgfutil@empty
			\def\pgfplots@view@dir@threedim@z{-1}% hold it at some arbitrary value
			\pgf@xa=-\pgfplots@view@dir@threedim@z\pgf@zx
			\pgf@ya=-\pgfplots@view@dir@threedim@z\pgf@zy
			\edef\pgfplots@loc@TMPa{%
				{%
					{\pgf@sys@tonumber\pgf@xx}{\pgf@sys@tonumber\pgf@yx}%
					{\pgf@sys@tonumber\pgf@xy}{\pgf@sys@tonumber\pgf@yy}%
				}%
				{%
					{\pgf@sys@tonumber\pgf@xa}{\pgf@sys@tonumber\pgf@ya}%
				}%
			}%
			\expandafter\pgfutilsolvetwotwoleq\pgfplots@loc@TMPa
			\def\pgfplots@loc@TMPb##1##2{%
				\def\pgfplots@view@dir@threedim@x{##1}%
				\def\pgfplots@view@dir@threedim@y{##2}%
			}%
			\expandafter\pgfplots@loc@TMPb\pgfmathresult
			%
			% Identify if we need to switch the sign.
			% To verify that these cases are useful, I suggest visualizing
			% that stuff using the TeX code from above...
			%
			% I guess it is correct up to collapsing views (as you see, I
			% did not properly identify the cases with "=0" )
			\def\pgfplots@scale{1}%
			\ifdim\pgf@xx>0pt
				\ifdim\pgf@yx<0pt
				\else
					%
					\ifdim\pgf@xy<0pt
					\else
						\ifdim\pgf@yy<0pt
							\def\pgfplots@scale{-1}%
						\fi
					\fi
					%
				\fi
			\else
				\ifdim\pgf@xx<0pt
					\ifdim\pgf@yx>0pt
					\else
						%
						\ifdim\pgf@xy>0pt
						\else
							\ifdim\pgf@yy>0pt
								\def\pgfplots@scale{-1}%
							\fi
						\fi
						%
					\fi
				\fi
			\fi
			\ifdim\pgf@xy>0pt
				\ifdim\pgf@yy<0pt
				\else
					%
					\ifdim\pgf@xx>0pt
					\else
						\ifdim\pgf@yx>0pt
							\def\pgfplots@scale{-1}%
						\fi
					\fi
					%
				\fi
			\else
				\ifdim\pgf@xy<0pt
					\ifdim\pgf@yy>0pt
					\else
						%
						\ifdim\pgf@xx<0pt
						\else
							\ifdim\pgf@yx<0pt
								\def\pgfplots@scale{-1}%
							\fi
						\fi
						%
					\fi
				\fi
			\fi
			\pgfmathmultiply@{\pgfplots@scale}{\pgfplots@view@dir@threedim@x}%
			\let\pgfplots@view@dir@threedim@x\pgfmathresult
			\pgfmathmultiply@{\pgfplots@scale}{\pgfplots@view@dir@threedim@y}%
			\let\pgfplots@view@dir@threedim@y\pgfmathresult
			\pgfmathmultiply@{\pgfplots@scale}{\pgfplots@view@dir@threedim@z}%
			\let\pgfplots@view@dir@threedim@z\pgfmathresult
			%
			\pgfplotsmathvectorfromstring{\pgfplots@view@dir@threedim@x,\pgfplots@view@dir@threedim@y,\pgfplots@view@dir@threedim@z}{default}%
			\let\pgfplots@view@dir@threedim=\pgfplotsretval
			% normalize. This is not absolutely required -- but it is used
			% to accumulate point depth (for the mesh handler) in pgfmath
			% arithmetics. At least \pgfplotsmathviewdepthxyz should use
			% a properly scaled view dir.
			\pgfplotsmathvectorlength{\pgfplotsretval}{default}%
			\pgfplotscoordmath{default}{op}{reciprocal}{{\pgfplotsretval}}%
			\pgfplotsmathvectorscale{\pgfplots@view@dir@threedim}{\pgfmathresult}{default}%
		\else
			\pgfplotsmathvectorfromstring{\pgfplots@view@dir@threedim}{default}%
		\fi
		%
		\pgfmath@smuggleone\pgfplotsretval
	\endgroup
		\let\pgfplots@view@dir@threedim=\pgfplotsretval
	\else
		\def\pgfplots@loc@TMPb##1##2##3{%
			\pgfplotsmathvectorfromstring{##1,##2,##3}{default}%
			\let\pgfplots@view@dir@threedim=\pgfplotsretval
		}%
		\expandafter\pgfplots@loc@TMPb\pgfplots@loc@TMPc
	\fi
%\message{\string\pgfplotsgetnormalforcurrentview:  got (\pgfplots@view@dir@threedim)^^J}%
}%

% PRECONDITION:
% 	none
% POSTCONDITION:
% 	\pgfplots@default@aspect@ratio is set.
\def\pgfplots@compute@default@aspect@ratio{%
	\expandafter\pgfmath@x\axisdefaultwidth
	\expandafter\pgfmath@y\axisdefaultheight
	\pgfmathlog@invoke@expanded\pgfmathdivide@{%
		{\pgf@sys@tonumber{\pgfmath@x}}%
		{\pgf@sys@tonumber{\pgfmath@y}}%
	}%
	\let\pgfplots@default@aspect@ratio=\pgfmathresult
}

\def\pgfplots@ifneeds@one@uniform@datascale#1#2{%
	\if3\pgfplots@scale@mode@choice
		% scale mode=scale uniformly
		\def\pgfplots@loc@TMPa{1}%
		%
		% if we have at least one unit vector given explicitly, the
		% meaning changes: in that case, we can (and probably should)
		% use different data scale factors in each direction.
		\ifx\pgfplots@x\pgfutil@empty
		\else
			\def\pgfplots@loc@TMPa{0}%
		\fi
		\ifx\pgfplots@y\pgfutil@empty
		\else
			\def\pgfplots@loc@TMPa{0}%
		\fi
		\ifx\pgfplots@z\pgfutil@empty
		\else
			\def\pgfplots@loc@TMPa{0}%
		\fi
	\else
		\def\pgfplots@loc@TMPa{0}%
	\fi
	\if1\pgfplots@loc@TMPa
		#1%
	\else
		#2%
	\fi
}%

\def\pgfplots@set@default@size@options{%
	% The axes 'x' and 'y' vectors will be scaled such that the total
	% size is (\axisdefaultwidth, \axisdefaultheight).
	%
	% If the user specifies ONE of width OR height, 
	% the plot will be resized; keeping the aspect ratio.
	%
	\let\pgfplots@default@aspect@ratio=\pgfutil@empty
	\pgfkeysgetvalue{/pgfplots/x}{\pgfplots@x}%
	\pgfkeysgetvalue{/pgfplots/y}{\pgfplots@y}%
	\pgfkeysgetvalue{/pgfplots/z}{\pgfplots@z}%
	%\pgfkeysgetvalue{/pgfplots/viewdir}{\pgfplots@viewdir}%
	\pgfkeysgetvalue{/pgfplots/width}{\pgfplots@width}%
	\pgfkeysgetvalue{/pgfplots/height}{\pgfplots@height}%
	\ifx\pgfplots@width\pgfutil@empty
		\def\pgfplots@user@provided@width{0}%
	\else
		\def\pgfplots@user@provided@width{1}%
		\pgfmathparse{\pgfplots@width}%
		\edef\pgfplots@width{\pgfmathresult pt}%
	\fi
	\ifx\pgfplots@height\pgfutil@empty
		\def\pgfplots@user@provided@height{0}%
	\else
		\def\pgfplots@user@provided@height{1}%
		\pgfmathparse{\pgfplots@height}%
		\edef\pgfplots@height{\pgfmathresult pt}%
	\fi
	%
	% CASES:
	% W := 'width' option non-empty
	% H := 'height' option non-empty
	%
	% 		W H
	% 		0 0 -> \axisdefaultwidth 
	% 		0 1 -> determine width out of H and the default aspect ratio
	% 		1 X -> ok, use the user parameter.
	% -> KEEP ASPECT RATIO if just one W, or H is given!
	\ifx\pgfplots@width\pgfutil@empty
		\ifx\pgfplots@height\pgfutil@empty
			% The case W=0 H=0:
			\let\pgfplots@width=\axisdefaultwidth
			\let\pgfplots@height=\axisdefaultheight
		\else
			% The case W=0 H=1:
			\pgfplots@compute@default@aspect@ratio
			\expandafter\pgfmath@y\pgfplots@height
			\pgfmathlog@invoke@expanded\pgfmathmultiply@{%
				{\pgf@sys@tonumber{\pgfmath@y}}%
				{\pgfplots@default@aspect@ratio}%
			}%
			\edef\pgfplots@width{\pgfmathresult pt}%
		\fi
	\else
		\ifx\pgfplots@height\pgfutil@empty
			% The case W=1 H=0:
			\pgfplots@compute@default@aspect@ratio
			\expandafter\pgfmath@x\pgfplots@width
			\pgfmathlog@invoke@expanded\pgfmathdivide@{%
				{\pgf@sys@tonumber{\pgfmath@x}}%
				{\pgfplots@default@aspect@ratio}%
			}%
			\edef\pgfplots@height{\pgfmathresult pt}%
		\else
			% The case W=1 H=1:
		\fi
	\fi
	\pgfkeyslet{/pgfplots/width}{\pgfplots@width}%
	\pgfkeyslet{/pgfplots/height}{\pgfplots@height}%
	%
	\ifpgfplots@threedim
		\pgfplots@set@default@size@options@threedim
	\fi
	%
	\pgfplots@set@scale@mode
}

% This method must be called BEFORE THE DATASCALING is initialized.
\def\pgfplots@set@scale@mode{%
	\pgfkeysgetvalue{/pgfplots/unit vector ratio}\pgfplots@loc@TMPb
	\ifx\pgfplots@loc@TMPb\pgfutil@empty
	\else
		\ifcase\pgfplots@scale@mode@choice
			% 'scale mode'=auto
			\def\pgfplots@scale@mode@choice{3}% set to 'scale uniformly'
			%
			\if1\pgfplots@compat@scale@mode@compatible@mode
				% backwards compatibility mode...
				\ifpgfplots@threedim
					% ... for 3d: there is no backwards compatibility
					% mode here; it was plain wrong for 3d axes:
					% neither lengths nor angles have been correct.
					\pgfplots@compat@scale@mode@compatible@mode@warning
				\fi
			\fi
		\or
			% scale mode=none: keep it this way.
			\immediate\write-1{PGFPlots: scale mode=none  and unit vector ratio is incompatible. Ignoring unit vector ratio.^^J}%
		\or
			% scale mode=stretch to fill
			\immediate\write-1{PGFPlots: scale mode=stretch to fill  and unit vector ratio might produce unexpected results. Consider using scale mode=auto^^J}%
		\fi
	\fi
	%
}

\def\pgfplots@compat@scale@mode@compatible@mode@warning{%
	\pgfplots@warning{The content of your 3d axis has CHANGED compared to previous versions of pgfplots. Please review it. ^^J   %
		[continued]   Explanation: you have a 3d axis with 'axis equal' and/or 'unit vector ratio' which has (probably) been optimized for an older version of pgfplots. Any version older than 1.6 produced wrong output.^^J   %
	[continued]	  To remove this warning, write \string\pgfplotsset{compat=1.6} (or newer) in your preamble (may change all figures in your document) or by adding that to the affected axis.}%
}%

\def\pgfplots@set@default@size@options@threedim{%
	\pgfplots@loc@tmpfalse
	\ifx\pgfplots@x\pgfutil@empty
	\else
		\pgfplots@loc@tmptrue
	\fi
	\ifx\pgfplots@y\pgfutil@empty
	\else
		\pgfplots@loc@tmptrue
	\fi
	\ifx\pgfplots@z\pgfutil@empty
	\else
		\pgfplots@loc@tmptrue
	\fi
	\ifpgfplots@loc@tmp
		% oh - we have at least one of the [xyz] unit vectors!
		% make sure all of them are there
		\ifx\pgfplots@x\pgfutil@empty
			\pgfplots@set@default@size@options@threedim@{x}{(1pt,0pt)}%
		\fi
		\ifx\pgfplots@y\pgfutil@empty
			\pgfplots@set@default@size@options@threedim@{y}{(0pt,1pt)}%
		\fi
		\ifx\pgfplots@z\pgfutil@empty
			\pgfplots@set@default@size@options@threedim@{z}{(0pt,1pt)}%
		\fi
		\pgfkeyslet{/pgfplots/view/az}\pgfutil@empty
		\pgfkeyslet{/pgfplots/view/el}\pgfutil@empty
	\fi
}
\def\pgfplots@set@default@size@options@threedim@#1#2{%
	\pgfplots@error{Sorry, a 3D axis needs either NONE or ALL of "x,y,z". I found partial information, but (at least) '#1' is lacking... please add '#1'}%
	\expandafter\def\csname pgfplots@#1\endcsname{#2}%
}

% A helper method for \pgfplots@initsizes which
% - applies the data scaling trafo to user arguments
% - sets calls pgfset#1vec
%
% #1: the vector to set (either 'x' or 'y')
% #2: the index of the vector to set (either 0 or 1)
% #3: the already precomputed temporary scale (see pgfplots@initsizes)
% #4: an output argument. It is a macro name which will be defined to
% '1' if and only if the finally set vector is parallel to the #1 axis
% of PGF, that means (x,0) for #1=x and (0,y) for #2=y.
\def\pgfplots@initsizes@setunitvector#1#2#3#4{%
	\pgfkeysgetvalue{/pgfplots/#1 dir/value}\pgfplots@loc@dirvalue
	\expandafter\let\expandafter\pgfplots@loc@TMPb\csname pgfplots@#1\endcsname
	\ifx\pgfplots@loc@TMPb\pgfutil@empty
		\def#4{1}% we have (#1,0) or (0,#1)
		%
%\message{Setting unitvector(#1) to auto-computed multiple of e_#2 ...}%
		\edef\pgfplots@loc@TMPa{#3}%
		\if r\pgfplots@loc@dirvalue
			\edef\pgfplots@loc@TMPa{-#3}%
		\fi
		\ifcase#2\relax
			\pgfsetxvec{\pgfqpoint{\pgfplots@loc@TMPa pt}{0pt}}%
		\or
			\pgfsetyvec{\pgfqpoint{0pt}{\pgfplots@loc@TMPa pt}}%
		\or
			\pgfsetzvec{\pgfqpoint{\pgfplots@loc@TMPa pt}{\pgfplots@loc@TMPa pt}}%
		\fi
	\else
		% Ok, we have a user-defined unit vector.
		%
		% That means we also need to apply the scaling trafo!
		%
		% 1. Check whether we have a complete vector of type (x,y):
		\expandafter\pgfutil@in@\expandafter(\expandafter{\pgfplots@loc@TMPb}%
		\ifpgfutil@in@
			% YES: we have (x,y):
			%
			\def#4{0}% we DON'T have (#1,0) or (0,#1). At least I think so.
			%
%\message{Setting unitvector(#1) to non-standard \csname pgfplots@#1\endcsname ...}%
			\def\pgfplots@loc@TMPa(##1,##2){%
				\pgfplotscoordmath{default}{parse}{##1}%
				\pgfplotscoordmath{default}{tofixed}{\pgfmathresult}%
				\let\pgfplots@loc@TMPb=\pgfmathresult
				\pgfplotscoordmath{default}{parse}{##2}%
				\pgfplotscoordmath{default}{tofixed}{\pgfmathresult}%
				\let\pgfplots@loc@TMPc=\pgfmathresult
				%
				\pgfplots@if{pgfplots@apply@datatrafo@#1}{%
					\pgfplotscoordmath{#1}{datascaletrafo noshift inverse to fixed}{\pgfplots@loc@TMPb}%
					\let\pgfplots@loc@TMPb=\pgfmathresult
					\pgfplotscoordmath{#1}{datascaletrafo noshift inverse to fixed}{\pgfplots@loc@TMPc}%
					\let\pgfplots@loc@TMPc=\pgfmathresult
				}{}%
				\csname pgfset#1vec\endcsname{%
					\pgfqpoint
						{\if r\pgfplots@loc@dirvalue -\fi\pgfplots@loc@TMPb pt}
						{\if r\pgfplots@loc@dirvalue -\fi\pgfplots@loc@TMPc pt}}%
			}%
			\expandafter\pgfplots@loc@TMPa\pgfplots@loc@TMPb%
			%
		\else
			% NO we simply have a scalar value.
			\def#4{1}% we have (#1,0) or (0,#1)
%\message{Setting unitvector(#1) to \csname pgfplots@#1\endcsname * e_{#2}...}%
			\pgfplots@if{pgfplots@apply@datatrafo@#1}{%
				\pgfplotscoordmath{default}{parse}{\csname pgfplots@#1\endcsname}%
				\pgfplotscoordmath{default}{tofixed}{\pgfmathresult}%
				\pgfplotscoordmath{#1}{datascaletrafo noshift inverse to fixed}{\pgfmathresult}%
				\edef\pgfplots@loc@TMPb{\pgfmathresult pt}%
			}{\relax}%
			\edef\pgfplots@loc@TMPb{\if r\pgfplots@loc@dirvalue -\fi\pgfplots@loc@TMPb}%
			\begingroup
				\pgf@xa=\pgfplots@loc@TMPb\relax
				\xdef\pgfplots@glob@TMPb{\pgf@sys@tonumber{\pgf@xa}}%
			\endgroup
			\ifcase#2\relax
				\pgfsetxvec{\pgfqpoint{\pgfplots@loc@TMPb}{0pt}}%
			\or
				\pgfsetyvec{\pgfqpoint{0pt}{\pgfplots@loc@TMPb}}%
			\or
				\pgfsetzvec{\pgfqpoint{\pgfplots@loc@TMPb}{\pgfplots@loc@TMPb}}%
			\fi
		\fi
	\fi
%\message{-> got unitvector(#1) = (\the\csname pgf@#1x\endcsname, \the\csname pgf@#1y\endcsname).^^J}%
}%

% Applies the 'axis equal' feature.
%
% PRECONDITION:
% 	- #1, #2, #3 contains the current scaling
% 	factors in x,y, z, resp. which are to be applied to unit vectors
% 	- neither unit vectors nor limits are in their final shape
% 	-  \pgfplots@set@default@size@options has been invoked before
%
% POSTCONDITION:
%   - #1, #2, #3 have been changed to accomodate unit vector ratio
%   - #4, #5, #6 [output] contain axis limit compensation scales
%
% There is just one algorithmic difficulty: the data scaling
% transformation. All unit vector length above are only meaningful in
% the UNTRANSFORMED range, so we have to mingle with the scaling
% transformation.
\def\pgfplots@apply@unit@ratio#1#2#3#4#5#6{%
	\begingroup
	\edef\pgfplots@target@unit@scale@inv@x{#1}%
	\edef\pgfplots@target@unit@scale@inv@y{#2}%
	\edef\pgfplots@target@unit@scale@inv@z{#3}%
	\def\pgfplots@target@limitrescale@x@{1}%
	\def\pgfplots@target@limitrescale@y@{1}%
	\def\pgfplots@target@limitrescale@z@{1}%
	%
	\pgfkeysgetvalue{/pgfplots/unit vector ratio}\pgfplots@unit@vector@ratio
	\ifx\pgfplots@unit@vector@ratio\pgfutil@empty
	\else
		\edef\pgfplots@unit@vector@ratio{\pgfplots@unit@vector@ratio\space1 1 }%
		%
		\expandafter\pgfplots@unit@vector@ratio@check@nop\pgfplots@unit@vector@ratio\pgfplots@EOI
		\ifpgfplots@loc@tmp
			%
			% Step 1: compute the unit vector which STAYS CONSTANT.
			%
			\pgfkeysgetvalue{/pgfplots/unit vector ratio axis}\pgfplots@apply@unit@ratio@reference
			\ifx\pgfplots@apply@unit@ratio@reference\pgfutil@empty
				\pgfplots@apply@unit@ratio@find@reference%
			\fi
			%
			% FIXME : I could spent some attention here to save work: 
			% both, unit ratios and the resulting scales are computed at
			% least twice (once in \pgfplots@apply@unit@ratio@find@reference and once in the
			% following).
			\expandafter\pgfplots@apply@unit@ratio@prepareratios\pgfplots@unit@vector@ratio\pgfplots@EOI
			%
%\message{USING REFERENCE UNIT VECTOR FROM \pgfplots@apply@unit@ratio@reference; ratio \pgfplots@unit@ratio@x\space \pgfplots@unit@ratio@y\space \pgfplots@unit@ratio@z.^^J}%
			%
			% Step 2: apply the scaling:
			\pgfplots@rescale@unit@vector@reltoreference{x}{\pgfplots@unit@ratio@x}%
			\pgfplots@rescale@unit@vector@reltoreference{y}{\pgfplots@unit@ratio@y}%
			\ifpgfplots@threedim
				\pgfplots@rescale@unit@vector@reltoreference{z}{\pgfplots@unit@ratio@z}%
			\fi
			%
		\else
%\message{Skipped application of 'unit vector ratio=\pgfkeysvalueof{/pgfplots/unit vector ratio}': it is already done by 'scale uniformly'.^^J}%
		\fi
	\fi
	\xdef\pgfplots@glob@TMPa{%
		\noexpand\def\noexpand#1{\pgfplots@target@unit@scale@inv@x}%
		\noexpand\def\noexpand#2{\pgfplots@target@unit@scale@inv@y}%
		\noexpand\def\noexpand#3{\pgfplots@target@unit@scale@inv@z}%
		\noexpand\def\noexpand#4{\pgfplots@target@limitrescale@x@}%
		\noexpand\def\noexpand#5{\pgfplots@target@limitrescale@y@}%
		\noexpand\def\noexpand#6{\pgfplots@target@limitrescale@z@}%
	}%
	\endgroup
	\pgfplots@glob@TMPa
}%

\def\pgfplots@appy@unit@ratio@reciprocal#1{%
	\pgfplotscoordmath{default}{parsenumber}{#1}%
	\pgfplotscoordmath{default}{op}{reciprocal}{{\pgfmathresult}}%
	\pgfplotscoordmath{default}{tofixed}{\pgfmathresult}%
}%

% Defines \ifpgfplots@loc@tmp := need to modify scaling factors
\def\pgfplots@unit@vector@ratio@check@nop#1 #2 #3 #4\pgfplots@EOI{%
	\pgfplots@loc@tmptrue
	\if3\pgfplots@scale@mode@choice
		% scale mode=scale uniformly
		\ifpgfplots@threedim
			\ifdim#1pt=#2pt
				\ifdim#1pt=#3pt
					% 'axis equal' is implicitly done by 'scale mode=scale
					% uniformly' anyway
					\pgfplots@loc@tmpfalse
				\fi
			\fi
		\else
			\ifdim#1pt=#2pt
				% 'axis equal' is implicitly done by 'scale mode=scale
				% uniformly' anyway
				\pgfplots@loc@tmpfalse
			\fi
		\fi
	\fi
	% activate the following line to deactivate optimization: [FIXME]
	%\pgfplots@loc@tmpfalse
}%

% This macro determines the reference axis for unit vector rescaling.
% The reference axis remains unscaled (it gets scaling factor 1 if you
% want it this way).
% 
% The other axes are scaled such that the desired unit vector ratios
% are fulfilled.
%
% The idea to select a reference axis is as follows:
% 1. Every unit vector scaling factor s should fulfill  s  <= 1.
% 2. Choose the reference axis such that the minimal amount of scaling
% is performed.
%
% The motivation for (1) is: if all involved scaling factors are at
% most 1, the resulting picture will only become *smaller*.
% Consequently, we can simply enlarge axis limits to restore the
% original width/height!
%
% The motivation for (2) is: a huge amount of scaling might reduce the
% size of the image too much. Of course, the figure will be enlarged
% to fit the original width/height, but most of it will be empty. So,
% use the smallest scaling.
%
% @POSTCONDITION The reference axis is stored in
% \pgfplots@apply@unit@ratio@reference .
%
% @see the key 'unit vector ratio axis=y' which allows to manually
% select the reference axis. This will illustrate what happens here.
\def\pgfplots@apply@unit@ratio@find@reference{%
	%
	\begingroup
	\let\pgfplots@ONE=\pgf@x
	\global\pgfplots@ONE=1.002pt
	%
	\def\pgfplots@optimum@sofar@axis{}%
	\let\pgfplots@optimum@sofar@value=\pgf@y
	\global\pgfplots@optimum@sofar@value=16000pt
	%
	%\pgfplots@apply@unit@ratio@find@reference@checkexplicitlimits
	%
	\ifx\pgfplots@optimum@sofar@axis\pgfutil@empty
		% set \pgfplots@loc@TMPa := 1 if and only if the axis is 3d
		\def\pgfplots@loc@TMPa{0}%
		\if0\b@pgfplots@unitvec@is@zero@z
			% ah, it IS 3d!
			\def\pgfplots@loc@TMPa{1}%
		\else
			% ok, 2d mode (includes view={0}{90})
			\def\pgfplots@loc@TMPa{0}%
		\fi
		\if1\pgfplots@loc@TMPa
			% 3D is more complicated than 2D:
			% for every fixed reference axis, we have to check *two*
			% scaling factors.
			%
			% Furthermore, the optimality condition (2) needs to be
			% performed on the maximum max{1-s_a, 1-s_b} provided both of
			% these numbers are positive.
			%
			\def\pgfplots@check@##1##2{%
				% PRECONDITION: \pgfplots@apply@unit@ratio@reference is defined.
				%
				% renormalize \pgfplots@unit@[xyz] :
				\expandafter\pgfplots@apply@unit@ratio@prepareratios\pgfplots@unit@vector@ratio\pgfplots@EOI
				%
				% compute s_a :
				\pgfplots@getscale@unit@vector@reltoreference ##1{\csname pgfplots@unit@ratio@##1\endcsname}%
				\let\pgfplots@scale@a=\pgfmathresult
				%
				% compute s_b :
				\pgfplots@getscale@unit@vector@reltoreference ##2{\csname pgfplots@unit@ratio@##2\endcsname}%
				\let\pgfplots@scale@b=\pgfmathresult
				%
				% check if the actual choice of
				% \pgfplots@apply@unit@ratio@reference is FEASIBLE.
				% That is the case if s_a <= 1  && s_b <= 1.
				% We check
				%  (1 - s_a >= 0 )  && ( 1 - s_b >= 0 )
				% instead, since I need the value 
				%   max( 1-s_a, 1-s_b )
				% anyway.
				\def\pgfplots@ref@is@feasible{1}%
				\pgf@xa=\pgfplots@ONE \advance\pgf@xa by-\pgfplots@scale@a pt
				\ifdim\pgf@xa<0sp
					\def\pgfplots@ref@is@feasible{0}%
				\else
					\pgf@xb=\pgfplots@ONE \advance\pgf@xb by-\pgfplots@scale@b pt
					\ifdim\pgf@xb<0sp
						\def\pgfplots@ref@is@feasible{0}%
					\fi
				\fi
				% compute max(1-s_a,1-s_b) into \pgf@xa:
				% pgf@xa= max(pgf@xa,pgf@xb):
				\ifdim\pgf@xb>\pgf@xa	\pgf@xa=\pgf@xb	\fi
				\if1\pgfplots@ref@is@feasible
					\ifdim\pgf@xa<\pgfplots@optimum@sofar@value
						% Ah, ok. The actual choice is BETTER as it
						% involves less scaling.
						%
						% Remember it!
						\let\pgfplots@optimum@sofar@axis=\pgfplots@apply@unit@ratio@reference
						\global\pgfplots@optimum@sofar@value=\pgf@xa
					\fi
				\fi
%\message{^^Junit vector ratio 3D searching reference: checking \pgfplots@apply@unit@ratio@reference. feasable=\pgfplots@ref@is@feasible. \if1\pgfplots@ref@is@feasible max=\the\pgf@xa. \fi Optimum so far: value =\the\pgfplots@optimum@sofar@value\space for axis \pgfplots@optimum@sofar@axis.^^J}%
			}%
			%
			% Check 'x' as reference :
			\def\pgfplots@apply@unit@ratio@reference{x}%
			\pgfplots@check@ yz%
			%
			% Check 'y' as reference :
			\def\pgfplots@apply@unit@ratio@reference{y}%
			\pgfplots@check@ xz%
			%
			% Check 'z' as reference :
			\def\pgfplots@apply@unit@ratio@reference{z}%
			\pgfplots@check@ xy%
			%
		\else
			% 2D is much simpler: find the scale s which fulfills s <= 1.
			% One of them MUST fulfill it.
			% 
			% try 'x' axis as reference:
			\def\pgfplots@apply@unit@ratio@reference{x}%
			%
			% renormalize:
			\expandafter\pgfplots@apply@unit@ratio@prepareratios\pgfplots@unit@vector@ratio\pgfplots@EOI
			%
			% compute scaling factor: 
			\pgfplots@getscale@unit@vector@reltoreference y\pgfplots@unit@ratio@y%
			% 
%\message{^^Junit vector ratio 2D searching reference: checking \pgfplots@apply@unit@ratio@reference. feasable=\pgfmathresult < 1: \ifdim\pgfmathresult pt <\pgfplots@ONE YES-> use x\else NO->use y\fi^^J}%
			% and check (1). The condition (2) is irrelevant; it is met
			% anyway.
			\ifdim\pgfmathresult pt<\pgfplots@ONE
				\def\pgfplots@optimum@sofar@axis{x}%
			\else
				\def\pgfplots@optimum@sofar@axis{y}%
			\fi
		\fi
	\else
%\message{^^Junit vector ratio chose \pgfplots@optimum@sofar@axis\space to fulfill explicitly provided limits (at least partially).^^J}%
	\fi
	%
	\ifx\pgfplots@optimum@sofar@axis\pgfutil@empty
		\if1\b@pgfplots@unitvec@is@zero@z
			\def\pgfplots@optimum@sofar@axis{y}%
		\else
			\def\pgfplots@optimum@sofar@axis{z}%
		\fi
		\pgfplots@warning{The algorithm to implement 'unit vector ratio' failed! It could not determine the axis which shall be scaled and decided to use 'unit vector ratio axis=\pgfplots@optimum@sofar@axis'.}%
	\fi
	\let\pgfplots@apply@unit@ratio@reference=\pgfplots@optimum@sofar@axis
	\pgfmath@smuggleone\pgfplots@apply@unit@ratio@reference
	\endgroup
}%
\def\pgfplots@apply@unit@ratio@find@reference@checkexplicitlimits{%
	\ifpgfplots@autocompute@ymax \else \def\pgfplots@optimum@sofar@axis{y}\fi
	\ifpgfplots@autocompute@ymin \else \def\pgfplots@optimum@sofar@axis{y}\fi
	\ifpgfplots@autocompute@xmax \else \def\pgfplots@optimum@sofar@axis{x}\fi
	\ifpgfplots@autocompute@xmin \else \def\pgfplots@optimum@sofar@axis{x}\fi
	\ifpgfplots@threedim
		\ifpgfplots@autocompute@zmax \else \def\pgfplots@optimum@sofar@axis{z}\fi
		\ifpgfplots@autocompute@zmin \else \def\pgfplots@optimum@sofar@axis{z}\fi
	\fi
}%

% This is ONLY applied to the value of 'unit vector ratio'. It does
% not touch the current axis scaling factors.
\def\pgfplots@apply@unit@ratio@prepareratios#1 #2 #3 #4\pgfplots@EOI{%
	\def\pgfplots@unit@ratio@x{#1}%
	\def\pgfplots@unit@ratio@y{#2}%
	\def\pgfplots@unit@ratio@z{#3}%
	%
	% 'unit vector ratio' is measured relative to the y axis for 2d
	% and relative to the z axis for 3d plots.
	% renormalize such that it is relative to
	% \pgfplots@apply@unit@ratio@reference.
	%
	% Furthermore, renormalize such that
	% unit@ratio@\pgfplots@apply@unit@ratio@reference is 1.
	\pgfmathreciprocal@{\csname pgfplots@unit@ratio@\pgfplots@apply@unit@ratio@reference\endcsname}%
	\let\pgfplots@loc@TMPa=\pgfmathresult
	\ifpgfplots@threedim
		\if z\pgfplots@apply@unit@ratio@reference
		\else
			\pgfmathmultiply@{\pgfplots@loc@TMPa}{\pgfplots@unit@ratio@z}%
			\let\pgfplots@loc@TMPa=\pgfmathresult
		\fi
		%
		\pgfmathmultiply@{\pgfplots@loc@TMPa}{\pgfplots@unit@ratio@x}%
		\let\pgfplots@unit@ratio@x=\pgfmathresult
		%
		\pgfmathmultiply@{\pgfplots@loc@TMPa}{\pgfplots@unit@ratio@y}%
		\let\pgfplots@unit@ratio@y=\pgfmathresult
		%
		\pgfmathmultiply@{\pgfplots@loc@TMPa}{\pgfplots@unit@ratio@z}%
		\let\pgfplots@unit@ratio@z=\pgfmathresult
	\else
		\if y\pgfplots@apply@unit@ratio@reference
		\else
			\pgfmathmultiply@{\pgfplots@loc@TMPa}{\pgfplots@unit@ratio@y}%
			\let\pgfplots@loc@TMPa=\pgfmathresult
		\fi
		%
		\pgfmathmultiply@{\pgfplots@loc@TMPa}{\pgfplots@unit@ratio@x}%
		\let\pgfplots@unit@ratio@x=\pgfmathresult
		%
		\pgfmathmultiply@{\pgfplots@loc@TMPa}{\pgfplots@unit@ratio@y}%
		\let\pgfplots@unit@ratio@y=\pgfmathresult
		%
		\def\pgfplots@unit@ratio@z{<unused>}%
	\fi
	%
}%


% Computes a new unit vector E_#1 for direction #1 such that
% ||E_#1|| = #2 * ||e_reference||.
% Here, #2 is a scaling factor and e_reference is a reference axis.
% The reference axis is stored in
% \pgfplots@apply@unit@ratio@reference, the macro contains one of
% {x,y,z}.
%
% The data limits for '#1' will be enlarged as well (for 'unit rescale
% keep size').
%
% #1 is the axis which should be scaled (i.e. #1 in {x,y,z}).
% It is allowed if #1 = \pgfplots@apply@unit@ratio@reference. In this
% case, you can provide a scale '#2' to rescale the axis.
%
% #2 is a desired scale, relative to
% \pgfplots@apply@unit@ratio@reference. #2 should be a number without
% unit.
%
% The parameter \pgfplots@apply@unit@ratio@reference is also one of
% {x,y,z}.
%
\def\pgfplots@rescale@unit@vector@reltoreference#1#2{%
	\def\pgfplots@loc@TMPa{0}%
	\if#1\pgfplots@apply@unit@ratio@reference
		\pgfplotsmath@ifapproxequal@dim{#2pt}{1pt}{0.0002pt}{%
		}{%
			\def\pgfplots@loc@TMPa{1}%
		}%
	\else
		\def\pgfplots@loc@TMPa{1}%
	\fi
	\if1\csname b@pgfplots@unitvec@is@zero@#1\endcsname
		\def\pgfplots@loc@TMPa{0}%
	\fi
	\if1\pgfplots@loc@TMPa
		%
		\pgfplots@getscale@unit@vector@reltoreference{#1}{#2}%
		\global\let\pgfplots@glob@TMPa=\pgfmathresult
		%
%\message{Rescaling '#1' by \pgfplots@glob@TMPa.^^J}%
		%
		\pgfmathdivide@{\csname pgfplots@target@unit@scale@inv@#1\endcsname}{\pgfplots@glob@TMPa}%
		\expandafter\let\csname pgfplots@target@unit@scale@inv@#1\endcsname=\pgfmathresult
		%
		\pgfmathreciprocal@\pgfplots@glob@TMPa
		\expandafter\let\csname pgfplots@target@limitrescale@#1@\endcsname=\pgfmathresult
		%
	\fi
}

% Updates the #1 axis limits such that the axis' dimensions
% stay the same after scaling the #1 unit vector by a scale 's'.
%
% PRECONDITION:
%   - the #1 unit vector has been rescaled by a factor s.
%   For example, e_xnew := e_x * 0.5 .
% 
% POSTCONDITION:
%   - the axis limits are enlarged by a factor 1/s such that 
%   1/s (#1max - #1min) * e_xnew = (#1max- #1min) * e_x.
%
% In other words, the unit vector rescale is componensated by
% modifying the axis limits: we want to add an absolute component 'd'
% to the range:
% 1/s (xmax - xmin ) = xmax - xmin +d 
% =>
% d = (1/s - 1) * (xmax - xmin)
%
% The only remaining thing to do is to distribute 'd' to 'xmax' and
% 'xmin'. Typically, 50% to each will be fine, I guess...
% 
% #1: either x, y or z. It denotes the direction which has been
% modified.
% #2: the INVERSE of the scaling factor, #2 = 1/s .
%
\def\pgfplots@apply@unit@vector@rescale@keep@size#1#2{%
	\ifdim#2pt=1pt
	\else
	\if0\pgfplots@unit@vector@rescale@keep@size
		% unit rescale keep size=false : do nothing. Ignore the
		% scaling request.
	\else
		% unit rescale keep size=true|unless limits declared
		%
%\message{'unit rescale keep size': Resizing data range for #1 by #2: from \csname pgfplots@#1min\endcsname:\csname pgfplots@#1max\endcsname\ to}%
		\pgfmathsubtract@{\csname pgfplots@#1max\endcsname}{\csname pgfplots@#1min\endcsname}%
		\begingroup
			\pgf@xa=\pgfmathresult pt
			\pgfmathsubtract@{#2}{1.0}%
			\pgf@xa=\pgfmathresult \pgf@xa% this is 'd'
			%
			% \pgfplots@glob@TMPb : will be subtracted from #1min
			% \pgfplots@glob@TMPc : will be added to #1max
			\pgfplots@if{pgfplots@autocompute@#1min}{%
				\pgfplots@if{pgfplots@autocompute@#1max}{%
					\pgf@xa=0.5 \pgf@xa
					\xdef\pgfplots@glob@TMPb{\pgf@sys@tonumber{\pgf@xa}}%
					\xdef\pgfplots@glob@TMPc{\pgfplots@glob@TMPb}%
				}{%
					\xdef\pgfplots@glob@TMPb{\pgf@sys@tonumber{\pgf@xa}}%
					\xdef\pgfplots@glob@TMPc{0.0}%
				}%
			}{%
				\pgfplots@if{pgfplots@autocompute@#1max}{%
					\xdef\pgfplots@glob@TMPb{0.0}%
					\xdef\pgfplots@glob@TMPc{\pgf@sys@tonumber{\pgf@xa}}%
				}{%
					\if1\pgfplots@unit@vector@rescale@keep@size
						% unit rescale keep size=true : FORCE
						% enlargement!
						\pgf@xa=0.5 \pgf@xa
						\xdef\pgfplots@glob@TMPb{\pgf@sys@tonumber{\pgf@xa}}%
						\xdef\pgfplots@glob@TMPc{\pgfplots@glob@TMPb}%
					\else
						% unit rescale keep size=unless limits declared: 
						% do not scale - all limits are declared
						% explicitly
						\xdef\pgfplots@glob@TMPb{0.0}%
						\xdef\pgfplots@glob@TMPc{0.0}%
					\fi
				}%
			}%
		\endgroup
		\pgfmathsubtract@{\csname pgfplots@#1min\endcsname}{\pgfplots@glob@TMPb}%
		\expandafter\global\expandafter\let\csname pgfplots@#1min\endcsname=\pgfmathresult
		\pgfmathadd@{\csname pgfplots@#1max\endcsname}{\pgfplots@glob@TMPc}%
		\expandafter\global\expandafter\let\csname pgfplots@#1max\endcsname=\pgfmathresult
%\message{\csname pgfplots@#1min\endcsname:\csname pgfplots@#1max\endcsname. [- \pgfplots@glob@TMPb; + \pgfplots@glob@TMPc]^^J}%
		%
		% Update auxiliary data members:
		\pgfplots@visphase@notify@changeofcanvaslimits{#1}%
	\fi
	\fi
}%

% #1: an axis which should be scaled
% #2: the desired final ratio  ||e_#1||/||e_ref||
\def\pgfplots@getscale@unit@vector@reltoreference#1#2{%
	% 
	% If the datascaling transformation is active (which is almost
	% everytime the case here), we have a transformation
	% T^{-1}(x)= 10^scale * x
	% with different scales for every axis.
	%
	% If the datascaling transformation is NOT active, scale is 0
	% and T^{-1} = Identity.
	%
	% Note that the datascaling transformation also has
	% translations (shifts). These are not important here.
	%
	% Goal:
	% compute E_#1 such that
	%   #2* || T^{-1} e_ref || = || T^{-1} E_#1 ||
	% where T^{-1} is the data scaling transformation and e_ref the
	% reference unit vector. Keep in mind that there are
	% *different* data scaling transformations for each axis.
	%
	% We are given e_ref and e_#1 and the desired aspect ratio
	% between e_ref and E_#1, which is available as #2.
	%
	% So: T^{-1} E_#1 :=  s* T^{-1} e_#1 where 
	%  s = #2 * ||T^{-1} e_ref|| / || T^{-1} e_#1 || 
	%    = |10^{scale_ref}| / |10^{scale_#1}| * #2 * || e_ref|| / ||e_#1||.
	% 
	% Then, E_#1 = T ( T^{-1} E_#1 ) = s * e_#1.
	%
	% -> compute 's'! 
	%
	% Part 1: compute
	% #2 * ||e_ref|| / ||e_#1||.
	%
	\def\pgfplots@loc@TMPa{1}%
	\if1\csname b@pgfplots@unitvec@is@zero@#1\endcsname
		\def\pgfplots@loc@TMPa{0}%
	\else
		\if1\csname b@pgfplots@unitvec@is@zero@\pgfplots@apply@unit@ratio@reference\endcsname
			\def\pgfplots@loc@TMPa{0}%
		\fi
	\fi
	\if0\pgfplots@loc@TMPa
		\def\pgfmathresult{16001}%
	\else
		% note that  x^{-1} / y^{-1}  == ( x/y )^{-1} == y/x .
		% consequently, we can use our @inv@[xyz] values here:
		\pgfmathdivide@
			{\csname pgfplots@target@unit@scale@inv@#1\endcsname}%
			{\csname pgfplots@target@unit@scale@inv@\pgfplots@apply@unit@ratio@reference\endcsname}
		\pgfmathmultiply@
			{\pgfmathresult}%
			{#2}%
		\global\let\pgfplots@glob@TMPa=\pgfmathresult
		% 
		% also compute 1/s, required as temporary value:
		%\pgfmathmultiply@
		%	{\csname pgfplots@\pgfplots@apply@unit@ratio@reference @inverseveclength\endcsname}
		%	{\csname pgfplots@target@unit@scale@#1\endcsname}%
		%\ifdim#2pt=1pt
		%\else
		%	\pgfmathdivide@{\pgfmathresult}{#2}%
		%\fi
		%\global\let\pgfplots@glob@TMPb=\pgfmathresult
		%
		% Part 2: handle data scaling trafo scales:
		\begingroup
			\def\pgfplots@tmp@exponentref{0}%
			\def\pgfplots@tmp@exponentK{0}%
			\pgfplots@if{pgfplots@apply@datatrafo@\pgfplots@apply@unit@ratio@reference }{%
				\pgfplots@letcsname{pgfplots@tmp@exponentref}={pgfplots@data@scale@trafo@EXPONENT@\pgfplots@apply@unit@ratio@reference }%
			}{}%
			\pgfplots@if{pgfplots@apply@datatrafo@#1}{%
				\pgfplots@letcsname{pgfplots@tmp@exponentK}={pgfplots@data@scale@trafo@EXPONENT@#1}%
			}{}%
			\c@pgf@counta=\pgfplots@tmp@exponentref\relax
			\advance\c@pgf@counta by-\pgfplots@tmp@exponentK\relax
			\ifnum\c@pgf@counta=0
			\else
				\pgfplotsmathmultiplypowten@{\pgfplots@glob@TMPa}{\c@pgf@counta}%
				\global\let\pgfplots@glob@TMPa=\pgfmathresult
			%	\pgfplotsmathmultiplypowten@{\pgfplots@glob@TMPb}{-\c@pgf@counta}%
			%	\global\let\pgfplots@glob@TMPb=\pgfmathresult
			\fi
			\xdef\pgfplots@glob@TMPc{\the\c@pgf@counta}%
		\endgroup
		\let\pgfmathresult=\pgfplots@glob@TMPa
	\fi
%\message{\string\pgfplots@getscale@unit@vector@reltoreference{#1}{#2} (reference \pgfplots@apply@unit@ratio@reference) = \pgfmathresult.^^J}%
}

% helper for \pgfplots@check@and@apply@datatrafo@for.
\def\pgfplots@compute@number@order@for@trafo@isdimen#1\tocount#2{%
	\edef\pgfplots@loc@TMPa{\pgf@sys@tonumber{#1}}%
	\pgfmathfloatparsenumber{\pgfplots@loc@TMPa}%
	\expandafter\pgfmathfloat@decompose@E\pgfmathresult\relax#2
	\advance#2 by1\relax
}

% helper for \pgfplots@check@and@apply@datatrafo@for.
% 
\def\pgfplots@compute@number@order@for@trafo@isfloat#1\tocount#2{%
	\pgfmathfloatparsenumber{#1}%
	\expandafter\pgfmathfloat@decompose@E\pgfmathresult\relax#2\relax
	\advance#2 by1\relax
}

% Initialises the data scale transformation such that it is optimal
% for direction #1 (using its axis limits and the target scaling size).
%
% Note that it will not be applied in any way; and it may still be
% modified.
%
% PRECONDITION:
%   - all axis limits are available in float representation
%   - \pgfplots@set@default@size@options has been called before
% POSTCONDITION:
%   - the scaling transformation is set up,
\def\pgfplots@set@optimal@datatrafo@for@#1{%
	\pgfplots@if{pgfplots@apply@datatrafo@#1}{%
		% initialise data scale transformation 
		%   T(x) = 10^{q-m} * x
		%
		\ifpgfplots@disabledatascaling
			% this here is a waste of time, because the NO-OP trafo
			% will be applied to all coordinates. One could really
			% safe a lot of CPU time when disabledatascaling is enabled...
			% but it requires so much extra cases; I really don't want
			% that!
			\gdef\pgfplots@glob@TMPa{0}%
			\gdef\pgfplots@glob@TMPb{0}%
		\else
			\begingroup
				\let\data@max@order=\c@pgf@counta
				\let\data@cur@order=\c@pgf@countb
				\let\data@dimen=\pgf@xa
				\let\data@tmp=\pgf@xb
				\let\data@dimen@order=\c@pgf@countc
				\let\data@EXPONENT=\c@pgf@countd
				\expandafter\let\expandafter\pgfplots@display@min@float\csname pgfplots@#1min\endcsname
				\expandafter\let\expandafter\pgfplots@display@max@float\csname pgfplots@#1max\endcsname
				\expandafter\let\expandafter\pgfplots@data@min@float\csname pgfplots@data@#1min\endcsname
				\expandafter\let\expandafter\pgfplots@data@max@float\csname pgfplots@data@#1max\endcsname
				\ifpgfplots@autocompute@all@limits
				\else
					\pgfplotscoordmath{#1}{max}{\pgfplots@display@max@float}{\pgfplots@data@max@float}%
					\let\pgfplots@data@max@float=\pgfmathresult
					\pgfplotscoordmath{#1}{min}{\pgfplots@display@min@float}{\pgfplots@data@min@float}%
					\let\pgfplots@data@min@float=\pgfmathresult
				\fi
				%
%\message{minmax = [\pgfplots@data@min@float,\pgfplots@data@max@float]}%
				% Step 1: compute 'm', the data order
				\pgfplots@compute@number@order@for@trafo@isfloat
					\pgfplots@data@min@float
					\tocount\data@cur@order
				%
				\data@max@order=\data@cur@order
				%
				\pgfplots@compute@number@order@for@trafo@isfloat
					\pgfplots@data@max@float
					\tocount\data@cur@order
				%
				\ifnum\data@cur@order>\data@max@order
					\data@max@order=\data@cur@order
				\fi
				%
				% Step 2: compute 'q', the #1-size of the axis.
				%\expandafter\ifx\csname pgfplots@#1\endcsname\pgfutil@empty
					% We have 'width' or 'height' (I always have them).
					%
					% Use the order of these parameters.
					\def\pgfplots@loc@TMPa{#1}%
					\def\pgfplots@loc@TMPb{x}%
					\ifx\pgfplots@loc@TMPa\pgfplots@loc@TMPb
						\data@dimen=\pgfplots@width\relax
					\else
						\if1\pgfplots@compat@scaling@zunitfix@enable
							\data@dimen=\pgfplots@height\relax
						\else
							% this code here belongs to versions up to
							% 1.3.1.
							% It is now deprecated and produces small
							% pixel differences.
							\def\pgfplots@loc@TMPb{y}%
							\ifx\pgfplots@loc@TMPa\pgfplots@loc@TMPb
								\data@dimen=\pgfplots@height\relax
							\else
								\data@dimen=42pt % this is actually different from 1.3.1: there, it was UNDEFINED.
							\fi
						\fi
					\fi
					\pgfplots@compute@number@order@for@trafo@isdimen
						\data@dimen
						\tocount\data@dimen@order
					% This here is to avoid inaccuracies in the final
					% axis rectangle size, see \pgfplots@initsizes:
					%\advance\data@dimen@order by-1
				%\else
					% FIXME:
					% we have either the 'x=1cm' or 'y=1cm' option!
					% How should I initialise the trafo!?
				%	\data@dimen@order=3
				%\fi
				%
%\message{Direction #1: data max order=\the\data@max@order;  data dimen order=\the\data@dimen@order. }%
				\data@EXPONENT=\data@dimen@order
				\advance\data@EXPONENT by-\data@max@order
				% Now, I introduce a loop which shall avoid cancellation of
				% significant digits.
				%
				% Harmless Example: 
				%  if we have data shift = -3 and 
				%  max = 2e6, min = 1e6, then max-min = 1e6; T(max)-T(min) = 1e3 which is ok.
				%  In this case, the loop won't change anything.
				%
				% Critical Example:
				%  if we have data shift = -3 and
				%  max = 1980, min = 1930 then 
				%    T(max) = 1.98 and T(min) = 1.93
				%  and thus T(max)-T(min) = 0.05 . 
				%  Considering that this is the axis range
				%  in which tick labels and plot points need to be computed, we
				%  only have two or three digits left! That happens because the
				%  prefix '19' is common and is cancelled in the subtraction.
				%  Idea: while T(max)-T(min) < O(10^2) -> increase shift by +1
				%  (and make sure that T(max) < MAX_VALID_TEX_NUMBER).
				%
				\def\pgfplotscoordmathnotifydatascalesetfor##1{}% disable temporarily. We are just testing it.
				\pgfplots@loop@CONTINUEtrue
				\pgfutil@loop
				\pgfplotscoordmath{#1}{datascaletrafo set params}{\the\data@EXPONENT}{0}%
				\pgfplotscoordmath{#1}{datascaletrafo}{\pgfplots@data@min@float}%
				\let\pgfplots@min@fixed=\pgfmathresult
				\ifpgfplots@loop@CONTINUE
					\pgfplotscoordmath{#1}{datascaletrafo}{\pgfplots@data@max@float}%
					\let\pgfplots@max@fixed=\pgfmathresult
					\data@tmp=\pgfplots@max@fixed pt
%\message{Current trafo EXPONENT for #1 direction: \the\data@EXPONENT; original #1 data limits: [\pgfplots@data@min@float:\pgfplots@data@max@float]; current transformed #1 limits: [\pgfplots@min@fixed:\pgfplots@max@fixed]; cancellation check max-min running...}%
					\ifdim\data@tmp<0pt
						% I need absolute values here:
						\multiply\data@tmp by-1\relax
					\fi
					\pgfmathsubtract@{\pgfplots@max@fixed}{\pgfplots@min@fixed}%
					\data@dimen=\pgfmathresult pt
					\pgfplots@loop@CONTINUEfalse
					\ifdim\data@tmp<1500pt % a multiplication with '10' results in max = 15000 which is the upper limit.
						\ifdim\data@dimen<100pt % I guess if max-min = O(100), we have quite good accuracy
							\ifdim\data@dimen<0.0001pt
							\else
								\advance\data@EXPONENT by1
								\pgfplots@loop@CONTINUEtrue
							\fi
						\fi
					\fi
					%--------------------------------------------------
					% \ifdim\data@dimen>1200pt% FIXME : is this here ok!? CHECK IT!
					% 	\ifdim\data@dimen>7999pt
					% 		\advance\data@EXPONENT by-2
					% 	\else
					% 		\advance\data@EXPONENT by-1
					% 	\fi
					% 	\pgfplots@loop@CONTINUEfalse
					% \fi
					%-------------------------------------------------- 
				\pgfutil@repeat
				\xdef\pgfplots@glob@TMPa{\the\data@EXPONENT}%
				\xdef\pgfplots@glob@TMPb{\pgfplots@min@fixed}%
			\endgroup
		\fi
		% COMPLETE INITIALISATION:
%\message{Initialising the data scale transformation in direction #1 to 10^\pgfplots@glob@TMPa*#1 - \pgfplots@glob@TMPb...^^J}%
		\pgfplotscoordmath{#1}{datascaletrafo set params}{\pgfplots@glob@TMPa}{\pgfplots@glob@TMPb}%
	}{%
		% case apply trafo == false:
		\pgfplotscoordmath{#1}{datascaletrafo set params}{0}{0}%
	}%
}


\def\pgfplots@set@optimal@datatrafos@allaxes{%
	\pgfplots@letcsname pgfplots@xmin@unscaled@as@float={pgfplots@xmin}%
	\pgfplots@letcsname pgfplots@xmax@unscaled@as@float={pgfplots@xmax}%
	%
	\pgfplots@letcsname pgfplots@ymin@unscaled@as@float={pgfplots@ymin}%
	\pgfplots@letcsname pgfplots@ymax@unscaled@as@float={pgfplots@ymax}%
	%
	\pgfplots@letcsname pgfplots@zmin@unscaled@as@float={pgfplots@zmin}%
	\pgfplots@letcsname pgfplots@zmax@unscaled@as@float={pgfplots@zmax}%
	%
	\pgfplots@ifneeds@one@uniform@datascale{%
		% Ah - we have to ensure that there is ONE common scale for
		% each unit (x, y, and z have the same).
		%
		% In this case, we need to choose one of the transformations
		% and apply it to all axes -- such that each axis gets the
		% same scale.
		%
		% this mode is used for axis equal and its variants.
		%
		% The strategy to fix the transformation is as follows:
		% 1. we assume that axis limits will be enlarged in order to
		% satisfy 'scale uniformly'. 
		% 2. we assume that the LARGEST axis limit dominates the
		% others.
		% 3. if one of the axes does not have datascaling (i.e. is
		% log scale), we disable all other datascalings.
		%
		% Consequently, we search for the axis with the largest limit
		% - and copy its data scaling to all other axes. If one of the
		% axes is log, that one overrules it and all data scaling
		% effects are disabled..
		\begingroup
		\let\pgfplots@data@scale@trafo@EXPONENT@common=\pgfutil@empty
		\def\pgfplots@data@scale@trafo@EXPONENT@common@arg{-}% this should not match anything in this context.
		\pgfplots@if{pgfplots@apply@datatrafo@x}{%
		}{%
			\def\pgfplots@data@scale@trafo@EXPONENT@common{{0}{0}}% disable scaling!
		}%
		\pgfplots@if{pgfplots@apply@datatrafo@y}{%
		}{%
			\def\pgfplots@data@scale@trafo@EXPONENT@common{{0}{0}}% disable scaling!
		}%
		\ifpgfplots@threedim
			\pgfplots@if{pgfplots@apply@datatrafo@z}{%
			}{%
				\def\pgfplots@data@scale@trafo@EXPONENT@common{{0}{0}}% disable scaling!
			}%
		\fi
		\ifx\pgfplots@data@scale@trafo@EXPONENT@common\pgfutil@empty
			% ah - we still need to compute one. ok, search for the
			% largest limit.
			%
			\pgfplots@get@axis@with@largest@limits
			\let\pgfplots@data@scale@trafo@EXPONENT@common@arg=\pgfplotsretval
			%
			% ok, compute data scaling transformation for the target axis:
			\expandafter\pgfplots@set@optimal@datatrafo@for@\pgfplots@data@scale@trafo@EXPONENT@common@arg%
			%
			\pgfplotscoordmath{\pgfplots@data@scale@trafo@EXPONENT@common@arg}{datascaletrafo get params}%
			\let\pgfplots@data@scale@trafo@EXPONENT@common=\pgfmathresult%
		\else
			% hm. early-out - we already have the scaling trafo.
			% return it.
		\fi
		\global\let\pgfplots@glob@TMPa=\pgfplots@data@scale@trafo@EXPONENT@common
		\global\let\pgfplots@glob@TMPb=\pgfplots@data@scale@trafo@EXPONENT@common@arg
		\endgroup
		%
		\xdef\pgfplots@glob@TMPc{\expandafter\pgfutil@firstoftwo\pgfplots@glob@TMPa}%
%\message{using datascaletrafo of axis '\pgfplots@glob@TMPb' for each axis.^^J}%
		%
		\def\pgfplots@loc@TMPd##1{%
			\if ##1\pgfplots@glob@TMPb
				% we need to set the scaling trafo for the target direction
				% (was lost after \endgroup)
				\def\pgfplots@loc@TMPa{\pgfplotscoordmath{##1}{datascaletrafo set params}}%
				\expandafter\pgfplots@loc@TMPa\pgfplots@glob@TMPa%
			\else
				\pgfplotscoordmath{##1}{datascaletrafo set params}{\pgfplots@glob@TMPc}{0}%
				\pgfplotscoordmath{##1}{datascaletrafo}{\csname pgfplots@##1min\endcsname}%
%\message{Initialising the data scale transformation in direction ##1 to 10^\pgfplots@glob@TMPc*##1 - \pgfmathresult...^^J}%
				\pgfplotscoordmath{##1}{datascaletrafo set params}{\pgfplots@glob@TMPc}{\pgfmathresult}%
			\fi
		}%
		\pgfplots@loc@TMPd x%
		\pgfplots@loc@TMPd y%
		\ifpgfplots@threedim
			\pgfplots@loc@TMPd z%
		\fi
	}{%
		% optimize individually:
		\pgfplots@set@optimal@datatrafo@for@ x%
		\pgfplots@set@optimal@datatrafo@for@ y%
		\ifpgfplots@threedim
			\pgfplots@set@optimal@datatrafo@for@ z%
		\fi
	}%
	%
}%

% Defines \pgfplotsretval to be one of x, y, or z, such that the
% return value indicates the axis with largest untransformed axis
% limits.
\def\pgfplots@get@axis@with@largest@limits{%
	\begingroup
	\let\pgfplotsretval@extreme=\pgfutil@empty
	\let\pgfplotsretval@extreme@arg=\pgfutil@empty
	\def\pgfplots@@##1{%
		% compute axis range for axis ##1 ...
		\pgfplotscoordmath{default}{parsenumber}{\csname pgfplots@##1min\endcsname}%
		\let\pgfplots@loc@TMPa=\pgfmathresult
		\pgfplotscoordmath{default}{parsenumber}{\csname pgfplots@##1max\endcsname}%
		\pgfplotscoordmath{default}{op}{subtract}{{\pgfmathresult}{\pgfplots@loc@TMPa}}%
		% ... ok, it is in \pgfmathresult.
		\let\candidate=\pgfmathresult
		\ifx\pgfplotsretval@extreme@arg\pgfutil@empty
			% ah: no extreme value so far. use ours.
			\def\pgfplotsretval@extreme@arg{##1}%
			\let\pgfplotsretval@extreme=\candidate
		\else
			\pgfplotscoordmath{default}{if less than}{\pgfplotsretval@extreme}{\candidate}{%
				% update extreme value:
				\def\pgfplotsretval@extreme@arg{##1}%
				\let\pgfplotsretval@extreme=\candidate
			}{%
			}%
		\fi
	}%
	\pgfplots@@ x%
	\pgfplots@@ y%
	\ifpgfplots@threedim
		\pgfplots@@ z%
	\fi
	\let\pgfplotsretval=\pgfplotsretval@extreme@arg
	\pgfmath@smuggleone\pgfplotsretval
	\endgroup
}%

% Initialises the data scale transformation and applies it to any
% user specified options.
%
% PRECONDITION:
%   - all axis limits are available in float representation
%   - \pgfplots@set@default@size@options has been called before
%   - the scaling transformation for direction x is set up
%   (\pgfplots@set@optimal@datatrafo@for@),
% POSTCONDITION:
%   - all axis limits are transformed, but no other axis inputs.
%
% Unit vectors and other axis input parameters will be scaled later.
%
% @see \pgfplots@check@and@apply@datatrafo@for
\def\pgfplots@apply@datatrafo@to@axis@limits#1{%
	\pgfplots@if{pgfplots@apply@datatrafo@#1}{%
		% Transform axis limits:
%\message{#1- display limits BEFORE data transformation: [\csname pgfplots@#1min\endcsname:\csname pgfplots@#1max\endcsname]^^J}%
		\pgfplotscoordmath{#1}{datascaletrafo}{\csname pgfplots@#1min\endcsname}%
		\expandafter\global\expandafter\let\csname pgfplots@#1min\endcsname=\pgfmathresult
		%
		\pgfplotscoordmath{#1}{datascaletrafo}{\csname pgfplots@#1max\endcsname}%
		\expandafter\global\expandafter\let\csname pgfplots@#1max\endcsname=\pgfmathresult
%\message{#1- display limits after data transformation: [\csname pgfplots@#1min\endcsname:\csname pgfplots@#1max\endcsname]^^J}%
	}{%
		% case apply trafo == false:
		\expandafter\let\csname pgfplots@#1min@unscaled@as@float\endcsname=\pgfutil@empty
		\expandafter\let\csname pgfplots@#1max@unscaled@as@float\endcsname=\pgfutil@empty
	}%
}
