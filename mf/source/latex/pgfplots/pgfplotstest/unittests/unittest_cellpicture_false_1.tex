\documentclass[a4paper]{article}

\usepackage{pgfplots}

% ATTENTION : this "unit test" did never really work. It is
% experimental

\begin{document}

\pgfplotsset{
	all axes/.style={xmin=-1,xmax=10},
	first axis/.style={all axes},
	second axis/.style={all axes, axis x line=none, axis y line=right,xshift=1cm,
		ymin=-10,
		ymax=100,
		},
}
\def\showorigin{%
	\node[pin=135:{(0,0)},fill=black,circle,scale=0.3] at (0pt,0pt) {};
}
\begin{tikzpicture}
%\tracingcommands=2\tracingmacros=2
	\begin{axis}[title=Normal anchor,first axis]
	\addplot {x};
	\end{axis}
	\begin{axis}[second axis]
	\end{axis}
	\showorigin
\end{tikzpicture}

\begin{tikzpicture}
%\tracingcommands=2\tracingmacros=2
	\pgfplotsset{cell picture=false}
	\begin{axis}[title=Internal anchor,first axis]
	\addplot {x};
	\end{axis}
	\begin{axis}[second axis]
	\end{axis}
	\showorigin
\end{tikzpicture}

\begin{tikzpicture}
%\tracingcommands=2\tracingmacros=2
	\pgfplotsset{cell picture=false,disabledatascaling}
	\begin{axis}[title=Internal anchor and disabledatascaling,first axis]
	\addplot {x};
	\end{axis}
	\begin{axis}[second axis]
	\end{axis}
	\showorigin
\end{tikzpicture}
\end{document}

