\documentclass{article}
% translate with >> pdflatex -shell-escape <file>

% This file is used as unit test for pgfplots, copyright by Christian Feuersaenger.
% 
% See
%   http://pgfplots.sourceforge.net/pgfplots.pdf
% for pgfplots.
%
% Any required input files (for <plot table> or <plot file> or the table package) can be downloaded
% at
% http://www.ctan.org/tex-archive/graphics/pgf/contrib/pgfplots/doc/latex/
% and
% http://www.ctan.org/tex-archive/graphics/pgf/contrib/pgfplots/doc/latex/plotdata/

\usepackage{pgfplots}
\pgfplotsset{compat=1.3}

\pagestyle{empty}

\begin{document}

% #1: tex expr
% #2: gnuplot expr or empty
% #3: options
\def\testexpression#1#2#3{%
	\subsection{$#1$}
	\begin{tikzpicture}
		\begin{axis}[title={$#1$},#3]
			\addplot+[mark=] {#1};
			\def\gnuplotexpr{#2}%
			\ifx\gnuplotexpr\empty
				\def\gnuplotexpr{#1}%
			\fi
			\addplot+[mark=] gnuplot {\gnuplotexpr};
			\legend{\TeX,gnuplot}
		\end{axis}
	\end{tikzpicture}%
}%

\testexpression{x*5*(1-100)/(1-x)}{}{domain=1000:3000}

Das liegt an der relativen genauigkeit und an der enge des datenbereichs:

\end{document}
