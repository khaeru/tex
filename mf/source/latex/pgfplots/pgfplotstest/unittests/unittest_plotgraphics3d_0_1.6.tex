\documentclass{article}
\usepackage{pgfplots}

\pgfplotsset{compat=1.6}

\begin{document}

\def\experiment[#1]{%
    \begin{tikzpicture}
%	\tracingmacros=2 \tracingcommands=2
    \begin{axis}[3d box, width=9cm,height=9.25cm, minor z tick num=2,
    xlabel=Amplitde,
    ylabel=Mittelwert,
    zlabel=H\"aufigkeit,
    %xtick=\empty,
    %ytick=\empty,
    %ztick=\empty,
	title={\ttfamily #1},
	y dir=reverse,
	#1,
	]
    \addplot3 graphics[debug,points={%
    (0.056377,1.635,0) => (202.5066,36.135)
    (4.7634,1.6667,0) => (360.5972,95.6072)
    (4.7465,-1.8814,0) => (232.6191,178.4166)
    (0.31545,0.19947,88.5) => (159.5963,185.1919)
    }]
    {3dcolumnchart.png};
    \node at (axis cs:-1.5,0.5,490) [inner sep=0pt, pin={[pin edge={thick,black},align=left]145:Interesting\\Data Point}] {};
    \end{axis}
    \end{tikzpicture}%
}%

\experiment[]~%
\experiment[zmin=0]

\vskip1cm
\experiment[zmax=50]~%
\experiment[zmin=0,zmax=50]

\end{document}
