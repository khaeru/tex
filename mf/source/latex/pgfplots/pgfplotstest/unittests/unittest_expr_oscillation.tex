\documentclass{article}
% translate with >> pdflatex -shell-escape <file>

% This file is used as unit test for pgfplots, copyright by Christian Feuersaenger.
% 
% See
%   http://pgfplots.sourceforge.net/pgfplots.pdf
% for pgfplots.
%
% Any required input files (for <plot table> or <plot file> or the table package) can be downloaded
% at
% http://www.ctan.org/tex-archive/graphics/pgf/contrib/pgfplots/doc/latex/
% and
% http://www.ctan.org/tex-archive/graphics/pgf/contrib/pgfplots/doc/latex/plotdata/

\usepackage{pgfplots}
\pgfplotsset{compat=newest}

\pagestyle{empty}

\usepgfplotslibrary{clickable}

\begin{document}

\begin{tikzpicture}
	\begin{semilogxaxis}[domain=1e-2:1e3, samples=101]
		\addplot[black] {(1+2*x*(1+x+252*x))/(1+4*x*(1+x+252*x))};
		\addplot[red] gnuplot {(1+2*x*(1+x+252*x))/(1+4*x*(1+x+252*x))};
	\end{semilogxaxis}
\end{tikzpicture}

\begin{tikzpicture}
	\begin{axis}[domain=1e-2:1e3, domain=400:500, clickable coords={COORD =(xy)},samples=101]
		\addplot[black,mark=*,only marks] {(1+2*x*(1+x+252*x))/(1+4*x*(1+x+252*x))};
		\addplot[red] gnuplot {(1+2*x*(1+x+252*x))/(1+4*x*(1+x+252*x))};
	\end{axis}
\end{tikzpicture}

{
\pgfkeys{/pgf/fpu}
\def\x{406.0}
\pgfmathparse{(1+2*\x*(1+\x+252*\x))/(1+4*\x*(1+\x+252*\x))}
\pgfmathresult
}

\end{document}

