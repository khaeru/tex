%%%%%%%%%%%%%%%%%%%%%%%%%%%%%%%%%%%%%%%%%%%%%%%%%%%%%%%%%%%%%%%%%%
% slides.tex
% Paul Kishimoto <mail@paul.kishimoto.name>
%
% In lieu of coding my own document class, here is a collection
% of package invocations and configuration commands. Use via:
% %%%%%%%%%%%%%%%%%%%%%%%%%%%%%%%%%%%%%%%%%%%%%%%%%%%%%%%%%%%%%%%%%%
% slides.tex
% Paul Kishimoto <mail@paul.kishimoto.name>
%
% In lieu of coding my own document class, here is a collection
% of package invocations and configuration commands. Use via:
% %%%%%%%%%%%%%%%%%%%%%%%%%%%%%%%%%%%%%%%%%%%%%%%%%%%%%%%%%%%%%%%%%%
% slides.tex
% Paul Kishimoto <mail@paul.kishimoto.name>
%
% In lieu of coding my own document class, here is a collection
% of package invocations and configuration commands. Use via:
% %%%%%%%%%%%%%%%%%%%%%%%%%%%%%%%%%%%%%%%%%%%%%%%%%%%%%%%%%%%%%%%%%%
% slides.tex
% Paul Kishimoto <mail@paul.kishimoto.name>
%
% In lieu of coding my own document class, here is a collection
% of package invocations and configuration commands. Use via:
% \input{slides}
%%%%%%%%%%%%%%%%%%%%%%%%%%%%%%%%%%%%%%%%%%%%%%%%%%%%%%%%%%%%%%%%%%

%% Suppress Warnings
% See http://conway.rutgers.edu/~ccshan/wiki/blog/posts/Beamer_and
% _mathdesign . The following line must be included before the
% \documentclass line in any source file:
% \expandafter\let\csname ver@amssymb.sty\endcsname\empty
\expandafter\let\csname ver@amssymb.sty\endcsname\relax

%% Packages
\input{common}
% ctable:
\usepackage{ctable}
% natbib: allow some squishing of the bibliography to save space,
% e.g. by using:
%
%   \setlength{\bibsep}{0.25\baselineskip}
%
% ...in the preamble.
\usepackage[square]{natbib}
  \newcommand{\newblock}{!}

%% Beamer formatting
\usetheme{khaeru}

%% Usual LaTeX Formatting
% paragraph spacing for readability
\setlength{\parskip}{0.5\baselineskip}

%% Custom commands and environments
\newcommand{\km}{~\mathrm{km}}

% author information
\author{Paul Natsuo Kishimoto\texorpdfstring{\\ }{ }\texttt{<\href{mailto:mail@paul.kishimoto.name}{mail@paul.kishimoto.name}>}}

% AIAA-style bibliography
\bibliographystyle{aiaa}

% From: http://newsgroups.derkeiler.com/Archive/Comp/comp.text.tex/2008-11/msg00945.html
\newenvironment{changemargin}[2]{%
\begin{list}{}{%
\setlength{\topsep}{0pt}%
\setlength{\leftmargin}{#1}%
\setlength{\rightmargin}{#2}%
\setlength{\listparindent}{\parindent}%
\setlength{\itemindent}{\parindent}%
\setlength{\parsep}{\parskip}%
}%
\item[]}{\end{list}}

% Make one image take up the entire slide content area in beamer,.:
% centered/centred full-screen image, with title:
% This uses the whole screen except for the 1cm border around it
% all. 128x96mm
\newcommand{\titledFrameImage}[3][]{
\begin{frame}{#2}
\begin{changemargin}{-1cm}{-1cm}
\begin{center}
\includegraphics[width=108mm,height=0.8\textheight,keepaspectratio,#1]{#3}
\end{center}
\end{changemargin}
\end{frame}
}

% Make one image take up the entire slide content area in beamer.:
% centered/centred full-screen image, no title:
% This uses the whole screen except for the 1cm border around it
% all. 128x96mm
\newcommand{\plainFrameImage}[1]{
\begin{frame}[plain]
%\begin{changemargin}{-1cm}{-1cm}
\begin{center}
\includegraphics[width=108mm,height=76mm,keepaspectratio]{#1}
\end{center}
%\end{changemargin}
\end{frame}
}

% Make one image take up the entire slide area, including borders, in beamer.:
% centered/centred full-screen image, no title:
% This uses the entire whole screen
\newcommand{\maxFrameImage}[2][]{
{\setbeamercolor{background canvas}{bg=blue058}
\begin{frame}[plain]
\begin{changemargin}{-1cm}{-1cm}
\begin{center}
\includegraphics[width=\paperwidth,height=\paperheight,keepaspectratio,#1]
{#2}
\end{center}
\end{changemargin}
\end{frame}
}
}

%%%%%%%%%%%%%%%%%%%%%%%%%%%%%%%%%%%%%%%%%%%%%%%%%%%%%%%%%%%%%%%%%%

%% Suppress Warnings
% See http://conway.rutgers.edu/~ccshan/wiki/blog/posts/Beamer_and
% _mathdesign . The following line must be included before the
% \documentclass line in any source file:
% \expandafter\let\csname ver@amssymb.sty\endcsname\empty
\expandafter\let\csname ver@amssymb.sty\endcsname\relax

%% Packages
%%%%%%%%%%%%%%%%%%%%%%%%%%%%%%%%%%%%%%%%%%%%%%%%%%%%%%%%%%%%%%%%%%
% common.tex
% Paul Kishimoto <mail@paul.kishimoto.name>
%
% This file is meant to be imported by other templates.
%%%%%%%%%%%%%%%%%%%%%%%%%%%%%%%%%%%%%%%%%%%%%%%%%%%%%%%%%%%%%%%%%%

% babel: real Canadian English!
\usepackage[english,canadian]{babel}
% charter: Bitstream Charter BT font 
\usepackage{charter}
% datetime: preferred little-endian date and time formats
\usepackage{datetime}
  % e.g. 01 January 2008
  \newdateformat{fulldate}{\twodigit{\THEDAY}~\monthname[\THEMONTH] \THEYEAR}
  % e.g. 01 Jan 08
  \newdateformat{meddate}{\THEDAY~\shortmonthname[\THEMONTH] \THEYEAR}
  % e.g. 2008-12-31
  \newdateformat{numdate}{\THEYEAR-\THEMONTH-\THEDAY}
% fontenc: encoding for fonts
\usepackage[T1]{fontenc}
% inputenc: source files are in UTF-8. 'ucs' must be loaded first
\usepackage{ucs}
\usepackage[utf8x]{inputenc}
% listings: source code listings
\usepackage{listings}
  \lstset{basicstyle=\ttfamily,frame=single}
% mathdesign: charter font for math as well
\usepackage[bitstream-charter]{mathdesign}
% nomencl: to use \nomenclature. Note the 'rubber' build system
% for LaTEX (http://www.pps.jussieu.fr/~beffara/soft/rubber/) does
% not yet include support for nomencl, so some support is hacked
% in here.
\usepackage{nomencl}
% rubber: module index
% rubber: onchange $base.nlo "makeindex '$base.nlo' -s nomencl.ist -o '$base.nls'"
% rubber: watch $base.nls
% rubber: clean $base.ilg $base.nlo $base.nls $base.out
% pdfpages: wholesale inclusion of pages from other PDF documents
\usepackage{pdfpages}
  \includepdfset{pages=-}
% beamer/non-beamer split
\makeatletter
\@ifpackageloaded{beamerbasercs}{
  % subfigure: create subfigures
  \usepackage{subfigure}
}{
  % subfig/subfigure: create subfigures
  \usepackage{subfig}
  % titling: access the document title and date with \thetitle, \thedate
  \usepackage{titling}
}
\makeatother

\newcommand\tabref{\tablename~\ref}
\newcommand\tref{\tabref}
\newcommand{\TODO}[1]{
  \colorbox{black}{
    \textcolor{white}{
      \bfseries TODO---#1.
    }
  } \\
}





% ctable:
\usepackage{ctable}
% natbib: allow some squishing of the bibliography to save space,
% e.g. by using:
%
%   \setlength{\bibsep}{0.25\baselineskip}
%
% ...in the preamble.
\usepackage[square]{natbib}
  \newcommand{\newblock}{!}

%% Beamer formatting
\usetheme{khaeru}

%% Usual LaTeX Formatting
% paragraph spacing for readability
\setlength{\parskip}{0.5\baselineskip}

%% Custom commands and environments
\newcommand{\km}{~\mathrm{km}}

% author information
\author{Paul Natsuo Kishimoto\texorpdfstring{\\ }{ }\texttt{<\href{mailto:mail@paul.kishimoto.name}{mail@paul.kishimoto.name}>}}

% AIAA-style bibliography
\bibliographystyle{aiaa}

% From: http://newsgroups.derkeiler.com/Archive/Comp/comp.text.tex/2008-11/msg00945.html
\newenvironment{changemargin}[2]{%
\begin{list}{}{%
\setlength{\topsep}{0pt}%
\setlength{\leftmargin}{#1}%
\setlength{\rightmargin}{#2}%
\setlength{\listparindent}{\parindent}%
\setlength{\itemindent}{\parindent}%
\setlength{\parsep}{\parskip}%
}%
\item[]}{\end{list}}

% Make one image take up the entire slide content area in beamer,.:
% centered/centred full-screen image, with title:
% This uses the whole screen except for the 1cm border around it
% all. 128x96mm
\newcommand{\titledFrameImage}[3][]{
\begin{frame}{#2}
\begin{changemargin}{-1cm}{-1cm}
\begin{center}
\includegraphics[width=108mm,height=0.8\textheight,keepaspectratio,#1]{#3}
\end{center}
\end{changemargin}
\end{frame}
}

% Make one image take up the entire slide content area in beamer.:
% centered/centred full-screen image, no title:
% This uses the whole screen except for the 1cm border around it
% all. 128x96mm
\newcommand{\plainFrameImage}[1]{
\begin{frame}[plain]
%\begin{changemargin}{-1cm}{-1cm}
\begin{center}
\includegraphics[width=108mm,height=76mm,keepaspectratio]{#1}
\end{center}
%\end{changemargin}
\end{frame}
}

% Make one image take up the entire slide area, including borders, in beamer.:
% centered/centred full-screen image, no title:
% This uses the entire whole screen
\newcommand{\maxFrameImage}[2][]{
{\setbeamercolor{background canvas}{bg=blue058}
\begin{frame}[plain]
\begin{changemargin}{-1cm}{-1cm}
\begin{center}
\includegraphics[width=\paperwidth,height=\paperheight,keepaspectratio,#1]
{#2}
\end{center}
\end{changemargin}
\end{frame}
}
}

%%%%%%%%%%%%%%%%%%%%%%%%%%%%%%%%%%%%%%%%%%%%%%%%%%%%%%%%%%%%%%%%%%

%% Suppress Warnings
% See http://conway.rutgers.edu/~ccshan/wiki/blog/posts/Beamer_and
% _mathdesign . The following line must be included before the
% \documentclass line in any source file:
% \expandafter\let\csname ver@amssymb.sty\endcsname\empty
\expandafter\let\csname ver@amssymb.sty\endcsname\relax

%% Packages
%%%%%%%%%%%%%%%%%%%%%%%%%%%%%%%%%%%%%%%%%%%%%%%%%%%%%%%%%%%%%%%%%%
% common.tex
% Paul Kishimoto <mail@paul.kishimoto.name>
%
% This file is meant to be imported by other templates.
%%%%%%%%%%%%%%%%%%%%%%%%%%%%%%%%%%%%%%%%%%%%%%%%%%%%%%%%%%%%%%%%%%

% babel: real Canadian English!
\usepackage[english,canadian]{babel}
% charter: Bitstream Charter BT font 
\usepackage{charter}
% datetime: preferred little-endian date and time formats
\usepackage{datetime}
  % e.g. 01 January 2008
  \newdateformat{fulldate}{\twodigit{\THEDAY}~\monthname[\THEMONTH] \THEYEAR}
  % e.g. 01 Jan 08
  \newdateformat{meddate}{\THEDAY~\shortmonthname[\THEMONTH] \THEYEAR}
  % e.g. 2008-12-31
  \newdateformat{numdate}{\THEYEAR-\THEMONTH-\THEDAY}
% fontenc: encoding for fonts
\usepackage[T1]{fontenc}
% inputenc: source files are in UTF-8. 'ucs' must be loaded first
\usepackage{ucs}
\usepackage[utf8x]{inputenc}
% listings: source code listings
\usepackage{listings}
  \lstset{basicstyle=\ttfamily,frame=single}
% mathdesign: charter font for math as well
\usepackage[bitstream-charter]{mathdesign}
% nomencl: to use \nomenclature. Note the 'rubber' build system
% for LaTEX (http://www.pps.jussieu.fr/~beffara/soft/rubber/) does
% not yet include support for nomencl, so some support is hacked
% in here.
\usepackage{nomencl}
% rubber: module index
% rubber: onchange $base.nlo "makeindex '$base.nlo' -s nomencl.ist -o '$base.nls'"
% rubber: watch $base.nls
% rubber: clean $base.ilg $base.nlo $base.nls $base.out
% pdfpages: wholesale inclusion of pages from other PDF documents
\usepackage{pdfpages}
  \includepdfset{pages=-}
% beamer/non-beamer split
\makeatletter
\@ifpackageloaded{beamerbasercs}{
  % subfigure: create subfigures
  \usepackage{subfigure}
}{
  % subfig/subfigure: create subfigures
  \usepackage{subfig}
  % titling: access the document title and date with \thetitle, \thedate
  \usepackage{titling}
}
\makeatother

\newcommand\tabref{\tablename~\ref}
\newcommand\tref{\tabref}
\newcommand{\TODO}[1]{
  \colorbox{black}{
    \textcolor{white}{
      \bfseries TODO---#1.
    }
  } \\
}





% ctable:
\usepackage{ctable}
% natbib: allow some squishing of the bibliography to save space,
% e.g. by using:
%
%   \setlength{\bibsep}{0.25\baselineskip}
%
% ...in the preamble.
\usepackage[square]{natbib}
  \newcommand{\newblock}{!}

%% Beamer formatting
\usetheme{khaeru}

%% Usual LaTeX Formatting
% paragraph spacing for readability
\setlength{\parskip}{0.5\baselineskip}

%% Custom commands and environments
\newcommand{\km}{~\mathrm{km}}

% author information
\author{Paul Natsuo Kishimoto\texorpdfstring{\\ }{ }\texttt{<\href{mailto:mail@paul.kishimoto.name}{mail@paul.kishimoto.name}>}}

% AIAA-style bibliography
\bibliographystyle{aiaa}

% From: http://newsgroups.derkeiler.com/Archive/Comp/comp.text.tex/2008-11/msg00945.html
\newenvironment{changemargin}[2]{%
\begin{list}{}{%
\setlength{\topsep}{0pt}%
\setlength{\leftmargin}{#1}%
\setlength{\rightmargin}{#2}%
\setlength{\listparindent}{\parindent}%
\setlength{\itemindent}{\parindent}%
\setlength{\parsep}{\parskip}%
}%
\item[]}{\end{list}}

% Make one image take up the entire slide content area in beamer,.:
% centered/centred full-screen image, with title:
% This uses the whole screen except for the 1cm border around it
% all. 128x96mm
\newcommand{\titledFrameImage}[3][]{
\begin{frame}{#2}
\begin{changemargin}{-1cm}{-1cm}
\begin{center}
\includegraphics[width=108mm,height=0.8\textheight,keepaspectratio,#1]{#3}
\end{center}
\end{changemargin}
\end{frame}
}

% Make one image take up the entire slide content area in beamer.:
% centered/centred full-screen image, no title:
% This uses the whole screen except for the 1cm border around it
% all. 128x96mm
\newcommand{\plainFrameImage}[1]{
\begin{frame}[plain]
%\begin{changemargin}{-1cm}{-1cm}
\begin{center}
\includegraphics[width=108mm,height=76mm,keepaspectratio]{#1}
\end{center}
%\end{changemargin}
\end{frame}
}

% Make one image take up the entire slide area, including borders, in beamer.:
% centered/centred full-screen image, no title:
% This uses the entire whole screen
\newcommand{\maxFrameImage}[2][]{
{\setbeamercolor{background canvas}{bg=blue058}
\begin{frame}[plain]
\begin{changemargin}{-1cm}{-1cm}
\begin{center}
\includegraphics[width=\paperwidth,height=\paperheight,keepaspectratio,#1]
{#2}
\end{center}
\end{changemargin}
\end{frame}
}
}

%%%%%%%%%%%%%%%%%%%%%%%%%%%%%%%%%%%%%%%%%%%%%%%%%%%%%%%%%%%%%%%%%%

%% Suppress Warnings
% See http://conway.rutgers.edu/~ccshan/wiki/blog/posts/Beamer_and
% _mathdesign . The following line must be included before the
% \documentclass line in any source file:
% \expandafter\let\csname ver@amssymb.sty\endcsname\empty
\expandafter\let\csname ver@amssymb.sty\endcsname\relax

%% Packages
%%%%%%%%%%%%%%%%%%%%%%%%%%%%%%%%%%%%%%%%%%%%%%%%%%%%%%%%%%%%%%%%%%
% common.tex
% Paul Kishimoto <mail@paul.kishimoto.name>
%
% This file is meant to be imported by other templates.
%%%%%%%%%%%%%%%%%%%%%%%%%%%%%%%%%%%%%%%%%%%%%%%%%%%%%%%%%%%%%%%%%%

% babel: real Canadian English!
\usepackage[english,canadian]{babel}
% charter: Bitstream Charter BT font 
\usepackage{charter}
% datetime: preferred little-endian date and time formats
\usepackage{datetime}
  % e.g. 01 January 2008
  \newdateformat{fulldate}{\twodigit{\THEDAY}~\monthname[\THEMONTH] \THEYEAR}
  % e.g. 01 Jan 08
  \newdateformat{meddate}{\THEDAY~\shortmonthname[\THEMONTH] \THEYEAR}
  % e.g. 2008-12-31
  \newdateformat{numdate}{\THEYEAR-\THEMONTH-\THEDAY}
% fontenc: encoding for fonts
\usepackage[T1]{fontenc}
% inputenc: source files are in UTF-8. 'ucs' must be loaded first
\usepackage{ucs}
\usepackage[utf8x]{inputenc}
% listings: source code listings
\usepackage{listings}
  \lstset{basicstyle=\ttfamily,frame=single}
% mathdesign: charter font for math as well
\usepackage[bitstream-charter]{mathdesign}
% nomencl: to use \nomenclature. Note the 'rubber' build system
% for LaTEX (http://www.pps.jussieu.fr/~beffara/soft/rubber/) does
% not yet include support for nomencl, so some support is hacked
% in here.
\usepackage{nomencl}
% rubber: module index
% rubber: onchange $base.nlo "makeindex '$base.nlo' -s nomencl.ist -o '$base.nls'"
% rubber: watch $base.nls
% rubber: clean $base.ilg $base.nlo $base.nls $base.out
% pdfpages: wholesale inclusion of pages from other PDF documents
\usepackage{pdfpages}
  \includepdfset{pages=-}
% beamer/non-beamer split
\makeatletter
\@ifpackageloaded{beamerbasercs}{
  % subfigure: create subfigures
  \usepackage{subfigure}
}{
  % subfig/subfigure: create subfigures
  \usepackage{subfig}
  % titling: access the document title and date with \thetitle, \thedate
  \usepackage{titling}
}
\makeatother

\newcommand\tabref{\tablename~\ref}
\newcommand\tref{\tabref}
\newcommand{\TODO}[1]{
  \colorbox{black}{
    \textcolor{white}{
      \bfseries TODO---#1.
    }
  } \\
}





% ctable:
\usepackage{ctable}
% natbib: allow some squishing of the bibliography to save space,
% e.g. by using:
%
%   \setlength{\bibsep}{0.25\baselineskip}
%
% ...in the preamble.
\usepackage[square]{natbib}
  \newcommand{\newblock}{!}

%% Beamer formatting
\useoutertheme{split}
\useinnertheme{rounded}
\usecolortheme{khaeru}
\usefonttheme{structurebold}
\usefonttheme{serif}
\setbeamertemplate{navigation symbols}{}
% frame number in footer
\expandafter\def\expandafter\insertshorttitle\expandafter{%
  \insertshorttitle\hfill%
  \insertframenumber\,/\,\inserttotalframenumber}

%% Usual LaTeX Formatting
% paragraph spacing for readability
\setlength{\parskip}{0.5\baselineskip}

%% Custom commands and environments
\newcommand{\km}{~\mathrm{km}}
\newcommand{\citeme}{\textbf{---CITE---}}

% author information
\author{Paul Natsuo Kishimoto\texorpdfstring{\\}{~}\texttt{<\href{mailto:mail@paul.kishimoto.name}{mail@paul.kishimoto.name}>}}

% AIAA-style bibliography
\bibliographystyle{aiaa}

% From: http://newsgroups.derkeiler.com/Archive/Comp/comp.text.tex/2008-11/msg00945.html
\newenvironment{changemargin}[2]{%
\begin{list}{}{%
\setlength{\topsep}{0pt}%
\setlength{\leftmargin}{#1}%
\setlength{\rightmargin}{#2}%
\setlength{\listparindent}{\parindent}%
\setlength{\itemindent}{\parindent}%
\setlength{\parsep}{\parskip}%
}%
\item[]}{\end{list}}

% Make one image take up the entire slide content area in beamer,.:
% centered/centred full-screen image, with title:
% This uses the whole screen except for the 1cm border around it
% all. 128x96mm
\newcommand{\titledFrameImage}[3][]{
\begin{frame}{#2}
\begin{changemargin}{-1cm}{-1cm}
\begin{center}
\includegraphics[width=108mm,height=0.8\textheight,keepaspectratio,#1]{#3}
\end{center}
\end{changemargin}
\end{frame}
}

% Make one image take up the entire slide content area in beamer.:
% centered/centred full-screen image, no title:
% This uses the whole screen except for the 1cm border around it
% all. 128x96mm
\newcommand{\plainFrameImage}[1]{
\begin{frame}[plain]
%\begin{changemargin}{-1cm}{-1cm}
\begin{center}
\includegraphics[width=108mm,height=76mm,keepaspectratio]{#1}
\end{center}
%\end{changemargin}
\end{frame}
}

% Make one image take up the entire slide area, including borders, in beamer.:
% centered/centred full-screen image, no title:
% This uses the entire whole screen
\newcommand{\maxFrameImage}[2][]{
{\setbeamercolor{background canvas}{bg=blue058}
\begin{frame}[plain]
\begin{changemargin}{-1cm}{-1cm}
\begin{center}
\includegraphics[width=\paperwidth,height=\paperheight,keepaspectratio,#1]
{#2}
\end{center}
\end{changemargin}
\end{frame}
}
}
